\usetikzlibrary{arrows.meta}

\begin{frame}<1>[label=attackPat]{attacking emulation patience}
    \begin{itemize}
    \item looking for unpacked virus in VM
    \item \ldots or other malicious activity
    \item when are you done looking?
    \vspace{.5cm}
    \item<2-> malware solution: \myemph<2>{take too long}
        \begin{itemize}
        \item not hard if emulator uses ``slow'' implementation
        \end{itemize}
    \item<3-> malware solution: \myemph<3>{don't infect consistently}
    \item<4-> malware solution: \myemph<4>{use more memory, etc.}
    \end{itemize}
\end{frame}

\againframe<2>{attackPat}

\againframe<3>{attackPat}

\begin{frame}[fragile,label=prob]{probability}
\lstset{language=C,style=smaller}
    \begin{tikzpicture}
    \node[draw,align=center] (malware) {
\begin{lstlisting}
if (randomNumber() == 4) {
    unpackAndRunEvilCode();
}
\end{lstlisting}
    };
    \node[draw,align=left,above right=1cm of malware] (avCase1) {
        antivirus emulator: \\
        \lstinline|randomNumber() == 3| \\
        \textbf{\color{green!60!black}looks clean!}
    }; 
    \node[draw,align=left,right=1cm of malware]  (avCase2) {
        real execution \#1: \\
        \lstinline|randomNumber() == 2| \\
        \textbf{\color{green!60!black}no infection!}
    }; 
    \node[draw,align=left,below right=1cm of malware] (avCase3) {
        real execution \#$N$: \\
        \lstinline|randomNumber() == 4| \\
        \textbf{\myemph{infect!}}
    }; 
    \draw[ultra thick,dashed,-Latex] (malware) -- (avCase1.west);
    \draw[ultra thick,dashed,-Latex] (malware) -- (avCase2.west);
    \draw[ultra thick,dashed,-Latex] (malware) -- (avCase3.west);
    \end{tikzpicture}
\end{frame}

\againframe<4>{attackPat}

