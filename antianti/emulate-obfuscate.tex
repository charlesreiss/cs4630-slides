\begin{frame}{emulation based obfuscation}
    \begin{itemize}
    \item so far: always producing machine code and running it
    \item analyzing machine code with virtual machine, debugger, etc.
    \vspace{.5cm}
    \item alternate idea: invent a new instruction set
    \item convert program to that instruction set
    \item include interpreter for that instruction set
    \end{itemize}
\end{frame}

\begin{frame}[fragile,label=VirtIntro]{example: Tigress Virtualize transform (1)}
input:
\begin{lstlisting}[style=smaller]
int example(int x) {
    if (x > 10) {
        printf("Yes!\n");
    }   
}
\end{lstlisting}
\begin{itemize}
\item Tigress generates instruction set for stack-based machine
    \begin{itemize}
    \item uses little stack instad of registers for most instructions
    \item same design used by, e.g., Java VM
    \end{itemize}
\item instructions can pop+push from stack for temporaries
\end{itemize}
\end{frame}

\begin{frame}[fragile,label=VirtISA]{example: Tigress Virtualize transform (2)}
\begin{itemize}
\item instruction set for example
\begin{itemize}
\item call OPERAND=funcId with arguments LOCALS[1]
\item pop t1, pop t2, push t1>t2
\item push OPERAND
\item push table[OPERAND]
    \begin{itemize}
    \item different variants for int, string, \ldots
    \end{itemize}
\item pop t1, LOCALS[OPERAND] = t1
\item pop t1, if (t1) goto OPERAND
\item return
\end{itemize}
\item customized for this function
\item each instruction has opcode, variable length (if operands)
\end{itemize}
\end{frame}

\begin{frame}[fragile,label=VirtISA2]{example: Tigress Virtualize transform (3)}
\begin{lstlisting}[style=script]
int example(int x) {
    if (x > 10) {
        printf("Yes!\n");
    }   
}
\end{lstlisting}
\begin{itemize}
\item each line below one ``instruction''
    \begin{itemize}
    \item (actually encoded as part of array of bytes)
    \end{itemize}
\begin{itemize}
\item push OPERAND=10
\item push table[OPERAND=\ldots] (argument \texttt{x})
\item pop t1 pop t2 push t1>t2
\item pop t1, if (t1) goto OPERAND=OUT
\item push table[OPERAND=\ldots] (string \texttt{"Yes!"})
\item pop t1, LOCALS[OPERAND=1] = t1
\item call OPERAND=\ldots (printf) with arguments LOCAL1
\item OUT: \ldots
\end{itemize}
\end{itemize}
\end{frame}

\begin{frame}[fragile,label=VirtISAEx]{example: Tigress emulator}
\begin{lstlisting}
  _1_example_$sp[0] = _1_example_$stack[0];
  _1_example_$pc[0] = _1_example_$array[0];
  while (1) {
    switch (*(_1_example_$pc[0])) {
    ...
    }
  }
\end{lstlisting}
\begin{itemize}
\item pc variable representing emulated stack
    \begin{itemize}
    \item switch statement based on opcode
    \end{itemize}
\item sp variable representing emulated stack
\end{itemize}
\end{frame}

\begin{frame}{effectiveness of this transformation?}
    \begin{itemize}
    \item huge performance impact
    \item can do analysis on new instruction set
        \begin{itemize}
        \item how much more difficult than working with original machine code?
        \end{itemize}
    \item instruction traces still helpful
        \begin{itemize}
        \item about as easy to get record of everything done
        \end{itemize}
    \end{itemize}
\end{frame}
