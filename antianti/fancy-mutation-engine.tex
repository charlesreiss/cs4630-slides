
\begin{frame}{mutation engines}
    \begin{itemize}
    \item tools for writing polymorphic viruses
    \item best: \myemph{no} constant bytes, \myemph{no} ``no-op'' instructions
    \item tedious work to build state-machine-based detector
        \begin{itemize}
        \item ((almost) a regular expression to match it)
        \item apparently done manually
        \item automatable?
        \end{itemize}
    \item (but probably can\ldots)
    \item pattern: used until reliably detected
    \end{itemize}
\end{frame}

\begin{frame}[fragile,label=fancyMut]{fancier mutation}
\begin{lstlisting}[style=script]
Mutate(original_machine_code, new_machine_code) {
    for (instruction in original_code) {
        new_machine_code += ChooseNewCodeFor(instruction)
    }
    FixupJumpsIn(new_machine_code);
}
\end{lstlisting}
    \begin{itemize}
    \item can do mutation on \myemph{generic machine code}
    \vspace{.5cm}
    \item ``just'' need full disassembler
    \item identify both \myemph{instruction lengths} and \myemph{addresses}
    \item hope machine code not written to rely on machine code sizes, etc.
    \item hope to identify \myemph{tables of function pointers}, etc.
    \end{itemize}
\end{frame}

\begin{frame}{fancier mutation}
    \begin{itemize}
    \item also an infection technique
        \begin{itemize}
        \item no ``cavity'' needed --- create one
        \end{itemize}
    \item obviously tricky to implement
        \begin{itemize}
        \item need to fix all executable headers
        \item what if you misparse assembly?
        \item what if you miss a function pointer?
        \end{itemize}
    \item example: Simile virus
    \end{itemize}
\end{frame}


