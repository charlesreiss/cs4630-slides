\begin{frame}<1>[fragile,label=escapeHatchSupportRc]{Rust escape hatch support}
    \begin{itemize}
        \item escape hatch: make new reference-like object
            \begin{itemize}
            \item (\ldots implemented by returning temporary `real' references)
            \end{itemize}
        \item<2> Rc: \verb|Rc<T>| acts like \verb|&T|
        \item callbacks on ownership ending (normally deallocation)
            \begin{itemize}
            \item Rust compiler enforces that ref-like object not in use when free call made
            \item<2> Rc: deallocating \verb|Rc<T>| decrements shared count
            \item<2> Rc: real object only decremented on count == 0
            \end{itemize}
        \item choice of what happens on copy
            \begin{itemize}
            \item<2> Rc: no implicit copy; explicit \verb|clone| operation increments count
            \end{itemize}
    \end{itemize}
\end{frame}
 
\begin{frame}[fragile,label=refCounting]{alternative rule: reference counting}
    \begin{itemize}
    \item keep track of number of references
    \item increment count when making new `clone' of reference
    \item decrement when reference goes away
        \begin{itemize}
        \item Rust borrowing rules will enforce it is not used when this happens
        \end{itemize}
    \item delete when count goes to zero
        \begin{itemize}
        \item Rust automatically calls destructor --- no programmer effort
        \end{itemize}
    \item explicit operation to make new reference
    \item Rust implement with Rc type (``counted reference'')
    \end{itemize}
\end{frame}

\begin{frame}[fragile,label=refCountingEx0a]{Ref Counting Example (0a)}
\begin{minted}[fontsize=\fontsize{9}{10}\selectfont]{Rust}
use std::rc::Rc;

fn main() {
    let s_ref: &String;
    let s1: Rc<String>;
    {
        let s2: Rc<String> = Rc::new(String::from("example"));
        s1 = Rc::clone(&s2);
        s_ref = &*s1;
        println!("{s1} {s_ref} {s2}");
            // example example example
        println!("count={}", Rc::strong_count(&s1));
            // count=2
    }
    println!("count={}", Rc::strong_count(&s1));
        // count=2
    println!("{s1} {s_ref}");
        // example example
}
\end{minted}
\end{frame}

\begin{frame}[fragile,label=refCountingEx0b]{Ref Counting Example (0b)}
\begin{minted}[fontsize=\fontsize{9}{10}\selectfont]{Rust}
use std::rc::Rc;

fn main() {
    let s_ref: &String;
    let s1: Rc<String>;
    {
        let s2: Rc<String> = Rc::new(String::from("example"));
        s1 = Rc::clone(&s2);
        s_ref = &*s1;
        println!("{s1} {s_ref} {s2}");
        println!("count={}", Rc::strong_count(&s1));
    }
    println!("count={}", Rc::strong_count(&s1));
    println!("{s1} {s_ref}");  // ERROR
}
\end{minted}
\begin{tikzpicture}[overlay,remember picture]
    \begin{visibleenv}<2>
    \node[fill=white,draw,very thick,font=\scriptsize,align=left] at (current page.center) {
\begin{lstlisting}[language={},style=script]
error[E0597]: `s2` does not live long enough
  --> src/main.rs:9:19
   |
7  |         let s2: Rc<String> = Rc::new(String::from("example"));
   |             -- binding `s2` declared here
8  |         s1 = Rc::clone(&s2);
9  |         s_ref = &*s2;
   |                   ^^ borrowed value does not live long enough
...
14 |     }
   |     - `s2` dropped here while still borrowed
...
17 |     println!("{}", s_ref);
   |                    ----- borrow later used here
\end{lstlisting}
};
    \end{visibleenv}
\end{tikzpicture}
\end{frame}

\begin{frame}{Rc<> into real ref}
    \begin{itemize}
    \item can turn Rc into real ref
    \item borrowing rule enforces that Rc object stays around
    \vspace{.5cm}
    \item allows us to ensure it's not drop()d while still in use
    \item \ldots and use it with functions that expect `normal' reference
    \end{itemize}
\end{frame}



\begin{frame}[fragile,label=refCountingEx]{Ref Counting Example (1)}
\begin{minted}[fontsize=\fontsize{9}{10}\selectfont]{Rust}
struct Grade {
    score: i32, studentName: String, assignmentName: String,
}
struct Student {
    name: String,
    grades: Vec<Rc<Grade>>,
}
struct Assignment {
    name: String,
    grades: Vec<Rc<Grade>>
}

fn add_grade(student: &mut Student, assignment: &mut Assignment, score: i32) {
    let grade = Rc::new(Grade {
        score: score,
        studentName: student.name.clone(),
        assignmentName: assignment.name.clone(),
    })
    student.grades.push(Rc::clone(&grade));
    assignment.grades.push(Rc::clone(&grade));
    println!("Added grade with score={}", grade.score);
}
\end{minted}
\end{frame}

\begin{frame}{Rc limitations}
    \begin{itemize}
    \item Rc: only gives references to \textit{read-only} objects
        \begin{itemize}
        \item cannot enforce ``only one mutable reference'' rule
        \item \ldots we'll look at doing that next, but needs more bookkeeping
        \end{itemize}
    \item Rc: allows memory leaks via circular references
        \begin{itemize}
        \item correct way to handle ciruclar references: Weak
        \item \ldots but not enforcement that it is used when needed
        \end{itemize}
    \end{itemize}
\end{frame}

\begin{frame}[fragile]{aside: Weak}
\begin{minted}[fontsize=\fontsize{10}{11}\selectfont]{Rust}
struct Foo {
    my_bar: Rc<Bar>,
    ...
}
struct Bar {
    my_foo: Weak<Foo>.
    ...
}
...
let bar: Bar = ...;
...
match bar.my_foo.upgrade() {
    Some(foo_rc) => {
        // foo_rc is an Rc<Foo>
        ...
    },
    None => {
        // the foo object was deleted
    }
}
\end{minted}
\end{frame}


\againframe<2>{escapeHatchSupportRc}

\begin{frame}[fragile,label=rcImplA]{Rc implementation (approx) (0)}
\begin{minted}[fontsize=\fontsize{10}{11}\selectfont]{Rust}
struct RcInner<T: ?Sized> {
    strong: Cell<usize>,    // <-- count of Rc<T>s pointing to this
    weak: Cell<usize>,      // <-- count of Weak<T>s pointing to this
    value: T,               // <-- actual data
}

pub struct Rc<T: ?Sized> {
    ptr: NonNull<RcInner<T>>,
    phantom: PhantomData<RcInner<T>>, // <- so compiler infers what operations are safe better
}
\end{minted}
\begin{itemize}
\item NonNull = raw pointer wrapper
\item Cell = way to have mutable field of immutable object
\end{itemize}
\end{frame}



\begin{frame}[fragile,label=rcImplA]{Rc implementation (approx) (1)}
\begin{minted}[fontsize=\fontsize{10}{11}\selectfont]{Rust}
impl<T: ?Sized> Clone for Rc<T> {
    ... 
    fn clone(&self) -> Rc<T> {
        self.inc_strong(); // <-- increment reference count
        Rc { ptr: self.ptr }
    }
}
\end{minted}
\end{frame}

\begin{frame}[fragile,label=rcImplB]{Rc implementation (approx) (2)}
\begin{minted}[fontsize=\fontsize{10}{11}\selectfont]{Rust}
unsafe impl<#[may_dangle] T: ?Sized> Drop for Rc<T> {
    ...
    fn drop(&mut self) { // <-- compilers calls on deallocation
        unsafe {
            self.inner().dec_strong();
            if self.inner().strong() == 0 {
                self.drop_slow();
            }
        }
    }
    ...
}
\end{minted}
\end{frame}

\begin{frame}[fragile,label=rcImplC]{Rc implementation (approx) (3)}
\begin{minted}[fontsize=\fontsize{10}{11}\selectfont]{Rust}
impl<T: ?Sized> Deref for Rc<T> {
    type Target = T;

    #[inline(always)]
    fn deref(&self) -> &T {
        &self.inner().value
    }
}
\end{minted}
\begin{itemize}
\item observation: returned reference still has lifetime
\item compiler will enforce that extracted reference only used when Rc object valid
\end{itemize}
\end{frame}

