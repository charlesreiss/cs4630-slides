\usetikzlibrary{positioning,shapes.callouts}
\begin{frame}[fragile,label=rustHelloWorld1]{simple Rust syntax (1)}
\begin{minted}{Rust}
fn main() {
    println!("Hello, World!\n");
}
\end{minted}
\end{frame}

\begin{frame}[fragile,label=rustHelloWorld2]{simple Rust syntax (2)}
\begin{minted}[fontsize=\fontsize{10}{11}]{Rust}
fn timesTwo(number: i32) -> i32 {
    return number * 2;
}
\end{minted}
\end{frame}

\begin{frame}[fragile,label=rustHelloWorld3]{simple Rust syntax (3)}
    \begin{minted}[fontsize=\fontsize{10}{11}]{Rust}
struct Student {
    name: String,
    id: i32,
}

fn get_example_student() -> Student {
    return Student {
        name: String::from("Example Fakelastname"),
        id: 42,
    };
}
\end{minted}
\end{frame}

\begin{frame}[fragile,label=rustHelloWorld4]{simple Rust syntax (4)}
    \begin{minted}[fontsize=\fontsize{10}{11}\selectfont,escapeinside=||]{Rust}
fn factorial(number: i32) -> i32 {
    let mut|\tikzmark{mut}| result = 1;
    let mut index = 1;
    while index <= number {
        result *= index;
        index = index + 1;
    }
    return result;
}
\end{minted}
    \begin{tikzpicture}[overlay,remember picture]
        \coordinate (box) at (current page.center);
        \begin{visibleenv}<2>
            \node[my callout=mut,anchor=center,align=left] at ([yshift=2cm]box) {
                ``result'' is a mutable variable \\
                type automatically inferred as i32 (32-bit int)
            };
        \end{visibleenv}
    \end{tikzpicture}
\end{frame}

