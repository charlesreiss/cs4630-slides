\begin{frame}[fragile,label=whyLifetime1]{why lifetimes? (1)}
\begin{minted}[fontsize=\fontsize{12}{13}]{Rust}
let x = vec![1, 2, 3, 4];
let mut q = &x[1];
{
    let mut r = &x[1];
    let y = vec![5, 6, 7, 8];
    if random() == 0 {
        r = &y[1]; // SHOULD BE FINE
        q = &y[1]; // SHOULD BE ERROR
    }
    println!("{}", *r);
}
println!("{}", *q);
\end{minted}
\begin{itemize}
\item need to prevent \texttt{q} referring to values that live too long
\end{itemize}
\end{frame}

\begin{frame}[fragile,label=whyLifetime2]{why lifetimes? (2)}
\begin{minted}[fontsize=\fontsize{12}{13}]{Rust}
fn mystery(ptr: &i32, vec: &Vec<i32>) -> &i32 {...}

fn example() {
    ...
    let mut x = vec![1, 2, 3, 4];
    let mut q = &x[1];
    {
        let mut y = vec![5, 6, 7, 8];
        q = mystery(q, &y);
    }
    println!("{}", *q);
}
\end{minted}
\begin{itemize}
\item question: should assignment to be q from mystery be allowed?
\end{itemize}
\end{frame}
