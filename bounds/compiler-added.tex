
\begin{frame}[fragile,label=fortifyMemCpyIntro]{adding bounds checking}
\lstset{language=C,style=small}
\begin{lstlisting}
char buffer[42];
memcpy(buffer, attacker_controlled, len);
\end{lstlisting}
    \begin{itemize}
        \item couldn't compiler add check for \texttt{len}
        \item modern Linux: it does
    \end{itemize}
\end{frame}

\begin{frame}[fragile,label=boundsChecking]{added bounds checking}
\lstset{language=C,style=small}
\begin{lstlisting}
char buffer[42];
memcpy(buffer, attacker_controlled, len);
\end{lstlisting}
\lstset{language=myasm,style=small}
\begin{lstlisting}
    subq   $72, %rsp
    leaq   4(%rsp), %rdi
    movslq len, %rdx
    movq   attacker_controlled, %rsi
    movl   $42, %ecx
    call   __memcpy_chk
\end{lstlisting}
    \begin{itemize}
        \item length \texttt{42} passed to \texttt{\_\_memcpy\_chk}
    \end{itemize}
\end{frame}

\begin{frame}{\_FORTIFY\_SOURCE}
    \begin{itemize}
        \item Linux C standard library + GCC features
        \item adds automatic checking to a bunch of string/array functions
        \item also printf (disable \texttt{\%n} unless format string is a constant)
        \vspace{.5cm}
        \item often enabled by default
        \item GCC options:
            \begin{itemize}
                \item \texttt{-D\_FORTIFY\_SOURCE=1} --- enable (backwards-compatible only)
                \item \texttt{-D\_FORTIFY\_SOURCE=2} --- enable (full)
                \item \texttt{-U\_FORTIFY\_SOURCE} --- disable
            \end{itemize}
    \end{itemize}
\end{frame}

\begin{frame}[fragile,label=fortifySrcExample]{bounds checking will happen...}
will add checks (gcc 9.3 \textbf{-O2})
\begin{lstlisting}[language=C,style=smaller]
void example1() {
    char dest1[1024]; memcpy(dest1, ...); ...
}
char dest2[1024];
void example2() {
    memcpy(dest2, ...); ...
}
void example3() {
    char *p = &dest2[4]; memcpy(p, ...); ...
}
\end{lstlisting}
\end{frame}

\begin{frame}[fragile,label=fortifySrcExmaple2]{bounds checking won't happen...}
will not add check (gcc 9.3 \textbf{-O2})
\begin{lstlisting}[language=C,style=smaller]
char dest2[1024];
void example4() {
    char *p = &dest2[mystery()]; memcpy(p, ...); ...
}
\end{lstlisting}
adds check for size 1024 (max possible size):
\begin{lstlisting}[language=C,style=smaller]
char dest2[1024];
void example5() {
    char dest3[128];
    char *p = dest2;
    if (mystery()) p = dest3;
    memcpy(p, ...); ...
}
\end{lstlisting}
\end{frame}
