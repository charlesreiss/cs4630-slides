\begin{frame}{looks like canaries? (1)}
    \begin{itemize}
    \item what attacks does this stop that canaries don't?
    \vspace{.5cm}
    \item<2-> ID does not need to be secret!
        \begin{itemize}
        \item<2-> assuming non-executable writeable memory, no!
        \item<2-> attacker can't write new places for return to go
        \end{itemize}
    \item<3-> avoids ``stack pivoting'' attacks
        \begin{itemize}
        \item attacker can't make stack pointer point to wrong part of stack\ldots
        \item and expect it to return differently
        \end{itemize}
    \end{itemize}
\end{frame}

\begin{frame}[fragile,label=cfiLikeCanariesP2]{looks like canaries? (2)}
    \begin{itemize}
    \item what attacks does this NOT stop that canaries do?
    \item example: SortList can be called from \texttt{Innocent}, \\ then return from \texttt{Dangerous}
        \begin{itemize}
        \item assumption: attacker can overwrite return address at right time (running on another core? problem with sortFunc1?)
        \end{itemize}
    \end{itemize}
\begin{minipage}{0.5\textwidth}
\begin{lstlisting}[language=C++,style=smaller]
void Innocent() {
  ...
  SortList(someList1,
           sortFunc1);
  Use(someList1);
  ...
}
\end{lstlisting}
\end{minipage}
\begin{minipage}{0.5\textwidth}
\begin{lstlisting}[language=C++,style=smaller]
void Dangerous() {
  ...
  SortList(someList2,
           sortFunc2);
  UseDangerously(someList2);
  ...
}
\end{lstlisting}
\end{minipage}
\end{frame}
