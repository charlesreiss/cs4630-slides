\usetikzlibrary{positioning}
\begin{frame}{linking}
    \begin{tikzpicture}
    \node[draw,font=\tt,very thick] (theCall) {callq printf};
    \node[draw,font=\tt,below=5cm of theCall,very thick] (theCallResolved) {callq 0x458F0};
    \draw[very thick,-Latex] (theCall) -- (theCallResolved);
    \end{tikzpicture}
\end{frame}

\begin{frame}{static v. dynamic linking}
    \begin{itemize}
    \item static linking --- linking \myemph{to create executable}
    \item dynamic linking --- linking \myemph{when executable is run}
    \vspace{.5cm}
    \item<2> conceptually: no difference in how they work
    \item<2> reality --- very different mechanisms
    \end{itemize}
\end{frame}

\begin{frame}{linking data structures}
    \begin{itemize}
    \item symbol table: {\tt name} $\Rightarrow$ (section, offset)
        \begin{itemize}
        \item example: {\tt main:} in assembly adds symbol table entry for {\tt main}
        \end{itemize}
    \item relocation table: offset $\Rightarrow$ (name, kind)
        \begin{itemize}
        \item example: {\tt call printf} adds relocation for name {\tt printf}
        \item kind depends on how instruction encodes address
        \end{itemize}
    \end{itemize}
\end{frame}

