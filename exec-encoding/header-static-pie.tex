\providecommand{\myemphTwo}[1]{\myemph<2>{#1}}
\providecommand{\myemphTwoB}[1]{\myemph<2>{\textbf<2>{#1}}}
\providecommand{\myemphThree}[1]{\myemph<3>{#1}}
\providecommand{\myemphFour}[1]{\myemph<4>{#1}}
\providecommand{\myemphFive}[1]{\myemph<5>{#1}}
\providecommand{\myemphSix}[1]{\myemph<6>{#1}}
\providecommand{\myemphSeven}[1]{\myemph<7>{#1}}

\begin{frame}[fragile,label=elfExOver1]{position-independent ELF example}
    \begin{itemize}
    \item compiled+linked ``Hello, World'' with {\tt gcc -static-pie}
    \item PIE = position-independent executable
    \end{itemize}
\begin{Verbatim}[commandchars=\\\{\},fontsize=\small]
a.out:     file format elf64-x86-64
a.out
architecture: i386:x86-64, flags 0x00000150:
HAS_SYMS, \myemphTwo{DYNAMIC}, D_PAGED
start address 0x0000000000009f90

Program Header:
...

\myemphTwo{Dynamic Section:}
...

Sections:
...
\end{Verbatim}
\begin{tikzpicture}[overlay,remember picture]
\coordinate (loc) at ([yshift=1cm]current page.south);
\tikzset{
    note/.style={draw,very thick,anchor=south,at=(loc)},
}
\node[note] {
    DYNAMIC instead of EXEC_P \\
    reuses dynamic linking mechanism
};
\end{tikzpicture}
\end{frame}
