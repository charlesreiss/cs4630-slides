
\begin{frame}{malware logistics: how?}
    \begin{itemize}
    \item what are they written in?
    \end{itemize}
\end{frame}

\begin{frame}{malware languages (1)}
    \begin{itemize}
    \item assembly language/machine code
        \begin{itemize}
        \item hand-coded or partially hand-coded
        \end{itemize}
    \vspace{.5cm}
    \item vulnerabilities deal with \myemph{machine code/memory layout}
    \item often better for hiding malware from anti-malware tools
    \end{itemize}
\end{frame}

\begin{frame}{malware languages (2)}
    \begin{itemize}
    \item high-level scripting languages
        \begin{itemize}
        \item fast prototyping
        \item maintainability/efficiency not priority
        \item sometimes malicious scripts
        \item non-machine-code parts can use anything!
        \end{itemize}
    \item sometimes specialized ``toolkits'' \\
          example: Virus Construction Kit
    \end{itemize}
\end{frame}



