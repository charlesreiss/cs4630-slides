

\begin{frame}{vulnerabilities}
    \begin{itemize}
    \item for viruses, worms
    \item for trojans + PUP that do more than is supposed to do be allowed
        \begin{itemize}
        \item e.g. getting location information without ``permission''
        \end{itemize}
    \vspace{.5cm}
    \item software \myemph{vulnerability}
    \vspace{.5cm}
    \item unintended program behavior \\ that can be used by an adversary
    \end{itemize}
\end{frame}

\begin{frame}{vulnerability example}
    \begin{itemize}
    \item website able to install software without prompting
    \item \myemph{not intended} behavior of web browser
    \end{itemize}
\end{frame}


\begin{frame}{software vulnerability classes (1)}
    \begin{itemize}
    \item \myemph{memory safety} bugs
        \begin{itemize}
        \item problems with pointers
        \item big topic in this course
        \end{itemize}
    \item ``injection'' bugs --- \myemph{type confusion}
        \begin{itemize}
        \item commands/SQL within name, label, etc.
        \end{itemize}
    \item integer overflow/underflow
    \item \ldots
    \end{itemize}
\end{frame}

\begin{frame}{software vulnerability classes (2)}
    \begin{itemize}
    \item not checking inputs/permissions
        \begin{itemize}
        \item \url{http://webserver.com/../../../../file-I-shouldn't-get.txt}
        \end{itemize}
    \item almost any ``undefined behavior'' in C/C++
    \item synchronization bugs: time-to-check to time-of-use
    \item \ldots{} more?
    \end{itemize}
\end{frame}

\begin{frame}{vulnerability versus exploit}
    \begin{itemize}
    \item exploit --- something that uses a vulnerability to do something
    \item proof-of-concept --- something = demonstration the exploit is there
        \begin{itemize}
        \item example: open a calculator program
        \end{itemize}
    \end{itemize}
\end{frame}


