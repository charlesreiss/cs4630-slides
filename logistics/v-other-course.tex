\begin{frame}{DADT's awkward history}
    \begin{itemize}
    \item Davidson conceived as course on \textit{malware}
        \begin{itemize}
        \item including exploits likely used by malware
        \end{itemize}
    \item \ldots until 2019 was our only regularly offered security elective
    \vspace{.5cm}
    \item but often other faculty taught it as general security class
        \begin{itemize}
        \item now have 3710 for that purpose
        \end{itemize}
    \item also have Network Security (CS 4760)
        \begin{itemize}
        \item nominally cryptographic protocols, data integrity, attack surfaces
        \item possibly also covers scanning, web security, \ldots
        \end{itemize}
    \end{itemize}
\end{frame}

\begin{frame}{malware changes}
    \begin{itemize}
    \item historically, a lot of `self-spreading' malware
        \begin{itemize}
        \item viruses, worms, \ldots
        \item (we'll discuss what these terms mean later)
        \end{itemize}
    \item these days, not the most common ways to get malware
        \begin{itemize}
        \item network-based exploits installing malicious software
        \item malicious/unwanted software distributed through app stores/etc.
        \end{itemize}
    \end{itemize}
\end{frame}

\begin{frame}{exploit changes}
    \begin{itemize}
    \item historically: memory-unsafety exploits
    \item biggest source of insecure software for a long time
    \item probably still biggest, but\ldots
    \vspace{.5cm}
    \item memory safety slowly becoming less of a problem
    \item how web browsers/mobile OSes/etc. work more important
    \end{itemize}
\end{frame}
