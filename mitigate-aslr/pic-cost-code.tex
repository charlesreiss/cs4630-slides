

\newcommand{\myemphTwo}[1]{\myemph<2>{#1}}%
\newcommand{\myemphThree}[1]{\myemph<3>{#1}}%
\newcommand{\myemphFour}[1]{\myemph<{#1}}%
\newcommand{\antiEmphThree}[1]{{\color<3>{blue}#1}}%
\newcommand{\antiEmphFour}[1]{{\color<4>{blue}#1}}%
\begin{frame}[fragile,label=PIEjtasm]{PIE jump-table}
\begin{tikzpicture}
\node[anchor=north east] (code) at (-1, 0) {
\lstset{
    language=myasm,
    style=smaller,
    escapeinside=~~,
}
\begin{lstlisting}
foo:
  movl	 $3, %eax
  cmpq	 $5, %rdi
  ja	 retDefault
  leaq	 ~\myemphTwo{jumpTable(\%rip)},~%rax
  movslq ~\myemphTwo{(\%rax,\%rdi,4)},~%rdx
  addq	 %rdx, %rax
  ~\myemphTwo{\bfseries jmp}~   ~\myemphTwo{*\%rax}~
returnTwo:
  movl  $2, %eax
  ret
returnOne:
  movl  $1, %eax
defaultCase:
  ret
\end{lstlisting}
};
\node[anchor=north west] (code2) at ([xshift=2cm]code.north east) {
\lstset{
    language={},
    style=smaller,
    escapeinside=~~,
}
\begin{lstlisting}
  .section	.rodata
jumpTable:
  .long	returnOne-jumpTable
  .long	returnTwo-jumpTable
  .long	returnOne-jumpTable
  .long	returnTwo-jumpTable
  .long	returnOne-jumpTable
  .long	returnOne-jumpTable
\end{lstlisting}
};
\end{tikzpicture}
\end{frame}

\begin{frame}[fragile,label=PIEjt]{PIE jump-table}
\begin{tikzpicture}
\node[anchor=north east] (code) at (-1, 0) {
\lstset{
    language=myasm,
    style=smaller,
    escapeinside=~~,
}
\begin{lstlisting}
~\myemphTwo{00000000000007ab}~ <foo>:
b8 03 00 00 00       	mov    $0x3,%eax
48 83 ff 05             cmp    $0x5,%rdi
77 1b                	ja     7d0 <foo+0x25>
48 8d 05 ab 00 00 00 	lea    ~\myemphThree{0xab(\%rip),}~%rax        # 868
48 63 14 b8          	movslq ~\myemphThree{(\%rax,\%rdi,4),}~%rdx
48 01 d0             	add    %rdx,%rax
ff e0                	jmpq   *%rax
b8 02 00 00 00       	mov    $0x2,%eax
c3                   	retq   
b8 01 00 00 00       	mov    $0x1,%eax
c3                	retq
...
~\textit{\myemphThree{@ 868}}:~ -156 /* offset */
~\textit{\myemphThree{@ 870}}:~ -162
...
\end{lstlisting}
};
\end{tikzpicture}
\end{frame}

\begin{frame}{added cost}
\begin{itemize}
    \item replace \texttt{jmp *jumpTable(,\%rdi,8)}
    \vspace{.5cm}
    \item with:
        \item \texttt{lea} (get table address --- with relative offset)
        \item \texttt{movslq} (do table lookup of offset)
        \item \texttt{add} (add to base)
        \item \texttt{jmp} (to computed base)
\end{itemize}

\end{frame}

\begin{frame}[fragile,label=x86Worse]{32-bit x86 is worse}
    \begin{itemize}
    \item no relative addressing for {\tt mov}, {\tt lea}, \ldots
    \item even changes ``stubs'' for printf:
    \end{itemize}
\lstset{
    language=myasm,
    style=smaller,
    escapeinside=~~,
}
\begin{lstlisting}
// BEFORE: (fixed addresses)
08048310 <__printf_chk@plt>:
 8048310: ff 25 ~\myemphTwo{10 a0 04 08}~  jmp    *0x804a010
    /* 0x804a010 == global offset table entry */

// AFTER: (position-independent)
00000490 <__printf_chk@plt>:
 490:	ff a3 10 00 00 00    jmp    *0x10~\myemphThree{(\%ebx)}~
    /* %ebx --- address of global offset table */
    /* needs to be set by caller */
\end{lstlisting}
\end{frame}

