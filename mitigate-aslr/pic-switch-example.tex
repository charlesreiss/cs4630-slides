
{
\providecommand{\myemphTwo}[1]{\myemph<2>{#1}}%
\providecommand{\myemphThree}[1]{\myemph<3>{#1}}%
\providecommand{\myemphFour}[1]{\myemph<{#1}}%
\providecommand{\antiEmphThree}[1]{{\color<3>{blue}#1}}%
\providecommand{\antiEmphFour}[1]{{\color<4>{blue}#1}}%
\begin{frame}[fragile,label=hardCodedAsm]{hard-coded addresses? (64-bit)}
\lstset{language=C,style=small}
\begin{tikzpicture}
\node[anchor=north east] (code) at (-1, 0) {
\begin{lstlisting}
int foo(long n) {
    switch (n) {
    case 0: 
    case 2:
    case 4:
    case 5:
        return 1;
    case 1:
    case 3:
        return 2;
    default:
        return 3;
    }
}
\end{lstlisting}
};
\draw[thick] (-.75, 0) -- (-.75,-7);
\node[anchor=north west,align=left] (disasm) at (-.5, 0) {
\lstset{
    language=myasm,
    style=smaller,
    escapeinside=~~,
}
\begin{lstlisting}
foo:
	movl	$3, %eax
	cmpq	$5, %rdi
	ja	defaultCase
	jmp	*lookupTable(,%rdi,8)
        /* code for defaultCase, returnOne, returnTwo */
        ...
	.section	.rodata
lookupTable: /* read-only pointers: */
	.quad	returnOne
	.quad	returnTwo
	.quad	returnOne
	.quad	returnTwo
	.quad	returnOne
	.quad	returnOne
\end{lstlisting}
};
\end{tikzpicture}
\end{frame}


\begin{frame}[fragile,label=hardCodedDisasm]{hard-coded addresses? (64-bit)}
\lstset{language=C,style=small}
\begin{tikzpicture}
\node[anchor=north east] (code) at (-1, 0) {
\begin{lstlisting}
int foo(long n) {
    switch (n) {
    case 0: 
    case 2:
    case 4:
    case 5:
        return 1;
    case 1:
    case 3:
        return 2;
    default:
        return 3;
    }
}
\end{lstlisting}
};
\draw[thick] (-.75, 0) -- (-.75,-7);
\node[anchor=north west,align=left] (disasm) at (-.5, 0) {
\lstset{
    language=myasm,
    style=smaller,
    escapeinside=~~,
}
\begin{lstlisting}
~\textit{400570 <foo>}~:
b8 03 00 00 00    mov    $0x3,%eax
48 83 ff 05       cmp    $0x5,%rdi
        /* jump to defaultCase: */
77 ~\antiEmphThree{12}~            ja     ~\antiEmphThree{0x40058d}~
        /* lookup table jump: */
ff 24 fd
~\myemphTwo{18 06 40 00}~      jmpq   *~\myemphTwo{0x400618}(,\%rdi,8)~
...
 /* lookupTable @ 0x400618 */
~\textit{@ 400618:} \myemphTwo{0x400588}~ /* returnOne */
~\textit{@ 400620:} \myemphTwo{0x400582}~ /* returnTwo */
~\textit{@ 400628:} \myemphTwo{0x400588}~
~\textit{@ 400630:} \myemphTwo{0x400582}~
\end{lstlisting}
};
\end{tikzpicture}
\end{frame}

}
