\usetikzlibrary{positioning,shapes.callouts}
\begin{frame}{exploit mitigations}
    \begin{itemize}
    \item idea: turn vulnerablity to something less bad
    \item e.g. crash instead of machine code execution
    \vspace{.5cm}
    \item many of these targetted at buffer overflows
    \end{itemize}
\end{frame}

\begin{frame}{mitigation agenda}
    \begin{itemize}
        \item we will look briefly at one mitigation --- stack canaries
        \item then look at exploits that don't care about it
        \item then look at more flexible mitigations
        \item then look at more flexible exploits
    \end{itemize}
\end{frame}

\begin{frame}<1>[label=mitigationPrios]{mitigation priorities}
    \begin{itemize}
    \item \myemph<2>{effective?}\tikzmark{effective} does it actually stop the attacker?
    \item fast? how much does it hurt performance?
    \item generic? does it require a recompile? rewriting software?
    \end{itemize}

    \begin{tikzpicture}[overlay,remember picture]
        \begin{visibleenv}<2>
        \node[my callout=effective,anchor=center,align=center] at (current page.center) {
            recurring theme: stop stack smashing, \\ but not other buffer overflows
        };
        \end{visibleenv}
    \end{tikzpicture}
\end{frame}
