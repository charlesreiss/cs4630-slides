\graphicspath{{./figures/}}
\usepackage{pdfpages}
\title{}
\date{}
\begin{document}
\begin{frame}
    \titlepage
\end{frame}

{
\setbeamercolor{background canvas}{bg=blue!40!black,fg=blue!10!white}
\setbeamercolor{normal text}{bg=blue!40!black,fg=blue!10!white}
\setbeamercolor{itemize/enumerate body}{fg=white}
\setbeamercolor{itemize/enumerate subbody}{fg=white}
\setbeamercolor{titlelike}{bg=blue!40!black,fg=blue!10!white}
\begin{frame}<1|handout:1>[noframenumbering]{Changelog}
    \begin{itemize}
    \item 26 April 2021: chroot ls: add first missing chroot command; consistently use /tmp/example
    \item 26 April 2021: mount namespaces API: add mount commands that would be needed to actually run ls
    \item 26 April 2021: Linux namespaces (3): mention that a chroot will happen in addition to mounting things; be more explicit about how user namespaces change things, opting out of sudo, effectively
    \end{itemize}
\end{frame}
}




\begin{frame}{last time}
    \begin{itemize}
    \item Rust options for ``smart pointer'' types
    \item other language enforcement ideas: e.g. constant-time
    \vspace{.5cm}
    \item what programs are allowed to do vs. what programs need
        \begin{itemize}
        \item goal: ``least privilege''
        \end{itemize}
    \item privilege separation as application design
        \begin{itemize}
        \item examples: Chrome, OpenSSH
        \end{itemize}
    \item mechanism: system call filtering
    \item problem: more precise system call filtering?
    \end{itemize}
\end{frame}

\begin{frame}{challenge/final logistics}
    \begin{itemize}
    \item CHALLENGE assignment plan:
        \begin{itemize}
        \item 9\% of final grade (= approx. 2 homeworks)
        \item 7 ``make this program output X'' problems
        \item solve any 5 for full credit
        \item expect to release no later than 30 April
        \item due 12 May 2021 @ 9pm
        \end{itemize}
    \item FINAL quiz
        \begin{itemize}
        \item 6\% of final grade
        \item focused on material that can't be covered by CHALLENGE
        \item intending something that won't take longer than 90 minutes 
            \begin{itemize}
            \item (it varies how good I am meeting that target)
            \end{itemize}
        \item released 12 May 2021 @ 9pm
        \item due 13 May 2021 @ 9pm
        \item (our official final period: 2-5pm 13 May)
        \end{itemize}
    \end{itemize}
\end{frame}

\begin{frame}{challenge assistance policy}
    \begin{itemize}
    \item Please do not discuss or expect TAs (or me) to answer questions about what strategy you should apply to particular challenges. You are responsible for figuring this out yourself.
    \item You may, however, ask TAs or share general information about how to identify whether an exploit technique is applicable to a particular program or about how to apply an exploit technique to other executables.
    \vspace{.5cm}
    \item We will supply reference solutions to homework assignments.
    \end{itemize}
\end{frame}

\section{more fine-grained filtering?}
\begin{frame}{Linux system call filtering: detailed}
    \begin{itemize}
    \item Linux supports more fine-grained system call filtering
    \item using BPF (Berkeley Packet Filter) programming language
        \begin{itemize}
        \item compiled in the kernel to assembly to check system calls
        \end{itemize}
    \vspace{.5cm}
    \item can check system call argument values, but\ldots
        \begin{itemize}
        \item problems with pointer arguments
        \item too many system calls
        \end{itemize}
    \end{itemize}
\end{frame}

\begin{frame}[fragile,label=open]{Linux system call: open}
\begin{lstlisting}[language=C,style=small]
open("foo.txt", O_RDONLY);
\end{lstlisting}
\begin{itemize}
\item parameters:
    \begin{itemize}
    \item system call number: 2 (``open'')
    \item argument 1: 0x7fffe983 (address of string ``foo.txt'')
    \item argument 2: 0 (value of ``\texttt{O\_RDONLY}'')
    \end{itemize}
\item very problematic to filter using BPF interface
\vspace{.5cm}
\item can deal with using `ptrace' --- Linux debugging interface
    \begin{itemize}
    \item BPF can trigger something like a debugger breakpoint
    \item breakpoint wakes up monitor program (attached like debugger)
    \item `monitor' program can perform system call on program's behalf
    \end{itemize}
\end{itemize}
\end{frame}

\begin{comment}
\begin{frame}[fragile,label=openRuleP1]{lots of ways to open (1)}
\begin{itemize}
\item let's say we want to disallow:
\end{itemize}
\begin{lstlisting}[language=C,style=smaller]
open("/dev/keyboard", O_RDONLY);
\end{lstlisting}
\begin{itemize}
\item problem 1: some other ways of doing that?
\begin{lstlisting}[language=C,style=smaller]
chdir("/dev");
open("keyboard, O_RDONLY);

open("../../../../../../dev/keyboard", O_RDONLY);

symlink("/dev", "/tmp/foo");
open("/tmp/foo/keyboard", O_RDONLY);
\end{lstlisting}
\end{frame}

\begin{frame}[fragile,label=openRuleP2]{lots of ways to open (2)}
\begin{itemize}
\item let's say we want to disallow:
\end{itemize}
\begin{lstlisting}[language=C,style=smaller]
open("/dev/keyboard", O_RDONLY);
\end{lstlisting}
\begin{itemize}
\item problem 2: filter language doesn't allow reading pointers
    \begin{itemize}
    \item string is passed via pointer
    \end{itemize}
\item problem 3: string can be changed from another core
    \begin{itemize}
    \item between when filter runs and when syscall runs
    \end{itemize}
\end{itemize}
\end{frame}

\begin{frame}[fragile,label=openRuleP3]{lots of ways to open (3)}
\begin{itemize}
\item let's say we want to disallow:
\end{itemize}
\begin{lstlisting}[language=C,style=smaller]
open("/dev/keyboard", O_RDONLY);
\end{lstlisting}
\begin{itemize}
\item problem 4: several other syscalls (that might be used innocently)
    \begin{itemize}
    \item openat, open\_by\_handle\_at
    \item would need to write additional filter rules
    \item \ldots or break programs that aren't trying to violate rule
    \end{itemize}
\end{itemize}
\end{frame}

\begin{frame}[fragile,label=howManySystemCalls]{Linux system calls}
\begin{itemize}
\item x86-64 linux: \myemph{313 system calls}
\item opening a file:
    \begin{itemize}
    \item open (number 2)
    \item openat (number 257)
    \item open\_by\_handle\_at (number 304)
    \end{itemize}
\item coordinating between threads (for using multiple cores):
    \begin{itemize}
    \item rt\_sigaction (number 13)
    \item rt\_sigprocmask (number 14)
    \item rt\_sigreturn (number 15)
    \item tkill (number 200)
    \item futex (number 202)
    \item set\_robust\_list (number 273)
    \item get\_robust\_list (number 274)
    \item more?
    \end{itemize}
\end{itemize}
\end{frame}
\end{comment}


% FIXME: without seccomp, ptrace-based filtering

\subsection{applied to VLC?}
\begin{frame}{filtering system calls?}
    \begin{itemize}
    \item example: video player VLC playing a local file on my laptop
    \item uses \myemph{73 unique kinds of system calls}
    \item opens many files that \myemph{are not the video file}
        \begin{itemize}
        \item libraries
        \item fonts
        \item configuration files
        \item translations of messages
        \end{itemize}
    \vspace{.5cm}
    \item can I limit the files my video player can read?
    \item how do I come up with a useful filter?
    \end{itemize}
\end{frame}


\subsection{shared services?}
\begin{frame}{shared services?}
    \begin{itemize}
    \item often programs do operations by talking to ``server'' program
        \begin{itemize}
        \item example: GUI management on Linux (X11 or Wayland), OS X (WindowServer)
        \item example: mixing sound from multiple applications
        \item \ldots
        \end{itemize}
    \item whole extra set of calls to sanitize
        \begin{itemize}
        \item when to allow ``get keyboard input'' for GUI
        \item when to allow ``get microphone input'' for sound manager
        \item making sure one isn't manipulating wrong program's windows?
        \end{itemize}
    \item also, server programs might have security problems
        \begin{itemize}
        \item common ``sandbox escape''
        \end{itemize}
    \end{itemize}
\end{frame}



\subsection{pro/con shared services}
\begin{frame}{exercise: app confinement options}
    \begin{itemize}
    \item sandboxed applications want to access display server
    \item which option seems best for security/performance?
    \begin{itemize}
    \item A. proxy for protocol display server supports natively that filters display calls
    \item B. custom protocol that sends bitmaps + receives inputs, plus copy of display server runs with application
    \item C. divide application into UI and non-UI part, sandbox just the non-UI part
    \item D. have application take over screen when running, give its own display server
    \end{itemize}
    \end{itemize}
\end{frame}


\subsection{SELinux}
\begin{frame}{SELinux}
    \begin{itemize}
    \item \textbf{S}ecurity \textbf{E}nhanced Linux
    \item ``Mandatory Access Control'' system for the Linux
        \begin{itemize}
        \item mandatory: can be configured to require enumeration of files programs can access
        \item (versus normally: specify what files programs can't access)
        \end{itemize}
    \item not necessairily run in mandatory control mode
    \vspace{.5cm}
    \item programs run in particular ``domain''
    \item objects (files, port numbers, other programs, etc.) can be assigned labels
    \item rules about what labels programs are allowed to access
    \end{itemize}
\end{frame}

\begin{frame}[fragile,label=selinuxLabels1]{viewing/assigning labels (1)}
\begin{Verbatim}
$ ls -Z /var/log/lastlog
-rw-r--r--. root root system_u:object_r:lastlog_t:s0   /var/log/lastlog
\end{Verbatim}
\begin{itemize}
\item above: default Red Hat Linux/CentOS configuration
\item system user
\item object role
\item lastlog type
\end{itemize}
\begin{Verbatim}
$ chcon --type=newtype_t some_file
\end{Verbatim}
\end{frame}

\begin{frame}[fragile,label=selinuxLabels2]{assigning labels (2)}
\begin{itemize}
\item labels via: ``file context mapping''
\end{itemize}
\begin{Verbatim}
$ semanage fcontext --add --type web_files_t '/var/www/html(/.*)?'
$ restorecon -R -v /var/www/html
\end{Verbatim}
\begin{itemize}
\item pattern matching rules set \textit{default} labels
\item restorecon --- switch to default labels, applying rules
\end{itemize}
\end{frame}

\begin{frame}[fragile,label=selinuxRules1]{assigning rules}
\begin{itemize}
\item subset of default rules for Apache httpd (webserver):
\end{itemize}
\begin{Verbatim}[fontsize=\small]
define(`read_files_pattern',`
  allow $1 $2:dir search_dir_perms;
  allow $1 $3:file read_file_perms;
')
...
define(`read_lnk_files_pattern',`
  allow $1 $2:dir search_dir_perms;
  allow $1 $3:lnk_file read_lnk_file_perms;
')
...
allow httpd_t httpd_config_t:dir list_dir_perms;
read_files_pattern(httpd_t, httpd_config_t, httpd_config_t)
read_lnk_files_pattern(httpd_t, httpd_config_t, httpd_config_t)
\end{Verbatim}
\begin{itemize}
\item httpd\_t: `type' for webserver process
\end{itemize}
\end{frame}



\subsection{versus capability-type approach}
\begin{frame}{changing what programs can name}
    \begin{itemize}
    \item seccomp, separate users: program tries to access X, checks if allowed
    \vspace{.5cm}
    \item alternate idea: changing what Xs program can name
    \end{itemize}
\end{frame}

\begin{frame}{aside: capabilities/ambient authority (1)}
    \begin{itemize}
    \item user permissions --- authority tied to each running program
    \item ``access control lists'' for resources
    \item sometimes called ``ambient authority''
    \vspace{.5cm}
    \item alternate model: ``capabilities''
    \item running program has list of things it can access/how
    \end{itemize}
\end{frame}

\begin{frame}{aside: capabilities/ambient authority (2)}
    \begin{itemize}
    \item capabilities = program has list of things it can access
    \item most common thing with design: open files
    \vspace{.5cm}
    \item used as basis of some operating system designs
        \begin{itemize}
        \item not desktop OSes, but\ldots
        \item Unix/Linux has many things with `flavor' of capabilities
        \end{itemize}
    \item in ``fully'' capability-based OSes also\ldots
        \begin{itemize}
        \item capabilities for accessing non-regular-file resources (processes, directories,
            network ports, \ldots)
        \item way of transferring capabilities between programs (instead of, e.g., filenames/PIDs/etc.)
        \item OS doesn't track user IDs/etc. (though maybe system services do)
        \end{itemize}
    \end{itemize}
\end{frame}


\subsection{chroot}
\begin{frame}{Unix filesystems and mounting}
    \begin{itemize}
    \item my Linux desktop has two disks:
        \begin{itemize}
        \item \texttt{/} --- an SSD
        \item \texttt{/mnt/extradisk} --- a hard drive
        \end{itemize}
    \item hard drive appears as \textit{subdirectory} of SSD
    \item subdirectory called a \textit{mount point}
    \end{itemize}
\end{frame}

\begin{frame}[fragile,label=perProcessRoot]{per-process root}
    \begin{itemize}
    \item on Unix: each process tracks its own root directory (/)
    \item can be changed with chroot() system call
        \begin{itemize}
        \item command-line tool to access: \texttt{chroot}
        \end{itemize}
    \vspace{.5cm}
    \item usage: can isolate program from other files on system
        \begin{itemize}
        \item example: limit what public file server can access?
        \end{itemize}
    \end{itemize}
\end{frame}

\begin{frame}[fragile,label=lsChrootExample]{chroot ls}
\begin{lstlisting}[language={},style=smaller]
# mkdir /tmp/example
# cp /bin/ls /tmp/example/ls
# chroot /tmp/example /ls
chroot: failed to run command ‘/ls’: No such file or directory
# cp -r /lib64 /tmp/example/lib64
# mkdir -p /tmp/example/lib
# cp -r /lib/x86_64-linux-gnu /tmp/example/lib/x86_64-linux-gnu
# chroot /tmp/example /ls
/ls: error while loading shared libraries: libpcre2-8.so.0: cannot open shared object file: No such file or directory
# cp /usr/lib/x86_64-linux-gnu/libpcre2-8* /tmp/example/lib/x86_64-linux-gnu
# chroot /tmp/example /ls /
lib  lib64  ls
# chroot /tmp/example /ls /..
lib  lib64  ls
# 
\end{lstlisting}
\end{frame}

\begin{frame}{chroot escapes}
    \begin{itemize}
    \item chroot prevents accessing files outside the new \texttt{/}
    \item but root (system adminstrator) user in chroot can access disks, etc.
    \vspace{.5cm}
    \item typical usage: combine chroot with extra user
    \end{itemize}
\end{frame}

\begin{frame}{chroot impracticality}
    \begin{itemize}
    \item some things make chroot impractical in general:
    \vspace{.5cm}
    \item seems like one needs extra copies of most of the system
    \item hard to communicate between separate roots
    \item requires administrator permissions to configure
        \begin{itemize}
        \item dangerous to let normal users configure b/c they could confuse priviliged (set-user-ID) programs like \texttt{sudo}
        \end{itemize}
    \end{itemize}
\end{frame}


\subsubsection{exercise}
\begin{frame}{exercise}
    \begin{itemize}
    \item what scenarios does chroot make most/least sense for?
    \begin{itemize}
        \item A. the rendering part of web browser
        \item B. a web server
        \item C. a media player
        \item D. a network time server (for other machines to set their clocks)
    \end{itemize}
    \end{itemize}
\end{frame}


\subsection{Linux namespaces}
\begin{frame}{Linux namespaces (1)}
    \begin{itemize}
    \item Linux: alternate sandboxing features
    \item ``namespaces'' for other resources
    \item chroot: each process has own idea of root directory
        \begin{itemize}
        \item change to OS: look up root directory in process, not global variable
        \end{itemize}
    \item can apply this to other resources:
        \begin{itemize}
        \item what filesystems (disks) are available
        \item what network devices are available
        \item what user identifier numbers are
        \item \ldots
        \end{itemize}
    \end{itemize}
\end{frame}

\begin{frame}{Linux namespaces (2)}
    \begin{itemize}
    \item user namespace:
    \vspace{.5cm}
    \item can run programs with new view of users:
    \vspace{.5cm}
    \item inside namespace: running as root
    \item outside namespace: root translated to innocent user ID
    \item allows running programs that expect different users
        \begin{itemize}
        \item \ldots without changes, but without giving special permissions
        \end{itemize}
    \vspace{.5cm}
    \item mechanism: reassign user ID numbers in kernel
        \begin{itemize}
        \item figuring out what user ID means --- always apply current process mapping
        \end{itemize}
    \end{itemize}
\end{frame}

\begin{frame}[fragile,label=LinuxCloneUnshare]{aside: Linux clone(), unshare() syscalls}
\begin{itemize}
\item Linux clone system call: start new process (or thread)
\item flags to specify environment of new process
\item these flags can include ``make a new namespace of a type''
\end{itemize}
\begin{lstlisting}[style=smaller,language=C++]
int id = clone(start_function, ..., CLONE_NEWUSER | other-flags);
\end{lstlisting}
\begin{itemize}
\item above option: new user namespace for new process
\vspace{.5cm}
\item alternative: for changing current process's namespace:
\end{itemize}
\begin{lstlisting}[style=smaller,language=C++]
unshare(CLONE_NEWUSER);
\end{lstlisting}
\end{frame}

\begin{frame}{user namespaces API}
\begin{itemize}
\item Linux: users identified by numerical \textit{user IDs} (UIDs)
\vspace{.5cm}
\item with user namespaces:
\item control file \texttt{/proc/PROCESS-ID/UID\_MAP} contains lines like:
    \begin{itemize}
    \item \texttt{0 1000 2} --- UID 0--1 maps to UID 1000--1001
    \item \texttt{1000 2000 100} --- UID 1000-1100 maps to UID 2000--2100
    \end{itemize}
\item can write to that file to reconfigure (if enough permissions)
\end{itemize}
\end{frame}

\begin{frame}{Linux namesapces (3)}
    \begin{itemize}
    \item mount namespaces:
        \begin{itemize}
        \item Unix: mounting disk = making the contents of the disk available as directories+files
        \end{itemize}
    \vspace{.5cm}
    \item different idea of what filesystems are available
    \item can be setup with \textit{bind mounts} to ``real FS''
        \begin{itemize}
        \item but otherwise: no access to directories outside mount namespace
        \item normally requires root --- but special case with user namespaces
        \end{itemize}
    \end{itemize}
\end{frame}

\begin{frame}[fragile,label=mountNSCmdLine]{mount namespaces API}
from command line:
\begin{lstlisting}[language={},style=smaller]
    # runs shell (/bin/sh) in new mount namesapce
shell1$ unshare --mount /bin/sh

    # setup directories in /tmp/workdir and make them aliases of things on normal FS 
    # these aliases will only exist for processes in mount namespace
shell2$ mkdir -p /tmp/workdir/bin
shell2$ mkdir -p /tmp/workdir/lib
shell2$ mkdir -p /tmp/workdir/usr
shell2$ mkdir -p /tmp/workdir/current
shell2$ mount -o bind,ro /bin /tmp/workdir/bin
shell2$ mount -o bind,ro /lib /tmp/workdir/lib
shell2$ mount -o bind,ro /usr /tmp/workdir/usr
shell2$ mount -o bind /home/someuser /tmp/workdir/current

    # start new shell with the root directory being /tmp/workdir
shell2$ chroot /tmp/workdir /bin/sh
shell3$ cd /
shell3$ /bin/ls
bin     current     lib     usr
\end{lstlisting}
\end{frame}

\begin{frame}{Linux namespaces (3)}
    \begin{itemize}
    \item user namespace and mount namespace together:
    \vspace{.5cm}
    \item run program in new user namespace
    \item map regular root (in namespace) to regular user
        \begin{itemize}
        \item ``opts out'' of programs like sudo
        \end{itemize}
    \item move to new mount namespace
    \item setup bind mounts + chroot
        \begin{itemize}
        \item special case: allowed because root in user namespce
        \item can't get ``real'' root (administrator) privileges ever
        \end{itemize}
    \item run program with subset of available files
    \end{itemize}
\end{frame}

\begin{frame}{Linux namespaces (4)}
    \begin{itemize}
    \item other resources with namespaces
    \item network --- common usage: virtual network device for set processes
        \begin{itemize}
        \item different ``what is my IP address?'' answer for different processes
        \end{itemize}
    \item hostname (``UTS'')
    \item process identifiers
    \item control groups (resource limits for memory, CPU usage, disk I/O, etc.)
    \end{itemize}
\end{frame}

\begin{frame}{Linux control groups}
    \begin{itemize}
    \item control groups --- tied to namespaces
    \item primarily: CPU/memory/IO performance restrictions
        \begin{itemize}
        \item primarily intended for `friendly sharing' (containers, etc.)
        \item important for preventing denial-of-service/etc.
        \item not as big a security conern as file/user/etc. access
        \end{itemize}
    \item also mechanism for adding IO device restrictions
    \item also mechanism to start/stop a bunch of processes together
    \end{itemize}
\end{frame}


% FIXME: Chrome sandboxing failing:
    % https://theori.io/research/escaping-chrome-sandbox/
        % Binder (Chrome internal IPC library)
    % % https://googleprojectzero.blogspot.com/2020/04/you-wont-believe-what-this-one-line.html
    % https://bugs.chromium.org/p/project-zero/issues/detail?id=1991
    % https://bugs.chromium.org/p/project-zero/issues/detail?id=1985

% FIXME: SELinux sandbox escape:
    % https://www.openwall.com/lists/oss-security/2016/09/25/1

\subsection{Linux programs that attempt confinement}
\begin{frame}{Linux sandboxing programs, generally}
    \begin{itemize}
    \item docker, lxc, lxd, containerd
        \begin{itemize}
        \item use namespaces to create ``container'' with own copy of OS libraries, services
        \item but containers share OS `kernel' and potentially files with host unlike VM
        \item (might also have options to use other ways of getting this functionality --- VM's, etc.)
        \end{itemize}
    \item bubblewrap, firejail
        \begin{itemize}
        \item use Linux namespace tools + ``bind mounts'' to give programs only subset of files, etc.
        \item firejail has option of running a ``proxy'' windowing system server
        \end{itemize}
    \item SELinux's sandbox
        \begin{itemize}
        \item uses Security Enhanced Linux's mandatory access controls instead of Linux namespaces
        \item includes option for ``proxy'' for windoing system server
        \end{itemize}
    \end{itemize}
\end{frame}


\subsection{containers}
\begin{frame}{containers}
    \begin{itemize}
    \item Linux's seccomp + namespaces + SELinux commonly used to implement containers
        \begin{itemize}
        \item (plus cgroups (control groups) for performance isolation)
        \end{itemize}
    \vspace{.5cm}
    \item usual goal: looks like virtual machine, but much lower overhead
    \item examples: Docker, Kubernetes
        \begin{itemize}
        \item (note: these may also support other ways of creating `lightweight VMs')
        \end{itemize}
    \end{itemize}
\end{frame}





\subsubsection{runC bug}
\begin{frame}{runc bug}
    \begin{itemize}
    \item 2019 bug in Docker, other container implementations (CVE-2019-5736)
        \begin{itemize}
        \item blog post for vulnerability finders: \\\scriptsize \url{https://blog.dragonsector.pl/2019/02/cve-2019-5736-escape-from-docker-and.html}
        \end{itemize}
    \vspace{.5cm}
    \item bug setup:
        \begin{itemize}
        \item user starts malicious container X
        \item user tells docker to start a new command in malicious container X
        \item \myemph<2>{malicious container X hijacks the ``new command'' starting program}
        \item hijacked program used to access stuff outside container
        \end{itemize}
    \item part of problem: Docker and others weren't using user namespaces at the time
        \begin{itemize}
        \item compatability problems
        \end{itemize}
    \end{itemize}
\end{frame}

\begin{frame}{setup: /proc/PID}
    \begin{itemize}
    \item Linux provides /proc directory to access info about programs
    \item used for implementing process list utils, debugging
        \begin{itemize}
        \item needed to make a functional container
        \end{itemize}
    \item subdirectory for each process in current container
        \begin{itemize}
        \item process ID PID has /proc/PID subdirectory
        \item /proc/self is alias for current process's subdirectory
        \end{itemize}
    \vspace{.5cm}
    \item included is /proc/PID/exe file --- alias for executable file
    \end{itemize}
\end{frame}

\begin{frame}{running a command in existing container}
    \begin{itemize}
    \item to run command X in existing container:
    \vspace{.5cm}
    \item step 1: switch current process to that container
    \item<2-> \myemph{code in container can access /proc here?}
    \item<2-> \myemph{including overwriting /proc/self/exe!}
        \begin{itemize}
        \item which is a program run as root!
        \end{itemize}
    \vspace{.25cm}
    \item step 2: execute command X
    \end{itemize}
\end{frame}


\begin{frame}{partial fix}
    \begin{itemize}
    \item can disable access to /proc/PID/exe (and related things)
    \item system call: \texttt{prctl(PR\_SET\_DUMPABLE, 0)}
    \item but\ldots the run-in-container tool did this for a while
    \vspace{.5cm}
    \item<2-> problem: this gets reset on executing a new program
    \item<2-> and attacker could make the new program be /proc/PID/exe
        \begin{itemize}
        \item one mechanism: symbolic links (file aliases)
        \end{itemize}
    \item<2-> but change dynamic linking setup to run attacker code
    \item<2-> \ldots which accesses /proc/self/exe
    \end{itemize}
\end{frame}

\begin{frame}{full fix}
    \begin{itemize}
    \item make single-use copy of start-in-container tool each time command run
        \begin{itemize}
        \item in-memory file
        \end{itemize}
    \item \ldots so modifying it doesn't change anything
        \begin{itemize}
        \item (but it's also protected from modification)
        \end{itemize}
    \vspace{.5cm}
    \item other solutions:
        \begin{itemize}
        \item make executable non-writable (e.g. SELinux, don't run container as root)
        \end{itemize}
    \end{itemize}
\end{frame}

\end{document}
