
\tikzset{
    stackBox/.style={very thick},
    onStack/.style={thick},
    frameOne/.style={fill=blue!15},
    frameTwo/.style={fill=red!15},
    markLine/.style={blue!50!black},
    markLineB/.style={red!90!black},
    hiLine/.style={red!90!black},
}


\begin{frame}{Linux memory allocation calls}
    \begin{itemize}
    \item \texttt{brk()}
        \begin{itemize}
        \item set `break' at end of heap region
        \item one big memory region for dynamic memory
        \item used to be only way to allocate memory
        \item minimum size of changes = 4KB (x86-64)
        \item want larger changes for speed
        \end{itemize}
    \item \texttt{mmap()}
        \begin{itemize}
        \item allocate new memory region
        \item more complex OS bookkeeping than brk()
            \begin{itemize}
            \item adding to list of memory regions, not changing a size
            \end{itemize}
        \item minimum size = 4KB
        \item want much larger allocations for speed
        \end{itemize}
    \end{itemize}
\end{frame}

\begin{frame}{malloc()/free()/etc. as memory partitioners}
    \begin{itemize}
    \item for ``small'' objects (less than kilobytes)
    \item malloc() allocates \textit{big chunks of memory}
    \item then subdivides them on the fly
    \vspace{.5cm}
    \item different strategies to do this
    \item all need to track \textit{metadata} about allocated objects!
    \end{itemize}
\end{frame}

\begin{frame}{malloc() metadata?}
    \begin{itemize}
    \item need to:
    \vspace{.5cm}
    \item find unused chunks of memory
        \begin{itemize}
        \item even after A=malloc(), B=malloc(), C=malloc(), free(B)
        \end{itemize}
    \item figure out how big allocation is when free() is called
    \vspace{.5cm}
    \item common strategy: keep this information next to objects
        \begin{itemize}
        \item avoids needing lookup table/etc. to find it
        \item avoids needing separate allocation for metadata space
        \end{itemize}
    \end{itemize}
\end{frame}

\begin{frame}[fragile,label=heapLayout]{heap object}
\begin{tikzpicture}
\node[anchor=north east] (code) at (-1,0) {
\begin{lstlisting}
struct AllocInfo {
  bool free;
  int size;
  AllocInfo *prev;
  AllocInfo *next;
};
\end{lstlisting}
};

\tikzset{xscale=0.9}
\begin{scope}[overlay]
    \draw[stackBox,fill=black!20] (0, 1) rectangle (3, -7);

    \draw[onStack] (0, 1) rectangle (3, 0) node[midway,font=\small,align=center] {free space \\ (deleted obj.)};
    \draw[onStack,fill=white] (0, -0.0) rectangle (3, -0.5) node[midway,font=\small] (freeANext) {next};
    \draw[onStack,fill=white] (0, -0.5) rectangle (3, -1.0) node[midway,font=\small] (freeAPrev) {prev};
    \draw[onStack,fill=white] (0, -1.0) rectangle (3, -1.5) node[midway,font=\small] (freeASize) {size/free};

    \draw[very thick, red, rounded corners] (0, 1) rectangle (3, -1.5);

    \draw[onStack,fill=blue!20] (0, -1.5) rectangle (3, -3.0) node[midway,font=\small,align=center] (freeBAlloc) {new'd object};
    \draw[onStack,fill=white] (0, -3.0) rectangle (3, -3.5) node[midway,font=\small] (freeBSize) {size/free};
    
    \draw[very thick, red, rounded corners] (0, -1.5) rectangle (3, -3.5);

    \draw[onStack] (0, -3.5) rectangle (3, -5.0) node[midway,font=\small] {free space};
    \draw[onStack,fill=white] (0, -5.0) rectangle (3, -5.5) node[midway,font=\small] (freeCNext) {next};
    \draw[onStack,fill=white] (0, -5.5) rectangle (3, -6.0) node[midway,font=\small] (freeCPrev) {prev};
    \draw[onStack,fill=white] (0, -6.0) rectangle (3, -6.5) node[midway,font=\small] (freeCSize) {size/free};
    
    \draw[very thick, red, rounded corners] (0, -3.5) rectangle (3, -6.5);
    
    \draw[-Latex,blue,thick] (freeAPrev) -- ++(1.75cm,0cm) |- (freeCSize);
    \draw[-Latex,blue,thick] (freeCNext) -- ++(2.00cm,0cm) |- (freeASize);
    \draw[-Latex,blue,thick,opacity=0.5] (freeCPrev) -- ++(1.25cm,0cm) -- ++(0cm,-2cm);
    \draw[-Latex,blue,thick,opacity=0.5] (freeANext) -- ++(1.75cm,0cm) -- ++(0cm,2cm);
\end{scope}
\draw[-Latex,line width=3pt,black!50] (3.5,-2.25) -- (5.5,-2.25) node[black,midway,above,font=\small\tt] {free};
\begin{scope}[overlay,xshift=6cm,name prefix=sec-]
    \draw[stackBox,fill=black!20] (0, 1) rectangle (3, -7);

    \draw[onStack] (0, 1) rectangle (3, -5.0) node[midway,font=\small] {free space};
    \draw[onStack,fill=white] (0, -5.0) rectangle (3, -5.5) node[midway,font=\small] (freeCNext) {next};
    \draw[onStack,fill=white] (0, -5.5) rectangle (3, -6.0) node[midway,font=\small] (freeCPrev) {prev};
    \draw[onStack,fill=white] (0, -6.0) rectangle (3, -6.5) node[midway,font=\small] (freeCSize) {size/free};
    
    \draw[-Latex,blue,thick,opacity=0.5] (freeCPrev) -- ++(1.25cm,0cm) -- ++(0cm,-2cm);
    \draw[-Latex,blue,thick,opacity=0.5] (freeCNext) -- ++(1.75cm,0cm) -- ++(0cm,2cm);
\end{scope}
\end{tikzpicture}
\end{frame}

\begin{frame}[fragile,label=freeImpl]{implementing free()}
\lstset{
    style=small,
    language=C,
    moredelim={**[is][\btHL<2|handout:0>]{~2~}{~end~}},
}
\begin{lstlisting}
int free(void *object) {
    ...
    block_after = object + object_size;
    if (block_after->free) {
        /* unlink from list, about to merge with previous block */
        new_block->size += block_after->size;
        ~2~block_after->prev->next = block_after->next;~end~
        block_after->next->prev = block_after->prev;
    }
    ...
}
\end{lstlisting}
\begin{itemize}
\item<2> \large \myemph<2>{arbitrary memory write}
\end{itemize}
\end{frame}

