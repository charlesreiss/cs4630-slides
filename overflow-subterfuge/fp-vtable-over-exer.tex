\usetikzlibrary{arrows.meta,matrix,decorations.pathmorphing}
\begin{frame}[fragile,label=vtExer]{exercise}
\vspace{-.25cm}
\begin{tikzpicture}
    \tikzset{
        vt/.style={fill=blue!30},
    }
    \matrix[tight matrix,nodes={text width=3.8cm,text depth=.1ex,font=\small\tt},
            label={[align=center]west:o\\b\\j\\s},anchor=north west] (vulnClass)  at (0, 0) {
        |[vt]| vtable pointer \\
        |[minimum height=.9cm]| buffer\\
        |[vt]| vtable pointer \\
        |[minimum height=1.1cm]| \ldots \\
    };
    \draw[white,line width=1mm] ([yshift=3.5mm,xshift=-1mm]vulnClass.south west) -- ++(2mm,0mm);
    \draw[white,line width=1mm] ([yshift=3.5mm,xshift=-1mm]vulnClass.south east) -- ++(2mm,0mm);
    \draw[thin] ([yshift=3mm,xshift=-1mm]vulnClass.south west) -- ++(2mm,0mm);
    \draw[thin] ([yshift=4mm,xshift=-1mm]vulnClass.south west) -- ++(2mm,0mm);
    \draw[thin] ([yshift=3mm,xshift=-1mm]vulnClass.south east) -- ++(2mm,0mm);
    \draw[thin] ([yshift=4mm,xshift=-1mm]vulnClass.south east) -- ++(2mm,0mm);
    \draw[thin,decorate,decoration={straight zigzag}] ([yshift=3mm,xshift=1mm]vulnClass.south west) --
                 ([yshift=3mm,xshift=-1mm]vulnClass.south east);
    \draw[thin,decorate,decoration={straight zigzag}] ([yshift=4mm,xshift=1mm]vulnClass.south west) --
                 ([yshift=4mm,xshift=-1mm]vulnClass.south east);
    \matrix[tight matrix,nodes={text width=3.8cm,text depth=.1ex,font=\small\tt},anchor=north west] (isVT) at (0, -3) {
        \tt slot for foo\\
        \tt slot for bar\\
    };
    \draw[thick,-Latex] (vulnClass-1-1.east) -- ++(.45cm,0cm) |- (isVT-1-1.east);
    \draw[thick,-Latex] (vulnClass-3-1.east) -- ++(.35cm,0cm) |- (isVT-1-1.east);
    \node[anchor=north west] at (5, 0) {
\begin{lstlisting}[language=C++,style=smaller]
class VulnerableClass {
public:
    char buffer[100];
    virtual void foo();
    virtual void bar();
};
VulnerableClass objs[10];
\end{lstlisting}
};
\end{tikzpicture}
\begin{itemize}
\item Assume \verb|gets(objs[0].buffer)| is run and eventually \verb|ptr->foo()| will be run where \verb|ptr == &objs[1]|.
\end{itemize}
\small \begin{tabular}{ll}
input start: \rule{2cm}{.1mm} \hspace{1cm} & A. shellcode \hspace{1cm} B. address of objs[0].buffer[0] \\
input+50 bytes: \rule{2cm}{.1mm} \hspace{1cm} & C. address of objs[0].buffer[50]\\
input+100 bytes: \rule{2cm}{.1mm} \hspace{1cm} & D. address of original vtable \\
~ & E. address of objs[0]'s vtable \\
~ & F. address of objs[1]'s vtable pointer \\
\end{tabular}
\end{frame}
