\usetikzlibrary{arrows.meta,matrix}
\begin{frame}[fragile,label=vtExer]{exercise}
\vspace{-.25cm}
\begin{tikzpicture}
    \tikzset{
        vt/.style={fill=blue!30},
    }
    \matrix[tight matrix,nodes={text width=3.8cm,text depth=.1ex,font=\small\tt},
            label={north:objArray},anchor=north west] (vulnClass)  at (0, 0) {
        |[vt]| vtable pointer \\
        |[minimum height=1cm]| buffer\\
        |[vt]| vtable pointer \\
        |[minimum height=1cm]| \ldots \\
    };
    \matrix[tight matrix,nodes={text width=3.8cm,text depth=.1ex,font=\small\tt},anchor=north west] (isVT) at (0, -3) {
        \tt slot for foo\\
        \tt slot for bar\\
    };
    \draw[thick,-Latex] (vulnClass-1-1.east) -- ++(.45cm,0cm) |- (isVT-1-1.east);
    \draw[thick,-Latex] (vulnClass-3-1.east) -- ++(.35cm,0cm) |- (isVT-1-1.east);
    \node[anchor=north west] at (5, 0) {
\begin{lstlisting}[language=C++,style=smaller]
class VulnerableClass {
public:
    char buffer[100];
    virtual void foo();
    virtual void bar();
};
VulnerableClass objArray[10];
\end{lstlisting}
};
\end{tikzpicture}
\begin{itemize}
\item if we can overflow \texttt{objArray[0].buffer} to change array[1]'s vtable pointer and know array[1].foo() will be called; finish the plan:
\end{itemize}
\small \begin{tabular}{ll}
buffer[0]: \rule{2cm}{.1mm} \hspace{1cm} & A. shellcode \\
buffer[50]: \rule{2cm}{.1mm} \hspace{1cm} & B. address of buffer[0]\\
array[1]'s vtable pointer: \rule{2cm}{.1mm} \hspace{1cm} & C. address of buffer[50] \\
~ & D. address of original vtable \\
~ & E. address of objArray[0]'s vtable \\
~ & E. address of objArray[1]'s vtable pointer \\
\end{tabular}
\end{frame}
