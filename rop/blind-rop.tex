\begin{frame}{utility gadgets}
    \begin{itemize}
    \item once we find return address through leak\ldots
    \item look for nearby address with particular behavior:
    \vspace{.5cm}
    \item `stop' gadget --- hang program
    \item `crash' gadget --- close connection prematurely
    \end{itemize}
\end{frame}

\begin{frame}{looking for pops}
    \begin{itemize}
    \item common form for gadget is \texttt{pop XXX; ret}
    \item how can we tell if we might have that?
    \item write to stack:   
        \begin{itemize}
        \item gadget being tested address, followed by
        \item stop gadget address, followed by
        \item crash gadget address
        \end{itemize}
    \item \texttt{pop XXX; ret} gadget will crash
        \begin{itemize}
        \item XXX becomes stop address; then ret to crash
        \end{itemize}
    \item \texttt{...; ret} gadget will hang
        \begin{itemize}
        \item ret to stop
        \end{itemize}
    \end{itemize}
\end{frame}

\begin{frame}{blind ROP outline}
    \begin{itemize}
    \item look for gadget that pops a lot from the stack
        \begin{itemize}
        \item likely allows setting lots of registers
        \end{itemize}
    \item look for strcmp() function
        \begin{itemize}
        \item should crash/not crash based on whether two registers are valid pointers
        \item use to set RDX (consequence of Linux libc implementation)
        \end{itemize}
    \item look for write() function 
    \item use write() function to output program machine code to network
    \end{itemize}
\end{frame}
