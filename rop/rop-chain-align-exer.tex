


\begin{frame}[fragile]{exercise: ROP chain alignment}
\begin{tikzpicture}
\node (code) {
\begin{lstlisting}[language=C++,style=smaller]
void getInitials(char *init) {
    char first[50]; char second[50];
    scanf("%s%s", first, second);
    init[0] = first[0];
    init[1] = second[0];
}
\end{lstlisting}
};
\node[anchor=north west] (asm) at ([xshift=.2cm]code.north east) {
\begin{lstlisting}[language=myasm,style=smaller]
getInitials: push %rbx
xor    %eax,%eax
mov    %rdi,%rbx
// lea "%s%s" -> %rdi
lea    0xe6e(%rip),%rdi 
sub    $0xa0,%rsp
// &second[0] -> %rdx
lea    0x50(%rsp),%rdx 
// &first[0] -> %rsi
mov    %rsp,%rsi
call   __isoc99_scanf@plt
mov    (%rsp),%al
mov    %al,(%rbx)
mov    0x50(%rsp),%al
mov    %al,0x1(%rbx)
add    $0xa0,%rsp
pop    %rbx
ret 
\end{lstlisting}
};
\node[anchor=north west,align=left,font=\small] at ([yshift=.1cm]code.south west) {
Suppose we have 64-byte ROP chain \\
w/o whitespace in it. How to write input? \\
(Multiple might work) \\
A. \textit{[80 As]}\textit{[ROP chain]} X \\
B. \textit{[160 As]}\textit{[ROP chain]} X \\
C. \textit{[168 As]}\textit{[ROP chain]} X \\
D. \textit{[ROP chain]}\textit{[36 As]}\textit{[ROP chain addr]} X \\
E. \textit{[ROP chain]} \textit{[80 As]}\textit{[ROP chain addr]} \\
F. X \textit{[88 As]}\textit{[ROP chain]} \\
G. X \textit{[96 As]}\textit{[ROP chain]} \\
};
\end{tikzpicture}
\end{frame}

\begin{frame}{solution preview}
    \begin{itemize}
    \item want \textit{stack pointer}, not \textit{program counter} to point to ROP chain
    \item program counter will point to gadgets
    \vspace{.5cm}
    \item will align ROP chain so it's top of stack as function returns
    \item program counter (where returned to) will be in gadgets
    \end{itemize}
\end{frame}

\begin{frame}{solution}
    \begin{itemize}
    \item {\scriptsize C. \textit{[168 As] X }\textit{[ROP chain]} X or E. X \textit{[88 As]}\textit{[ROP chain]}}
    \item stack layout: 
        \begin{itemize}
        \item ~[first (80 bytes)][second (80 bytes)][saved RBX][return address]
        \end{itemize}
    \item \texttt{first} at return address - 168 bytes
    \item \texttt{second} at return address - 88 bytes
    \item ROP chain's first 8 bytes = address of first gadget to run
        \begin{itemize}
        \item e.g. address of \texttt{pop \%rdi, ret}
        \end{itemize}
    \item ROP chain's next bytes = things popped by first gadget
        \begin{itemize}
        \item e.g. value for \%rdi, followed by next gadget address
        \end{itemize}
    \end{itemize}
\end{frame}
