\begin{frame}[fragile,label=useFPtrOverwrite1]{using function pointer overwrite (1)}
\begin{lstlisting}[language=C,style=script]
struct Example {
    char input[1000];
    void (*process_function)(Example *, long, char *);
};
void vulnerable(struct Example *e) {
    long index; char name[1000];
    gets(e->input); /* can overwrite process_function */
    sscanf(e->input, "%ld,%s", &index, &name[0]); /* expects <decimal number>,<string> */
    (e->process_function)(e /* rdi */, index /* rsi */, name /* rdx */);
}
\end{lstlisting}
\begin{itemize}
\item \small if we overwrite process\_function's address with the address of the gadget
    \texttt{mov \%rsi, \%rsp; ret}, then input (scanf) start with \ldots?
    \begin{itemize}
    \item A. the shellcode to run (assuming exec+writeable memory)
    \item B. an ROP chain to run
    \item C. the address of shellcode (or existing function) in decimal
    \item D. the address of the ROP chain to run written out in decimal
    \item E. the address of a RET instruction written out in decimal
    \end{itemize}
\end{itemize}
\end{frame}

\iftoggle{heldback}{\excludecomment{soln}}{\includecomment{soln}}
\begin{soln}
\begin{frame}[fragile,label=useFPtrOverwrite1Explain]{explanation}
\begin{lstlisting}[language=C++,style=script]
gets(e->input); /* can overwrite process_function */
sscanf(e->input, "%ld,%s", &index, &name[0]); /* expects <decimal number>,<string> */
(e->process_function)(e /* rdi */, index /* rsi */, name /* rdx */);
\end{lstlisting}
\texttt{"1234,FOO......."} + addr of \texttt{mov \%rsi, \%rsp, ret}
\begin{itemize}
\item arguments setup registers for gadget:
\begin{itemize}
    \item \%rdi (irrelevant) is "1234,FOO..." (copy in e)
    \item \%rsi is 1234 (from scanf)
    \item \%rdx (irrelevant) is "FOO..." (pointer to name)
\end{itemize}
\item mov in gadget: \%rsi (1234) becomes \%rsp
\item ret in gadget: read pointer at 1234, set \%rsp to 1234 + 8
    \begin{itemize}
    \item jump to next gadget (whose address should be stored at 1234)
    \item if that gadget returns, will read new return address from 1238
    \end{itemize}
\end{itemize}
\end{frame}
\end{soln}

\begin{frame}[fragile,label=useFPtrOverwrite2]{using function pointer overwrite (2)}
\begin{lstlisting}[language=C,style=script]
struct Example {
    char input[1000];
    void (*process_function)(Example *, long, char *);
};
void vulnerable(struct Example *e) {
    long index; char name[1000];
    gets(e->input); /* can overwrite process_function */
    scanf("%ld,%s", &index, &name[0]); /* expects <decimal number>,<string> */
    (e->process_function)(e /* rdi */, index /* rsi */, name /* rdx */);
}
\end{lstlisting}
\begin{itemize}
\item \small if we overwrite process\_function's address with the address of the gadget
    \texttt{push \%rdx; jmp *(\%rdi)}, then the beginning of the input should contain\ldots \\
    \begin{itemize}
    \item A. the shellcode to run (assuming exec+writeable memory)
    \item B. an ROP chain to run
    \item C. the address of shellcode (or existing function) 
    \item D. the address of the ROP chain 
    \item E. the address of a RET instruction
    \end{itemize}
\end{itemize}
\end{frame}

\begin{soln}
\begin{frame}[fragile,label=useFPtrOverwrite1Explain]{explanation (one option)}
\begin{lstlisting}[language=C++,style=script]
gets(e->input); /* can overwrite process_function */
sscanf(e->input, "%ld,%s", &index, &name[0]); /* expects <decimal number>,<string> */
(e->process_function)(e /* rdi */, index /* rsi */, name /* rdx */);
\end{lstlisting}
\texttt{"FOOBARBAZ......."} + addr of \texttt{push \%rdx; jmp *(\%rdi)}
\begin{itemize}
\item arguments setup registers for gadget:
\begin{itemize}
    \item \%rdi is "FOOBARBAZ...." (copy in e)
    \item \%rsi (irrelevant) is uninitialized? (scanf failed)
    \item \%rdx (irrelevant) is uninitialized? (scanf failed)
\end{itemize}
\item push in gadget: top of stack becomes copy of uninit. value 
\item jmp in gadget
    \begin{itemize}
    \item interpret ``FOOBARBA'' as 8-byte address
    \item jump to that address
    \end{itemize}
\end{itemize}
\end{frame}

\begin{frame}[fragile,label=useFPtrOverwrite2Explain]{explanation (unlikely alternative?)}
\begin{lstlisting}[language=C++,style=script]
gets(e->input); /* can overwrite process_function */
sscanf(e->input, "%ld,%s", &index, &name[0]); /* expects <decimal number>,<string> */
(e->process_function)(e /* rdi */, index /* rsi */, name /* rdx */);
\end{lstlisting}
\texttt{"1234567890,FOO......."} + addr of \texttt{push \%rdx; jmp *(\%rdi)}
\begin{itemize}
\item arguments setup registers for gadget:
\begin{itemize}
    \item \%rdi is address of string "12345678,FOO..." (copy in e)
    \item \%rsi is 12345678
    \item \%rdx is address of string "FOO..." (copy in name)
\end{itemize}
\item push in gadget: top of stack becomes address of "FOO..."
\item jmp in gadget
    \begin{itemize}
    \item interpret \textit{ASCII encoding of ``12345678''} (???) as 8-byte address
    \item jump to that address
    \end{itemize}
\end{itemize}
\end{frame}
\end{soln}
