\date{}
\title{}
\date{}
\begin{document}
\begin{frame}
    \titlepage
\end{frame}


\makeatletter
\newenvironment<>{btHighlight}[1][]
{\begin{onlyenv}#2\begingroup\tikzset{bt@Highlight@par/.style={#1}}\begin{lrbox}{\@tempboxa}}
{\end{lrbox}\bt@HL@box[bt@Highlight@par]{\@tempboxa}\endgroup\end{onlyenv}}

\newcommand<>\btHL[1][]{%
  \only#2{\begin{btHighlight}[#1]\bgroup\aftergroup\bt@HL@endenv}%
}
\def\bt@HL@endenv{%
  \end{btHighlight}%   
  \egroup %
}
\tikzset{
    btHLbox/.style={
        fill=red!30,outer sep=0pt,inner xsep=1pt, inner ysep=0pt, rounded corners=3pt
    },
}
\newcommand{\bt@HL@box}[2][]{%
  \tikz[#1]{%
    \pgfpathrectangle{\pgfpoint{1pt}{0pt}}{\pgfpoint{\wd #2}{\ht #2}}%
    \pgfusepath{use as bounding box}%
    \node[text width={},draw=none,anchor=base west, btHLbox, minimum height=\ht\strutbox+1pt,#1]{\raisebox{1pt}{\strut}\strut\usebox{#2}};
  }%
}

\lst@CCPutMacro
    \lst@ProcessOther {"2A}{%
      \lst@ttfamily 
         {\raisebox{2pt}{*}}% used with ttfamily
         {\raisebox{2pt}{*}}}% used with other fonts
    \@empty\z@\@empty

\lstdefinelanguage
   [x8664gas]{Assembler}     % add a "x64" dialect of Assembler
   [x86masm]{Assembler} % based on the "x86masm" dialect
   % with these extra keywords:
   {morekeywords={CDQE,CQO,CMPSQ,CMPXCHG16B,JRCXZ,LODSQ,MOVSXD,%
                  POPFQ,PUSHFQ,SCASQ,STOSQ,IRETQ,RDTSCP,SWAPGS,.TEXT,.STRING,.ASCIZ,%
                  BEQ,LW,SW,LB,SB,ADDIU,J,BEQZ,BNEZ,BNE,%
                  MOVUPD,MULPD,MOVSD,MULSD,%
                  SHLADD,MOV,CMP.LT,TBIT.NZ,BR.RET.SPTK.MANY,%
                  ADDQ,POPQ,PUSHQ,RRMOVQ,MRMOVQ,RMMOVQ,IRMOVQ,%
                  <-,LL,SC,ADDI,ADDL,VMOVDQA,ADDQ,CMPL,JB,JBE,MOVL,CLTQ,%
                  MOVW,PUSHW,MOV,ADD,SUB,INT,PUSH,MOV,ADD,REP,MOVSB,%
                  TESTQ,CMPQ,MOVL,MOVQ,ADDQ,JMPQ,XORQ,%
                  LEAQ,LEAL,LEA,RETQ,RET,POPL,POPW,PUSHL,PUSHW,%
                  LEAW,%
                  SUBQ,SYSCALL,.ASCII,CALLQ,MOVSLQ,JMP,ANDQ,SHRQ,MOVB,INCQ,TESTL,XORL,%
                  SHRL,LEAL,SARL,SUBL,IMULL,IMULQ,MOVDQU,PADDD,XORL,%
                  MOVZBL,MOVZB,SHRB,SRAL,SHRL,ANDL,%
                  CMOVNS,SRAL,SRAQ,MOVZBW,MOVZBQ,%
                  PADDW,PADDQ,MODUPS,MOVAPD,%
                  MOVL,RET,.GLOBL,%
		  PAUSE,LFENCE,JMP,%
                  },
    deletekeywords={eax,ebx,sp,si,cx,di,ds,cs,es,fs,dx,ax,bx,al,esi,ebp,ecx,rip,eip,edx,edi,rdi,esp},
    deletekeywords=[2]{size},
    alsoletter={\%},
    alsoother={()},
    emphstyle={\color{violet!50!black}},
    emph={\%rax,\%rbx,\%rcx,\%rdx,\%r8,\%r9,\%r10,\%r11,\%r12,\%r13,\%r14,\%r15,\%eax,\%ebx,\%sp,\%si,\%cx,\%di,\%ds,\%cs,\%es,\%fs,\%dx,\%ax,\%bx,\%al,\%esi,\%ebp,\%ecx,\%rip,\%eip,\%edx,\%edi,\%rdi,\%esp,\%rsp},
    %moreemph={eax,ebx,sp,si,cx,di,ds,cs,es,fs,dx,ax,bx,al,esi,ebp,ecx,rip,eip,edx,edi,rdi,esp},
    morecomment=[l]{\#},
    morecomment=[l]{\/\/},
    morecomment=[s]{/*}{*/},
    sensitive=false,
    keepspaces=true} % et

\lstalias[]{myasm}[x8664gas]{Assembler}

\lstdefinelanguage{JavaScript}{
  keywords={typeof, new, true, false, catch, function, return, null, catch, switch, var, if, in, while, do, else, case, break},
  ndkeywords={class, export, boolean, throw, implements, import, this},
  sensitive=false,
  comment=[l]{//},
  morecomment=[s]{/*}{*/},
  morestring=[b]',
  morestring=[b]"
}

\newcommand{\keywordstyle}{\sourcecodeprolight\bfseries\color{blue!30!black}}
\newcommand{\stringstyle}{\color{blue!20!black}\ttfamily}

\lstset{
    language=C,
    basicstyle=\sourcecodepro\EmptyMapping,
    escapechar=`,
    keywordstyle=\keywordstyle\EmptyMapping,
    identifierstyle=\sourcecodepro\EmptyMapping,
    numberstyle=\small\color{black!70},
    commentstyle=\color{red!60!black}\ttfamily\itshape,
    stringstyle=\color{blue!20!black}\ttfamily,
    ndkeywordstyle=\bfseries\color{blue!30!black},
    upquote=true,
}



\lstdefinestyle{medium}{
    basicstyle=\sourcecodepro\EmptyMapping\fontsize{12}{13}\selectfont,
    keywordstyle=\sourcecodepro\EmptyMapping\fontsize{12}{13}\selectfont\keywordstyle,
}

\lstdefinestyle{small}{
    basicstyle=\sourcecodepro\EmptyMapping\small,
    keywordstyle=\sourcecodepro\EmptyMapping\small\keywordstyle,
}

\lstdefinestyle{smaller}{
    basicstyle=\sourcecodepro\EmptyMapping\fontsize{11}{12}\selectfont,
    keywordstyle=\sourcecodepro\EmptyMapping\fontsize{11}{12}\selectfont\keywordstyle,
}

\lstdefinestyle{size105}{
    basicstyle=\sourcecodepro\EmptyMapping\fontsize{10.5}{11.5}\selectfont,
    keywordstyle=\sourcecodepro\EmptyMapping\fontsize{10.5}{11.5}\selectfont\keywordstyle,
}

\lstdefinestyle{size10}{
    basicstyle=\sourcecodepro\EmptyMapping\fontsize{10}{11}\selectfont,
    keywordstyle=\sourcecodepro\EmptyMapping\fontsize{10}{11}\selectfont\keywordstyle,
}

\lstdefinestyle{size9}{
    basicstyle=\sourcecodepro\EmptyMapping\fontsize{9}{10}\selectfont,
    keywordstyle=\sourcecodepro\EmptyMapping\fontsize{9}{10}\selectfont\keywordstyle,
}
\lstdefinestyle{size8}{
    basicstyle=\sourcecodepro\EmptyMapping\fontsize{8}{9}\selectfont,
    keywordstyle=\sourcecodepro\EmptyMapping\fontsize{8}{9}\selectfont\keywordstyle,
}



\lstdefinestyle{script}{
    basicstyle=\sourcecodepro\EmptyMapping\scriptsize,
    keywordstyle=\sourcecodepro\EmptyMapping\scriptsize\bfseries,
}




\usetikzlibrary{arrows.meta,patterns}

\begin{frame}{last time}
    \begin{itemize}
    \item finding gadgets in programs
        \begin{itemize}
        \item look for RET, JMP, etc
        \item go back some bytes, try disassembling
        \end{itemize}
    \item heuristics for generating automatic chains
        \begin{itemize}
        \item look for specific gadgets --- likely in big program
        \item canned technique for putting together
        \end{itemize}
    \item starting chains with non-returns
        \begin{itemize}
        \item one idea: set \%rsp, keep going
        \item another: chain jmp+calls
        \end{itemize}
    \item jump-oriented programming
        \begin{itemize}
        \item dispatcher gadgets
        \end{itemize}
    \item blind ROP
    \end{itemize}
\end{frame}

\begin{frame}{review: blind ROP}
    \begin{itemize}
    \item key idea: `stack reading'
        \begin{itemize}
        \item guess what's on stack in overflow
        \item if match: no crash; otherwise crash
        \item keep retrying because server rebooted
        \end{itemize}
    \item guess + observe to find useful gadgets
        \begin{itemize}
        \item know we can find: `crash' and `hang' gadgets
        \item whether `hang' gadget runs indicates what earlier parts of chain pops
        \item know we can find: function epilogue that pops registers
        \item \ldots in consistent order (because compiler is consistent)
        \item know we can find: stubs for library funcions
        \end{itemize}
    \end{itemize}
\end{frame}

\begin{frame}{post-mortem on some assignments}
    \begin{itemize}
    \item OVER
        \begin{itemize}
        \item surprised by submissions not overflowing buffer (too little output)
        \item some students trying to use ROP chain (not needed/surprising)
        \item apparently WSL (Windows Subsystem for Linux) can ignore executable stack setting? (we test on system like portal)
        \end{itemize}
    \item 
    \end{itemize}
\end{frame}

\begin{frame}{scheduling notes (1)}
    \begin{itemize}
    \item upcoming assignments:
    \item ROP --- this week
    \item use-after-free (UAF) --- next week
        \begin{itemize}
        \item this week lecture topic: on the heap
        \end{itemize}
    \item (tentative) RUST --- after that
        \begin{itemize}
        \item adapt some use-after-free'ing code in `memory safe' programming language
        \item new assignment, need to update lecture material
        \end{itemize}
    \item (tentative) FUZZ --- using whitebox fuzz-testing tool
    \item (planned) SANDBOX 
        \begin{itemize}
        \item use seccomp (Linux system call sandboxing tool) to `secure' library
        \end{itemize}
    \end{itemize}
\end{frame}

\begin{frame}{scheduling notes (2)}
    \begin{itemize}
    \item immediate topic: the heap
        \begin{itemize}
        \item so far: relying on predictable stack+global vars
        \item ``randomness'' of the malloc/etc. seems hard to deal with?
        \item topic 1: what can buffer overflows on the heap do
        \item topic 2: taking advantage of use-after-free bugs
        \end{itemize}
    \item next topics (unsure re: order)
        \begin{itemize}
        \item ``memory-safe'' non-garbage-collected programming languages (example: Rust)
        \item finding memory safety bugs: grey/whitebox fuzz testing, symbolic execution
        \item `fast' bounds-checking support
        \item sandboxing + privilege separation (running dangerous code where it can't do harm)
        \item (if time) web security (same-origin; CSRF; XSS)
        \end{itemize}
    \item topics I'm tentatively skipping:
        \begin{itemize}
        \item format string exploits
        \end{itemize}
    \end{itemize}
\end{frame}



\section{overflows on the heap, first look}
\subsection{simple case}
\usetikzlibrary{arrows.meta,patterns}

\tikzset{
    stackBox/.style={very thick},
    onStack/.style={thick},
    frameOne/.style={fill=blue!15},
    frameTwo/.style={fill=red!15},
    markLine/.style={blue!50!black},
    markLineB/.style={red!90!black},
    hiLine/.style={red!90!black},
}
\begin{frame}[fragile,label=heapOverflowInObj]{easy heap overflows}
\begin{tikzpicture}
\node[anchor=north east] (code) at (-1,0) {
\begin{lstlisting}
struct foo {
    char buffer[100];
    void (*func_ptr)(void);
};
\end{lstlisting}
};
\draw[stackBox] (0, 0) rectangle (4, -6);
\draw[thick,-Latex] (-.25,-5) -- (-.25, -1) node [midway, above, sloped] {increasing addresses};
\draw[onStack] (0, -0.5) rectangle (4, -6) node[midway,font=\small] {\texttt{buffer}};
\draw[onStack] (0, -0) rectangle (4, -0.5) node[midway,font=\small] {\texttt{func\_ptr}};
\end{tikzpicture}
\end{frame}



\subsection{adjacent on the heap}
\usetikzlibrary{arrows.meta,patterns}

\tikzset{
    stackBox/.style={very thick},
    onStack/.style={thick},
    frameOne/.style={fill=blue!15},
    frameTwo/.style={fill=red!15},
    markLine/.style={blue!50!black},
    markLineB/.style={red!90!black},
    hiLine/.style={red!90!black},
}

\begin{frame}[fragile,label=heapOverflowAdj]{heap overflow: adjacent allocations}
\begin{tikzpicture}
\lstset{language=C++,style=small}
\node[anchor=north east] (code) at (-1,0) {
\begin{lstlisting}
class V {
  char buffer[100];
public:
  virtual void ...;
  ...
};
...
V *first = new V(...);
V *second = new V(...);
strcpy(first->buffer,
       attacker_controlled);
\end{lstlisting}
};
\node[anchor=south] at (2, 0) {the heap};
\draw[thick,-Latex] (-.25,-5) -- (-.25, -1) node [midway, above, sloped] {increasing addresses};
\draw[stackBox,fill=black!20] (0, 0) rectangle (4, -6);
\draw[onStack,fill=white] (0, -0.5) rectangle (4, -2.0) node[midway,font=\small] {\texttt{second}'s \texttt{buffer}};
\draw[onStack,fill=white] (0, -2.0) rectangle (4, -2.5) node[midway,font=\small] {\texttt{second}'s \textbf{vtable}};

\draw[onStack,fill=white] (0, -3.0) rectangle (4, -4.5) node[midway,font=\small] {\texttt{first}'s \texttt{buffer}};
\draw[onStack,fill=white] (0, -4.5) rectangle (4, -5.0) node[midway,font=\small] {\texttt{first}'s \textbf{vtable}};

\begin{visibleenv}<2>
    \fill[pattern=north west lines,pattern color=red!80] (0, -2.0) rectangle (4, -4.5);
    \node[anchor=west,font=\small,align=left] at (4.1, -3.25) {result of\\overflowing\\buffer};
\end{visibleenv}
\end{tikzpicture}
\end{frame}




\subsection{preview: heap structure}
\begin{frame}{heap structure}
    \begin{itemize}
    \item where does malloc, free, new, delete, etc. keep info?
    \item often in data structures next to objects on the heap
    \vspace{.5cm}
    \item special case of adjacent heap objects problem
    \item topic for later
    \end{itemize}
\end{frame}


% FIXME: go into example from sudo
\section{sudo exploit example}
\begin{frame}{sudo exploit}
\begin{itemize}
\item this writeup: summary from \url{https://www.openwall.com/lists/oss-security/2021/01/26/3}
\item from group at Qualys
\end{itemize}
\end{frame}

\begin{frame}[fragile,label=sudoBug]{sudo bug}
\begin{itemize}
\item the bug:
\end{itemize}
\begin{lstlisting}[language=C++,style=smaller]
for (size = 0, av = NewArgv + 1; *av; av++)
     size += strlen(*av) + 1;
if (size == 0 || (user_args = malloc(size)) == NULL) { ... }
...
for (to = user_args, av = NewArgv + 1; (from = *av); av++) {
while (*from) {
  if (from[0] == '\\' && !isspace((unsigned char)from[1]))
    from++;
  *to++ = *from++;
...
\end{lstlisting}
\begin{itemize}
\item can skip \texttt{\textbackslash 0} if prefixed with backslash
\item but \texttt{strlen} used to allocate buffer
\item disagreement about copied string length
\item heap overflow!
\end{itemize}
\end{frame}

\begin{frame}{brute-forcing?}
\begin{itemize}
\item method: tried to lots of buffer overflows, get crashes
\item looked at them by hand, found interesting ones\ldots
\end{itemize}
\end{frame}

\begin{frame}[fragile,label=oneCrash]{one crash}
% FIXME: picture showing adjacent struct
\begin{lstlisting}[language={},style=script]
0x000056291a25d502 in process_hooks_getenv (name=name@...ry=0x7f4a6d7dc046 "SYSTEMD_BYPASS_USERDB", value=value@...ry=0x7ffc595cc240) at ../../src/hooks.c:108

=> 0x56291a25d502 <process_hooks_getenv+82>:    callq  *0x8(%rbx)

108         rc = hook->u.getenv_fn(name, &val, hook->closure);
\end{lstlisting}
\begin{itemize}
\item they overwrote a function pointer on the heap!
\item next inquiry: where did that usually point?
\end{itemize}
\end{frame}

\begin{frame}[fragile,label=sudoersSoCode]{sudoers.so}
\begin{lstlisting}[language={},style=smaller]
    *** interesting standard library function: ***
0000000000008a00 <execv@plt>:
    8a00:      endbr64 
    8a04:      bnd jmpq *0x55565(%rip)        # 5df70 <execv@GLIBC_2.2.5>
    8a0b:      nopl   0x0(%rax,%rax,1)
...
    *** usual value of function pointer: ***
000000000000ea00 <sudoers_hook_getenv>:
    ea00:      endbr64 
    ea04:      xor    %eax,%eax
    ea06:      cmpb   $0x0,0x51d36(%rip)        # 60743 <sudoers_policy@@Base+0x2003>
    ea0d:      jne    eaf8 <freeaddrinfo@plt+0x60a8>
    ea13:      cmpq   $0x0,0x51d45(%rip)        # 60760 <sudoers_policy@@Base+0x2020>
\end{lstlisting}
\begin{itemize}
\item<2-> observations (that hold true even with ASLR):
    \begin{itemize}
    \item addr(\texttt{execv@plt}) - addr(\texttt{sudoers\_hook\_getenv}) = \texttt{-0x6000}
    \item last 12 bits of execv@plt always \texttt{a00} (page alignment)
    \end{itemize}
\end{itemize}
\end{frame}

\begin{frame}{changing pointer (part one)}
\begin{itemize}
\item suppose hook\_getenv pointer is \texttt{0xabcdef8a00}
    \begin{itemize}
    \item as bytes: \texttt{\myemph{00 8a} ef cd ab 00 00 00}
    \end{itemize}
\item then execv@plt pointer is \texttt{0xabcdef3a00}
    \begin{itemize}
    \item as bytes: \texttt{\myemph{00 3a} ef cd ab 00 00 00}
    \end{itemize}
\vspace{.5cm}
\item only need to change the last two bytes
\item also: same change would work if pointer had different high bits
\item<2-> only four bits of random data from ASLR!
\end{itemize}
\end{frame}

\begin{frame}{changing pointer (part two)}
\begin{itemize}
\item solution: guess hook\_getenv pointer at  \texttt{0x} (unknown) \texttt{8a00}
\item overwrite last two bytes with \texttt{00 3a}
\vspace{.5cm}
\item if right: will execute your program
\item if wrong: will crash
\vspace{.5cm}
\item<2-> what if crashes? try again! 
    \begin{itemize}
    \item would work about once every 16 tries\ldots
    \item but actual exploit needed to write a 00 byte at the end (strcpy)
    \item so worked `only' about once every 4096 tries
    \end{itemize}
\end{itemize}
\end{frame}

\begin{frame}{into exploit}
\begin{itemize}
\item make \texttt{SYSTEMD\_BYPASS\_USERDB} program in current directory
\item run sudo, triggering buffer overflow to change \\ \texttt{\small sudoers\_hook\_getenv("SYSTEMD\_BYPASS\_USERDB", ...)} \\  into \\
      \texttt{\small execv(SYSTEMD\_BYPASS\_USERDB, ...)} 
    \begin{itemize}
    \item (well, try to change --- it won't always work)
    \end{itemize}
\end{itemize}
\end{frame}



\section{heap smashing}

\begin{frame}{heap smashing}
    \begin{itemize}
    \item ``lucky'' adjancent objects
    \item same things possible on stack
    \item but stack overflows had nice generic ``stack smashing''
    \item is there an equivalent for the heap?
    \item yes (mostly)
    \end{itemize}
\end{frame}




\subsection{heap bookkeeping}

\tikzset{
    stackBox/.style={very thick},
    onStack/.style={thick},
    frameOne/.style={fill=blue!15},
    frameTwo/.style={fill=red!15},
    markLine/.style={blue!50!black},
    markLineB/.style={red!90!black},
    hiLine/.style={red!90!black},
}


\begin{frame}{Linux memory allocation calls}
    \begin{itemize}
    \item \texttt{brk()}
        \begin{itemize}
        \item set `break' at end of heap region
        \item one big memory region for dynamic memory
        \item used to be only way to allocate memory
        \item minimum size of changes = 4KB (x86-64)
        \item want larger changes for speed
        \end{itemize}
    \item \texttt{mmap()}
        \begin{itemize}
        \item allocate new memory region
        \item more complex OS bookkeeping than brk()
            \begin{itemize}
            \item adding to list of memory regions, not changing a size
            \end{itemize}
        \item minimum size = 4KB
        \item want much larger allocations for speed
        \end{itemize}
    \end{itemize}
\end{frame}

\begin{frame}{malloc()/free()/etc. as memory partitioners}
    \begin{itemize}
    \item for ``small'' objects (less than kilobytes)
    \item malloc() allocates \textit{big chunks of memory}
    \item then subdivides them on the fly
    \vspace{.5cm}
    \item different strategies to do this
    \item all need to track \textit{metadata} about allocated objects!
    \end{itemize}
\end{frame}

\begin{frame}{malloc() metadata?}
    \begin{itemize}
    \item need to:
    \vspace{.5cm}
    \item find unused chunks of memory
        \begin{itemize}
        \item even after A=malloc(), B=malloc(), C=malloc(), free(B)
        \end{itemize}
    \item figure out how big allocation is when free() is called
    \vspace{.5cm}
    \item common strategy: keep this information next to objects
        \begin{itemize}
        \item avoids needing lookup table/etc. to find it
        \item avoids needing separate allocation for metadata space
        \end{itemize}
    \end{itemize}
\end{frame}

\begin{frame}[fragile,label=heapLayout]{heap object}
\begin{tikzpicture}
\node[anchor=north east] (code) at (-1,0) {
\begin{lstlisting}
struct AllocInfo {
  bool free;
  int size;
  AllocInfo *prev;
  AllocInfo *next;
};
\end{lstlisting}
};

\tikzset{xscale=0.9}
\begin{scope}[overlay]
    \draw[stackBox,fill=black!20] (0, 1) rectangle (3, -7);

    \draw[onStack] (0, 1) rectangle (3, 0) node[midway,font=\small,align=center] {free space \\ (deleted obj.)};
    \draw[onStack,fill=white] (0, -0.0) rectangle (3, -0.5) node[midway,font=\small] (freeANext) {next};
    \draw[onStack,fill=white] (0, -0.5) rectangle (3, -1.0) node[midway,font=\small] (freeAPrev) {prev};
    \draw[onStack,fill=white] (0, -1.0) rectangle (3, -1.5) node[midway,font=\small] (freeASize) {size/free};

    \draw[very thick, red, rounded corners] (0, 1) rectangle (3, -1.5);

    \draw[onStack,fill=blue!20] (0, -1.5) rectangle (3, -3.0) node[midway,font=\small,align=center] (freeBAlloc) {new'd object};
    \draw[onStack,fill=white] (0, -3.0) rectangle (3, -3.5) node[midway,font=\small] (freeBSize) {size/free};
    
    \draw[very thick, red, rounded corners] (0, -1.5) rectangle (3, -3.5);

    \draw[onStack] (0, -3.5) rectangle (3, -5.0) node[midway,font=\small] {free space};
    \draw[onStack,fill=white] (0, -5.0) rectangle (3, -5.5) node[midway,font=\small] (freeCNext) {next};
    \draw[onStack,fill=white] (0, -5.5) rectangle (3, -6.0) node[midway,font=\small] (freeCPrev) {prev};
    \draw[onStack,fill=white] (0, -6.0) rectangle (3, -6.5) node[midway,font=\small] (freeCSize) {size/free};
    
    \draw[very thick, red, rounded corners] (0, -3.5) rectangle (3, -6.5);
    
    \draw[-Latex,blue,thick] (freeAPrev) -- ++(1.75cm,0cm) |- (freeCSize);
    \draw[-Latex,blue,thick] (freeCNext) -- ++(2.00cm,0cm) |- (freeASize);
    \draw[-Latex,blue,thick,opacity=0.5] (freeCPrev) -- ++(1.25cm,0cm) -- ++(0cm,-2cm);
    \draw[-Latex,blue,thick,opacity=0.5] (freeANext) -- ++(1.75cm,0cm) -- ++(0cm,2cm);
\end{scope}
\draw[-Latex,line width=3pt,black!50] (3.5,-2.25) -- (5.5,-2.25) node[black,midway,above,font=\small\tt] {free};
\begin{scope}[overlay,xshift=6cm,name prefix=sec-]
    \draw[stackBox,fill=black!20] (0, 1) rectangle (3, -7);

    \draw[onStack] (0, 1) rectangle (3, -5.0) node[midway,font=\small] {free space};
    \draw[onStack,fill=white] (0, -5.0) rectangle (3, -5.5) node[midway,font=\small] (freeCNext) {next};
    \draw[onStack,fill=white] (0, -5.5) rectangle (3, -6.0) node[midway,font=\small] (freeCPrev) {prev};
    \draw[onStack,fill=white] (0, -6.0) rectangle (3, -6.5) node[midway,font=\small] (freeCSize) {size/free};
    
    \draw[-Latex,blue,thick,opacity=0.5] (freeCPrev) -- ++(1.25cm,0cm) -- ++(0cm,-2cm);
    \draw[-Latex,blue,thick,opacity=0.5] (freeCNext) -- ++(1.75cm,0cm) -- ++(0cm,2cm);
\end{scope}
\end{tikzpicture}
\end{frame}

\begin{frame}[fragile,label=freeImpl]{implementing free()}
\lstset{
    style=small,
    language=C,
    moredelim={**[is][\btHL<2|handout:0>]{~2~}{~end~}},
}
\begin{lstlisting}
int free(void *object) {
    ...
    block_after = object + object_size;
    if (block_after->free) {
        /* unlink from list, about to merge with previous block */
        new_block->size += block_after->size;
        ~2~block_after->prev->next = block_after->next;~end~
        block_after->next->prev = block_after->prev;
    }
    ...
}
\end{lstlisting}
\begin{itemize}
\item<2> \large \myemph<2>{arbitrary memory write}
\end{itemize}
\end{frame}



\subsection{bookkeeping to pointer subterfuge}
\usetikzlibrary{arrows.meta,decorations.pathreplacing,patterns}

\tikzset{
    stackBox/.style={very thick},
    onStack/.style={thick},
}
\begin{frame}[fragile,label=vulnHeapSmash]{vulnerable code}
\lstset{
    style=small,
    language=C,
    moredelim={**[is][\btHL<4|handout:0>]{~2~}{~end~}},
    moredelim={**[is][\btHL<2-3|handout:0>]{~3~}{~end~}},
    moredelim={**[is][\btHL<6-|handout:0>]{~6~}{~end~}},
}
\begin{tikzpicture}
\node[anchor=north east] (code) at (-1,0) {
\begin{lstlisting}
char *buffer = malloc(100);
... 
~2~strcpy(buffer, attacker_supplied);~end~
... 
~3~free(buffer);~end~
~6~free(other_thing);~end~
...
\end{lstlisting}
};

\tikzset{xscale=0.9}
\begin{scope}[overlay]
    \draw[stackBox,fill=black!20] (0, 1) rectangle (3, -7);

    \begin{pgfonlayer}{fg}
    \draw[very thick, orange, rounded corners] (0, 1) rectangle (3, -1.5);
    \draw[very thick, orange, rounded corners] (0, -1.5) rectangle (3, -3.5);
    \draw[very thick, orange, rounded corners] (0, -3.5) rectangle (3, -6.5);
    \draw[very thick,-Latex] (3.1, -6.5) -- ++(0, 3) node[sloped,right] {incr. addrs};
    \end{pgfonlayer}

    \draw[onStack] (0, 1) rectangle (3, 0) node[midway,font=\small] {free space};
    \draw[onStack,fill=white] (0, -0.0) rectangle (3, -0.5) node[midway,font=\small] (freeANext) {next};
    \draw[onStack,fill=white] (0, -0.5) rectangle (3, -1.0) node[midway,font=\small] (freeAPrev) {prev};
    \draw[onStack,fill=white] (0, -1.0) rectangle (3, -1.5) node[midway,font=\small] (freeASize) {size/free};

    \draw[onStack,fill=blue!20] (0, -1.7) rectangle (3, -3.0) node[midway,font=\small,align=center] (freeBAlloc) {alloc'd object};
    \begin{visibleenv}<4->
    \fill[pattern color=red,pattern=north west lines] (0, -0.0) rectangle (3, -3.0);
    \end{visibleenv}
    \draw[onStack,fill=white] (0, -3.0) rectangle (3, -3.5) node[midway,font=\small] (freeBSize) {size/free};

    \draw[onStack] (0, -3.5) rectangle (3, -5.0) node[midway,font=\small] {free space};
    \draw[onStack,fill=white] (0, -5.0) rectangle (3, -5.5) node[midway,font=\small] (freeCNext) {next};
    \draw[onStack,fill=white] (0, -5.5) rectangle (3, -6.0) node[midway,font=\small] (freeCPrev) {prev};
    \draw[onStack,fill=white] (0, -6.0) rectangle (3, -6.5) node[midway,font=\small] (freeCSize) {size/free};
    
    \draw[-Latex,blue,thick] (freeAPrev) -- ++(1.75cm,0cm) -- ++ (0cm, -1cm) -- ++ (4cm,0cm); %|- (freeCSize);
    \draw[-Latex,blue,thick] (freeCNext) -- ++(2.00cm,0cm) -- ++ (0cm, 1cm) -- ++ (4cm,0cm); % |- (freeASize);
    \draw[-Latex,blue,thick,opacity=0.5] (freeCPrev) -- ++(1.25cm,0cm) -- ++(0cm,-2cm);
    \draw[-Latex,blue,thick,opacity=0.5] (freeANext) -- ++(1.75cm,0cm) -- ++(0cm,2cm);

    \begin{visibleenv}<4->
        \begin{scope}[xshift=-5cm,yshift=-3cm]
        \tikzset{gotBox/.style={thick,fill=orange!30}}
        \draw[gotBox,alt=<6>{fill=red!10}] (0, -1) rectangle (4, -1.5) node[midway,font=\small] (freeEntry) {GOT entry: free};
        \draw[gotBox] (0, -1.5) rectangle (4, -2) node[midway,font=\small] {GOT entry: malloc };
        \draw[gotBox] (0, -2) rectangle (4, -2.5) node[midway,font=\small] (printfEntry) {GOT entry: printf };
        \draw[gotBox] (0, -2.5) rectangle (4, -3.0) node[midway,font=\small] (gotFopen) {GOT entry: fopen };
        \end{scope}
        \draw[-Latex,red,ultra thick,dashed] (freeAPrev) -- ++(-2cm,0cm) |- ([yshift=-.25cm]printfEntry.east);
        \draw[-Latex,red,ultra thick,dashed] (freeANext) -- ++(-2.5cm,0cm) -- ++(0cm,-2.5cm) -- ++(-.5cm,0cm) node [black,solid,draw,left,align=left,fill=white] (scode) {shellcode/etc.};
    \end{visibleenv}
    \begin{visibleenv}<3>
        \draw[ultra thick,decorate,decoration={brace}] (3.15, 1) -- ++ (0, -2.5) node[midway,right,font=\small,align=left] {to be removed \\ from linked list};
    \end{visibleenv}
    \begin{visibleenv}<2>
        \draw[ultra thick,decorate,decoration={brace}] (3.15, 1) -- ++ (0, -5.5) node[midway,right,font=\small,align=left] {free() tries \\ to merge these};
    \end{visibleenv}
    \begin{visibleenv}<5->
        \begin{scope}[xshift=-5cm,yshift=-3cm]
        \draw[blue,opacity=0.8] (-3, -2) rectangle (0, -2.5) node[midway,font=\small] {prev->size/free};
        \draw[blue,opacity=0.8] (-3, -1.5) rectangle (0, -2) node[midway,font=\small] {prev->prev};
        \draw[blue,opacity=0.8] (-3, -1) rectangle (0, -1.5) node[midway,font=\small] {prev->next};
        \end{scope}
        \draw[-Latex,blue,ultra thick,dotted] (freeEntry.north) -- ++(0cm,.1cm) -| (scode.south);
        \node[draw,very thick] at ([xshift=0cm,yshift=-1.5cm]gotFopen.south west) {
            \lstinline|block_after->prev->next = block_after->next|
        };
    \end{visibleenv}
\end{scope}
\end{tikzpicture}
\end{frame}




% FIXME: exercise, using linked-list heap exploit to write something
\subsection{exericse}
\usetikzlibrary{arrows.meta,matrix,fit}
\begin{frame}[fragile,label=heapSubterExercise]{heap overflow exercise}
\begin{lstlisting}[language=C++,style=script]
void operator delete(void *p) {
    ...
    block_after->prev->next = block_after->next;
    ...
}
...
class MyBuffer : public GenericMyBuffer {
public:
    virtual void store(const char *p) override {
        strcpy(buffer, p);
    }
private:
    char buffer[64];
};
...
    GenericMyBuffer *a = new MyBuffer;
    ...
    a->store(attacker_controlled);
    ...
    delete a;
    ...
\end{lstlisting}
\begin{tikzpicture}[overlay,remember picture]
\coordinate (place) at ([yshift=-1.5cm,xshift=-1cm]current page.north east);
\matrix[tight matrix,nodes={text width=3cm,font=\fontsize{9}{10}\selectfont,thick},thick,anchor=north east,label={north:heap object layout}]  (diag) at (place) {
    |[draw=none,align=center,font=\it]| when free \& |[draw=none,align=center]| when used \\
    |[fill=yellow!10]| size+free (8 B) \& |[fill=yellow!10,alias=arrowSide]| size+free (8 B) \\
    |[fill=blue!10]| next pointer (8 B) \& |[fill=violet!10]| vtable pointer (8 B) \\
    |[fill=blue!10]| prev pointer (8 B) \& |[alias=bufPt1]| ~ \\
    |[alias=unusedPt1]| ~                  \& |[alias=bufPt2]| ~ \\
    |[alias=unusedPt2]| ~                  \& |[alias=bufPt3]| ~ \\
    |[alias=unusedPt3]| ~                  \& |[fill=green!10]| unused space (16 B) \\
    |[alias=unusedPt4]| \& (next size+free)   \\
     (next size+free) \\
};
\draw[very thick,-Latex] ([xshift=.25cm]arrowSide.north east) -- ++(0cm, -1cm);
\node[inner sep=0mm,fill=violet!10,fit=(bufPt1) (bufPt2) (bufPt3),font=\small,draw,thick] {buffer (64B)};
\node[inner sep=0mm,fill=green!10,fit=(unusedPt1) (unusedPt2) (unusedPt3) (unusedPt4),font=\small,draw,thick] {unused space \\ (?? B)};
\begin{visibleenv}<1>
\node[anchor=north west,align=left,draw,very thick,font=\fontsize{12}{13}\selectfont] at ([xshift=-1cm]diag.south west) {
    exercise 1: \\
    to attack this buffer overflow \\
    by overwriting the heap data structures \\
    does it matter if space after \texttt{a} \\
    is already free or not?
};
\end{visibleenv}
\begin{visibleenv}<2>
\node[anchor=north west,align=left,draw,very thick,font=\fontsize{12}{13}\selectfont] at ([xshift=-1cm]diag.south west) {
    exercise 2:
    if \texttt{a} at address 0x10000, \\ 
    and attacker wants to overwrite \\
    value at address 0x20000 with 0x30000, \\
    where should attacker put 0x20000, 0x30000\\
    in \texttt{attacker\_controlled}? \\

};
\end{visibleenv}
\end{tikzpicture}
\end{frame}


\section{alternate malloc designs?}
\begin{frame}{other malloc designs?}
\begin{itemize}
\item there are a lot of different malloc/new implementations
\item often multiple free lists
\item free block list might not be kept with linked list
\vspace{.5cm}
\item some place metadata next to allocations like this
\item some keep it separate
\vspace{.5cm}
\item usually performance determines which is chosen
\end{itemize}
\end{frame}


\section{double-free}
\usetikzlibrary{arrows.meta,patterns}

\tikzset{
    stackBox/.style={very thick},
    onStack/.style={thick},
    frameOne/.style={fill=blue!15},
    frameTwo/.style={fill=red!15},
    markLine/.style={blue!50!black},
    markLineB/.style={red!90!black},
    hiLine/.style={red!90!black},
}


\begin{frame}[fragile,label=dblFree]{double-frees}
\lstset{
    style=small,
    language=C,
    moredelim={**[is][\btHL<2|handout:0>]{~2~}{~end~}},
    moredelim={**[is][\btHL<3|handout:0>]{~3~}{~end~}},
    moredelim={**[is][\btHL<4|handout:0>]{~4~}{~end~}},
}
\begin{tikzpicture}
\node[anchor=north east] (code) at (-1,0) {
\begin{lstlisting}
~2~free(thing);~end~
~3~free(thing);~end~
char *~4~p = malloc(...);~end~
// p points to next/prev
//   on list of avail.
//   blocks
strcpy(p, attacker_controlled);
malloc(...);
char *q = malloc(...);
// q points to attacker-
//   chosen address
strcpy(q, attacker_controlled2);
...
\end{lstlisting}
};

\tikzset{xscale=0.9}
\begin{scope}[overlay]
    \draw[stackBox,fill=black!20] (0, 1) rectangle (3, -7);

    \draw[onStack] (0, 1) rectangle (3, 0) node[midway,font=\small] {free space};
    \draw[onStack,fill=white] (0, -0.0) rectangle (3, -0.5) node[midway,font=\small] (freeANext) {next};
    \draw[onStack,fill=white] (0, -0.5) rectangle (3, -1.0) node[midway,font=\small] (freeAPrev) {prev};
    \draw[onStack,fill=white] (0, -1.0) rectangle (3, -1.5) node[midway,font=\small] (freeASize) {size};

    \draw[onStack,fill=blue!20] (0, -1.7) rectangle (3, -3.0) node[midway,font=\small,align=center] (freeBAlloc) {alloc'd object};
    \draw[onStack,fill=white] (0, -3.0) rectangle (3, -3.5) node[midway,font=\small] (freeBSize) {size};

    \draw[onStack,fill=yellow!20] (0, -3.5) rectangle (3, -6.0) node[midway,font=\small,align=center,yshift=.5cm] {alloc'd object\\
                \tt thing\only<4->{/p}};
    \begin{visibleenv}<2->
    \draw[onStack,fill=yellow!10,dashed] (0, -5.5) rectangle (3, -6.0) node[midway,font=\small] (freeCPrev) {prev};
    \draw[onStack,fill=yellow!10,dashed] (0, -5.0) rectangle (3, -5.5) node[midway,font=\small] (freeCNext) {next};
    \end{visibleenv}
    \begin{visibleenv}<2>
    \draw[-Latex,blue,thick,opacity=0.5] (freeCPrev) -- ++(1.75cm,0cm) node[right] {NULL};
    \draw[-Latex,blue,thick,opacity=0.5] (freeCNext) -- ++(2.25cm,0cm) -- ++(0cm,2cm);
    \draw[Latex-,blue,thick,opacity=0.5] (freeCPrev.west) -- ++(-1cm,0cm) -- ++(0cm,-1cm) node[below,font=\small] {list head};
    \end{visibleenv}
    \begin{visibleenv}<3->
    \draw[-Latex,blue,thick,opacity=0.5] (freeCPrev) -- ++(1.75cm,0cm) node[right] {NULL};
    \draw[-Latex,blue,thick,opacity=0.5] (freeCNext) -- ++(2.25cm,0cm) -- ++(0cm,.35cm) -- ++(-4cm,0cm) |- (freeCPrev.west);
    \draw[Latex-,blue,thick,opacity=0.5] (freeCPrev.west) -- ++(-1cm,0cm) -- ++(0cm,-1cm) node[below,font=\small] {list head};
    \end{visibleenv}
    \draw[onStack,fill=white] (0, -6.0) rectangle (3, -6.5) node[midway,font=\small] (freeCSize) {size};
    
    \draw[-Latex,blue,thick,opacity=0.5] (freeAPrev) -- ++(1.75cm,0cm) -- ++(0cm,-2cm);
    \draw[-Latex,blue,thick,opacity=0.5] (freeANext) -- ++(2.25cm,0cm) -- ++(0cm,2cm);
\end{scope}
    \begin{visibleenv}<4>
        \node[draw=red, ultra thick, anchor=east,align=left,fill=white] at (-.5, -4) {
            malloc returns something \myemph{still on free list} \\
            because double-free made \myemph{loop} in linked list
        };
    \end{visibleenv}
\end{tikzpicture}
\end{frame}


% FIXME: as pictures
\begin{frame}[fragile,label=dblFreeExpand]{double-free expansion}
\lstset{
    style=smaller,
    language=C,
    moredelim={**[is][\btHL<2|handout:0>]{~2~}{~end~}},
    moredelim={**[is][\btHL<3|handout:0>]{~3~}{~end~}},
    moredelim={**[is][\btHL<4|handout:0>]{~4~}{~end~}},
    moredelim={**[is][\btHL<5|handout:0>]{~5~}{~end~}},
    moredelim={**[is][\btHL<6|handout:0>]{~6~}{~end~}},
}
\begin{tikzpicture}
\node[anchor=north east] (code) at (-1,0) {
\begin{lstlisting}
// free/delete 1:
~2~double_freed->next = first_free;~end~
~2~first_free = chunk;~end~
// free/delete 2:
~3~double_freed->next = first_free;~end~
~3~first_free = chunk~end~
// malloc/new 1:
~4~result1 = first_free;~end~
~4~first_free = first_free->next;~end~
// + overwrite:
~4~strcpy(result1, ...);~end~
// malloc/new 2:
~5~first_free = first_free->next;~end~
// malloc/new 3: 
result3 = first_free;
strcpy(result3, ...);
\end{lstlisting}
};
\draw[stackBox] (0,0) rectangle (5, -2);
\draw[onStack,fill=blue!20] (0,0) rectangle (5, -1) node[midway,font=\small] {next / double free'd object};
\draw[onStack] (0,-1) rectangle (5, -2) node[midway,font=\small] {size};
\draw[stackBox] (0, -4) rectangle (5, -5) node[midway,font=\small,align=center] {first\_free \\ (global)};
\draw[stackBox,dashed] (0.5, -2.5) rectangle (5.5, -3.5) node[midway,font=\small,align=center] {(original first free)};

\begin{visibleenv}<1>
\draw[-Latex,blue,thick] (0, -4.5) -- ++(-0.5cm,0cm) |- (0.5, -3);
\end{visibleenv}
\begin{visibleenv}<2-4>
\draw[-Latex,blue,thick] (0, -4.5) -- ++(-0.5cm,0cm) |- (0, -1.5);
\end{visibleenv}
\begin{visibleenv}<2>
\draw[-Latex,blue,thick] (5, -0.5) -- ++(1.5cm,0cm) |- (5.5, -3);
\end{visibleenv}

\begin{visibleenv}<3->
\draw[-Latex,blue,thick] (5, -0.5) -- ++(0.5cm,0cm) |- (5, -1.5);
\end{visibleenv}

\begin{visibleenv}<4->
\fill[pattern=north west lines,pattern color=red] (0, 0) rectangle (5, -1);
\draw[red,dashed,ultra thick,-Latex] (5,-0.5) -- ++(1cm, 0cm) |- (3, -5.5);
\draw[stackBox,fill=orange!30] (0, -5.25) rectangle (3, -6.25) node[midway,font=\small] {GOT entry: free};
\end{visibleenv}
\begin{visibleenv}<4->
\node[overlay,text=black,draw=red!80,thick,font=\small\tt,anchor=south] at (3, 0) { first/second malloc };
\end{visibleenv}

\begin{visibleenv}<5->
\draw[-Latex,blue,thick] (0, -4.5) -- ++(-1cm, 0cm) |- (0, -5.75);
\node[text=black,draw=red!80,thick,font=\small\tt,anchor=north west] at (3, -6) { third malloc };
\end{visibleenv}
\end{tikzpicture}
\end{frame}

\begin{frame}{double-free notes}
    \begin{itemize}
    \item this attack has apparently not been possible for a while
    \item most malloc/new's \myemph{check for double-frees} explicitly
        \begin{itemize}
        \item (e.g., look for a bit in {\tt size} data)
        \end{itemize}
    \item prevents this issue --- also catches programmer errors
    \item pretty cheap
    \end{itemize}
\end{frame}


% FIXME: exercise, double-free usage?
\subsection{exercise}
\begin{frame}[fragile,label=doubleFreeExer]{double-free exercise}
\begin{tikzpicture}
\node[draw] (code) { 
\begin{lstlisting}[language=C++,style=script]
free(...) {
    freed->next = first_free
    first_free = freed;
}
malloc(...) {
    if (can use first free) {
        void *to_return = first_free;
        first_free = first_free->next;
        return to_return;
    }
}
vulnerable() {
    char *p = malloc(100);
    free(p);
    free(p);
    char *q = malloc(100);
    char *r = malloc(100);
    strlcpy(q, attacker_input1, 100);
    char *s = malloc(100);
    strlcpy(r, attacker_input2, 100);
    strlcpy(s, attacker_input3, 100);
}
\end{lstlisting}
};
\node[anchor=north west,align=left] at (code.north east) {
To do memory[0x123456] $\leftarrow$ 0x789abc \\
what should input1/input2/input3 be?
};
\end{tikzpicture}
\end{frame}

 

\section{use-after-free}
\begin{frame}{use-after-free}
\end{frame}


\tikzset{
    stackBox/.style={very thick},
    onStack/.style={thick},
}
\begin{frame}[fragile,label=vulnUAF]{vulnerable code}
\lstset{
    language=C++,
    style=smaller,
    moredelim={**[is][\btHL<2|handout:0>]{~2~}{~end~}},
}
\begin{tikzpicture}
\node[anchor=north east] (code) at (-1,0) {
\begin{lstlisting}
class Foo {
    ...
};
Foo *the_foo;
the_foo = new Foo;
...
delete the_foo;
...
something_else = new Bar(...);
the_foo->something();
\end{lstlisting}
};
\node[draw,anchor=north west,align=left] at (-3,0) {
    {\tt something\_else} likely where {\tt the\_foo} was
};
\begin{visibleenv}<2>
\draw[stackBox] (0, -2) rectangle (3, -5);
\draw[stackBox] (4, -2) rectangle (7, -5);
\draw[onStack,fill=blue!20] (0, -2) rectangle (3, -3) node[midway,align=center,font=\small] { vtable ptr (Foo) };
\draw[onStack,fill=blue!20] (0, -3) rectangle (3, -5) node[midway,align=center,font=\small] {data for Foo };
\draw[onStack,fill=yellow!20] (4, -2) rectangle (7, -3) node[midway,align=center,font=\small] { vtable ptr (Bar)? \\
                                                                                   other data? };
\draw[onStack,fill=yellow!20] (4, -3) rectangle (7, -5) node[midway,align=center,font=\small] { data for Bar  };
\end{visibleenv}
\end{tikzpicture}
\end{frame}



\subsection{reuse observation}
\begin{frame}{easy heap reuse}
    \begin{itemize}
    \item strategy of keeping linked list of free items?
    \item simplest way to write code:
        \begin{itemize}
        \item free() = add to head of list
        \item malloc() = scan from head of list
        \end{itemize}
    \item if done, makes it easy to predict what will reuse allocation
    \end{itemize}
\end{frame}

\begin{frame}{complicating easy reuse}
    \begin{itemize}
    \item usually can't precisely control what is allocated/free'd
    \item some allocators mostly use different ordering than last in, first-out
        \begin{itemize}
        \item example: lowest to highest address
        \end{itemize}
    \item often different lists for different size ranges/threads
    \item freeing big object may make space for multiple future allocations
    \end{itemize}
\end{frame}

\begin{frame}{aside: heap feng shui/grooming}
    \begin{itemize}
    \item \url{http://www.phreedom.org/research/heap-feng-shui/heap-feng-shui.html}
    \item one idea:
    \item allocate lots of objects to fill up likely holes
        \begin{itemize}
        \item choose sizes/etc. based on allocator
        \item allocators usually have separate `regions' for different sizes
        \end{itemize}
    \item allocate three objects of appropriate size
        \begin{itemize}
        \item probably three consecutive allocations
        \end{itemize}
    \item free `middle' object + expect it to be reused
    \end{itemize}
\end{frame}


\subsection{pattern}

\begin{frame}{exploiting use after-free}
\begin{itemize}
\item trigger many ``bogus'' frees; then
\item allocate many things of same size with ``right'' pattern  
    \begin{itemize}
    \item pointers to shellcode?
    \item pointers to pointers to {\tt system()}?
    \item objects with something useful in VTable entry?
    \end{itemize}
\item trigger use-after-free thing
\end{itemize}
\end{frame}



\subsection{consistency?}
\begin{frame}{consistency?}
    \begin{itemize}
    \item how to predict what gets reused?
    \vspace{.5cm}
    \item use debugger + print out all the addreses
        \begin{itemize}
        \item look for duplicates
        \item probably fixed number of allocations before duplicate
        \end{itemize}
    \item allocators like reusing `perfectly size' space
        \begin{itemize}
        \item free something + immediately allocate same size
        \end{itemize}
    \item trigger use-after-free bug lots of times
        \begin{itemize}
        \item one of them will match up
        \end{itemize}
    \end{itemize}
\end{frame}


\subsection{exercise: info leak}
\begin{frame}[fragile]{exercise}
\begin{Verbatim}[fontsize=\fontsize{9}{10}]
struct Codec {
    const char *name; void (*DecodeFrame)(...); void (*Seek)(...); ...
};
struct Codec H264 = { "H264", ... }, H265 = { "H265", ...}, MJPEG = { ... };
struct Video {
    struct Codec *codec; /* one of H264, ... */
    const char *filename;
    int framerate, width, height, frames; FILE *fh;
    ...
};
struct BrowserWindow {
    int num_tabs; int active_tab_index; struct BrowserTab *all_tabs; 
    ...
};
struct BrowserTab {
    struct BrowserWindow *window;
    char current_url[1024];
    ...
};
\end{Verbatim}
\begin{itemize}
\item \small Suppose UAF of BrowserTab being overwritten by new Video object\ldots
\item \small To break ASLR, what methods to get data from BrowserTab would be useful?
\end{itemize}
\end{frame}


\subsection{exercise: subterfuge}
\begin{frame}[fragile]{exercise}
\begin{Verbatim}[fontsize=\fontsize{9}{10}]
struct String {
    size_t alloc_size;
    size_t used_size;
    char *data;
    bool is_utf8;
};
struct FileInfo {
    const char *name;
    time_t creation_time;
    time_t modification_time;
    FILE *file_data;
}
\end{Verbatim}
\begin{itemize}
\item If we have a String + FileInfo in same place from use-after-free \\
What sequence of String/FileInfo operations to modify memory at 0x12345678?
\end{itemize}
\end{frame}


\subsection{exercise: vtable}
\begin{frame}[fragile,label=uafExericse]{exercise}
\begin{tikzpicture}
\node[draw,very thick,label={north:vuln. code}] (a) {
\begin{lstlisting}[language=C++,style=script]
std::istream *in =
    new std::ifstream("in.txt");
...
delete in;
...
char *other_buffer =
    new char[strlen(INPUT) + 1];
strcpy(other_buffer, INPUT);
...
char c = in->get();
\end{lstlisting}
};
\node[draw,very thick,label={north:ifstream internals},anchor=north west] (b) at (a.north east) {
\begin{lstlisting}[language=C++,style=script]
class istream {
    ...
    int get() { ... buf->uflow(); ... }
    streambuf *buf;
    ~istream() { delete buf; }
};
class streambuf {
    ...
protected:
    virtual type_for_char uflow() = 0;
        /* called to get next char*/
};
class _File_streambuf : public streambuf { ... }
\end{lstlisting}
};
\end{tikzpicture}
\begin{itemize}
\item attacker goal: change what uflow() call does
\item Q1: {\small assuming same size $\rightarrow$ likely to get same address}, what size for attacker to choose for \texttt{INPUT}?
\item Q2: where in INPUT to place pointer to code to run?
\end{itemize}
\end{frame}


\subsection{example}
\usetikzlibrary{calc,fit,matrix}
\begin{frame}{real UAF exploitable bug}
    \begin{itemize}
        \item 2012 bug in Google Chrome
        \item exploitable via JavaScript
        \item discovered/proof of concept by PinkiePie
        \item allowed arbitrary code execution via VTable manipulation
    \end{itemize}
\end{frame}


\begin{frame}[fragile,label=UAFTriggering]{UAF triggering code}
\lstset{
    language=JavaScript,
    style=smaller,
    moredelim={**[is][\btHL<2|handout:0>]{~2~}{~end~}},
    moredelim={**[is][\btHL<3-4|handout:0>]{~3~}{~end~}},
    moredelim={**[is][\btHL<4|handout:0>]{~4~}{~end~}},
}
\begin{tikzpicture}
\node[anchor=north east] (code) at (0, 0) {
\begin{lstlisting}
// in HTML near this JavaScript:
// <video id="vid"> (video player element)
function source_opened() {
  buffer = ms.addSourceBuffer('video/webm; codecs="vorbis,vp8"');
  ~2~vid.parentNode.removeChild(vid);~end~
  gc(); // force garbage collector to run now
  // garbage collector frees unreachable objects
  // (would be run automatically, eventually, too)
  // buffer now internally refers to delete'd player object
  ~3~buffer.timestampOffset = 42;~end~
}
ms = new WebKitMediaSource();
ms.addEventListener('webkitsourceopen', source_opened);
vid.src = window.URL.createObjectURL(ms);
\end{lstlisting}
};
    \begin{visibleenv}<4->
        \node[fill=white,opacity=0.6,fit=(code)] {};
        \node[draw=red,ultra thick,anchor=north east,fill=white,overlay] (cppCode) at (-.25, .125) { 
\lstset{
    language=C++,
    style=smaller,
    moredelim={**[is][\btHL<2|handout:0>]{~2~}{~end~}},
    moredelim={**[is][\btHL<3-4|handout:0>]{~3~}{~end~}},
    moredelim={**[is][\btHL<4|handout:0>]{~4~}{~end~}},
    moredelim={**[is][\btHL<5|handout:0>]{~5~}{~end~}},
}
\begin{lstlisting}
// implements JavaScript buffer.timestampOffset = 42
void SourceBuffer::setTimestampOffset(...) {
     if (m_source->setTimestampOffset(...))
        ...
}
bool MediaSource::setTimestampOffset(...) {
    // m_player was deleted when video player element deleted
    // but this call does *not* use a VTable
    if (!~4~m_player~end~->sourceSetTimestampOffset(id, offset)) 
        ...
}
bool MediaPlayer::sourceSetTimestampOffset(...) {
    // m_private deleted when MediaPlayer deleted
    // this *is* a VTable-based call
    return ~5~m_private~end~->sourceSetTimestampOffset(id, offset);
}
\end{lstlisting}
    };
    \end{visibleenv}
\end{tikzpicture}
\imagecredit{via \url{https://bugs.chromium.org/p/chromium/issues/detail?id=162835}}
\end{frame}

\begin{frame}[fragile,label=theExploit]{UAF exploit (approx. pseudocode)}
\begin{lstlisting}[language=JavaScript,style=smaller]
... /* use information leaks to find relevant addresses */ 
buffer = ms.addSourceBuffer('video/webm; codecs="vorbis,vp8"');
vid.parentNode.removeChild(vid);
vid = null;
gc();
// allocate object to replace m_private
var array = new Uint32Array(168/4);
// allocate object to replace m_player
// type chosen to keep m_private pointer unchanged
rtc = new webkitRTCPeerConnection({'iceServers': []});
array[0] = ... /* fill in array with chosen values */
// trigger VTable Call that uses chosen address
buffer.timestampOffset = 42;
\end{lstlisting}
\end{frame}

\begin{frame}{type confusion}
    \begin{tikzpicture}
    \matrix[tight matrix,nodes={text width=6.8cm,text depth=.1ex,font=\small\tt},anchor=north west,
            label={north:{\tt MediaPlayer} (deleted but used)}] (PlayerVT) at (0, 0) {
        m\_private {\normalfont (pointer to PlayerImpl)} \\
        m\_timestampOffset {\normalfont (double)} \\
    };
    \matrix[tight matrix,nodes={text width=6.8cm,text depth=.1ex,font=\small\tt},anchor=north west,
            label={north:{\tt webkitRTC\ldots} (replacement)}] (SomethingVT) at (8, 0) {
        (something not changed) \\
        m\_??? {\normalfont (pointer)} \\
        \ldots \\
    };
    \matrix[tight matrix,nodes={text width=6.8cm,text depth=.1ex,font=\small\tt},anchor=north west,
            label={north:{\tt PlayerImpl} (deleted but used)}] (PlayerImplVT) at (0, -2) {
        VTable pointer \\
        \ldots \\
    };
    \matrix[tight matrix,nodes={text width=6.8cm,text depth=.1ex,font=\small\tt},anchor=north west,
            label={north:array of 32-bit ints (replacement)}] (ArrayVT) at (8, -2) {
        array[0], array[1] \\
        array[2], array[3] \\
        \ldots \\
    };
    \end{tikzpicture}
\end{frame}

\begin{frame}{missing pieces: information disclosure}
    \begin{itemize}
        \item need to learn address to set VTable pointer to
            \begin{itemize}
            \item (and other addresses to use)
            \end{itemize}
        \item allocate types other than \texttt{Uint32Array}
        \item rely on confusing between different types, e.g.
    \end{itemize}
    \begin{tikzpicture}
    \matrix[tight matrix,nodes={text width=6.8cm,text depth=.1ex,font=\small\tt},anchor=north west,
            label={north:{\tt MediaPlayer} (deleted but used)}] (PlayerVT) at (0, 0) {
        m\_private {\normalfont (pointer to PlayerImpl)} \\
        m\_timestampOffset {\normalfont (double)} \\
    };
    \matrix[tight matrix,nodes={text width=6.8cm,text depth=.1ex,font=\small\tt},anchor=north west,
            label={north:{\tt Something} (replacement)}] (SomethingVT) at (8, 0) {
        \ldots \\
        m\_buffer {\normalfont (pointer)} \\
    };
    \end{tikzpicture}
    \begin{itemize}
    \item allows reading timestamp value to get a pointer's address
    \end{itemize}
\end{frame}


\subsection{JS and similar interfaces v use-after-free}

\begin{frame}{use-after-free easy cases}
\begin{itemize}
    \item common problem for JavaScript implementations
    \item use-after-free'd object often some complex C++ object
        \begin{itemize}
            \item example: representation of video stream
        \end{itemize}
    \item exploits can \myemph{choose type of object that replaces}
        \begin{itemize}
            \item allocate that kind of object in JS
        \end{itemize}
    \item can often arrange to read/write vtable pointer
        \begin{itemize}
            \item depends on layout of thing created
            \item easy examples: string, array of floating point numbers
        \end{itemize}
\end{itemize}
\end{frame}


    % FIXME: add slide on using debugger, finding memory reuse pattern
    % FIXME: add slide on "heap spary" type ideas
    % FIXME: jemalloc case study


\section{backup slides}
\begin{frame}{backup slides}
\end{frame}

\end{document}
