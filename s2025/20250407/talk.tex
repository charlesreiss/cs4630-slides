\date{}
\title{}
\date{}
\usepackage[outputdir=latex.out]{minted}
\begin{document}
\begin{frame}
    \titlepage
\end{frame}


\makeatletter
\newenvironment{btHighlight}[1][]
{\begingroup\tikzset{bt@Highlight@par/.style={#1}}\begin{lrbox}{\@tempboxa}}
{\end{lrbox}\bt@HL@box[bt@Highlight@par]{\@tempboxa}\endgroup}

\ifdefined\NewDocumentCommand
\NewDocumentCommand\btHL{D<>{all} +m}{%
  \only<#1>{\begin{btHighlight}[#2]\bgroup\aftergroup\bt@HL@endenv}%
}
\else
\newcommand<>\btHL[1][]{%
  \only#2{\begin{btHighlight}[#1]\bgroup\aftergroup\bt@HL@endenv}%
}
\fi
\def\bt@HL@endenv{%
  \end{btHighlight}%   
  \egroup %
}
\tikzset{
    btHLbox/.style={
        fill=orange!30,outer sep=0pt,inner xsep=1pt, inner ysep=0pt, rounded corners=3pt
    },
}
\newcommand{\bt@HL@box}[2][]{%
  \tikz[#1]{%
    \pgfpathrectangle{\pgfpoint{1pt}{0pt}}{\pgfpoint{\wd #2}{\ht #2}}%
    \pgfusepath{use as bounding box}%
    \node[text width={},draw=none,anchor=base west, btHLbox, minimum height=\ht\strutbox+1pt,#1]{\raisebox{1pt}{\strut}\strut\usebox{#2}};
  }%
}

\lst@CCPutMacro
    \lst@ProcessOther {"2A}{%
      \lst@ttfamily 
         {\raisebox{2pt}{*}}% used with ttfamily
         {\raisebox{2pt}{*}}}% used with other fonts
    \@empty\z@\@empty

\lstdefinelanguage
   [x8664gas]{Assembler}     % add a "x64" dialect of Assembler
   [x86masm]{Assembler} % based on the "x86masm" dialect
   % with these extra keywords:
   {morekeywords={CDQE,CQO,CMPSQ,CMPXCHG16B,JRCXZ,LODSQ,MOVSXD,%
                  POPFQ,PUSHFQ,SCASQ,STOSQ,IRETQ,RDTSCP,SWAPGS,.TEXT,.STRING,.ASCIZ,%
                  BEQ,LW,SW,LB,SB,ADDIU,J,BEQZ,BNEZ,BNE,%
                  MOVUPD,MULPD,MOVSD,MULSD,%
                  SHLADD,MOV,CMP.LT,TBIT.NZ,BR.RET.SPTK.MANY,%
                  ADDQ,POPQ,PUSHQ,RRMOVQ,MRMOVQ,RMMOVQ,IRMOVQ,%
                  <-,LL,SC,ADDI,ADDL,VMOVDQA,ADDQ,CMPL,JB,JBE,MOVL,CLTQ,%
                  MOVW,PUSHW,MOV,ADD,SUB,INT,PUSH,MOV,ADD,REP,MOVSB,%
                  TESTQ,CMPQ,MOVL,MOVQ,ADDQ,JMPQ,XORQ,%
                  LEAQ,LEAL,LEA,RETQ,RET,POPL,POPW,PUSHL,PUSHW,%
                  LEAW,%
                  SUBQ,SYSCALL,.ASCII,CALLQ,MOVSLQ,JMP,ANDQ,SHRQ,MOVB,INCQ,TESTL,XORL,%
                  SHRL,LEAL,SARL,SUBL,IMULL,IMULQ,MOVDQU,PADDD,XORL,%
                  MOVZBL,MOVZB,SHRB,SRAL,SHRL,ANDL,%
                  CMOVNS,SRAL,SRAQ,MOVZBW,MOVZBQ,%
                  PADDW,PADDQ,MODUPS,MOVAPD,%
                  MOVL,RET,.GLOBL,%
		  PAUSE,LFENCE,JMP,%
                  },
    deletekeywords={eax,ebx,sp,si,cx,di,ds,cs,es,fs,dx,ax,bx,al,esi,ebp,ecx,rip,eip,edx,edi,rdi,esp},
    deletekeywords=[2]{size},
    alsoletter={\%},
    alsoother={()},
    emphstyle={\color{violet!50!black}},
    emph={\%rax,\%rbx,\%rcx,\%rdx,\%r8,\%r9,\%r10,\%r11,\%r12,\%r13,\%r14,\%r15,\%eax,\%ebx,\%sp,\%si,\%cx,\%di,\%ds,\%cs,\%es,\%fs,\%dx,\%ax,\%bx,\%al,\%esi,\%ebp,\%ecx,\%rip,\%eip,\%edx,\%edi,\%rdi,\%esp,\%rsp},
    %moreemph={eax,ebx,sp,si,cx,di,ds,cs,es,fs,dx,ax,bx,al,esi,ebp,ecx,rip,eip,edx,edi,rdi,esp},
    morecomment=[l]{\#},
    morecomment=[l]{\/\/},
    morecomment=[s]{/*}{*/},
    sensitive=false,
    keepspaces=true} % et

\lstalias[]{myasm}[x8664gas]{Assembler}

\lstdefinelanguage{JavaScript}{
  keywords={typeof, new, true, false, catch, function, return, null, catch, switch, var, if, in, while, do, else, case, break},
  ndkeywords={class, export, boolean, throw, implements, import, this},
  sensitive=false,
  comment=[l]{//},
  morecomment=[s]{/*}{*/},
  morestring=[b]',
  morestring=[b]"
}

\newcommand{\keywordstyle}{\sourcecodeprolight\bfseries\color{blue!30!black}}
\newcommand{\stringstyle}{\color{blue!20!black}\ttfamily}

\lstset{
    language=C,
    basicstyle=\sourcecodepro\EmptyMapping,
    escapechar=`,
    keywordstyle=\keywordstyle\EmptyMapping,
    identifierstyle=\sourcecodepro\EmptyMapping,
    numberstyle=\small\color{black!70},
    commentstyle=\color{red!60!black}\ttfamily\itshape,
    stringstyle=\color{blue!20!black}\ttfamily,
    ndkeywordstyle=\bfseries\color{blue!30!black},
    upquote=true,
}



\lstdefinestyle{medium}{
    basicstyle=\sourcecodepro\EmptyMapping\fontsize{12}{13}\selectfont,
    keywordstyle=\sourcecodepro\EmptyMapping\fontsize{12}{13}\selectfont\keywordstyle,
}

\lstdefinestyle{small}{
    basicstyle=\sourcecodepro\EmptyMapping\small,
    keywordstyle=\sourcecodepro\EmptyMapping\small\keywordstyle,
}

\lstdefinestyle{smaller}{
    basicstyle=\sourcecodepro\EmptyMapping\fontsize{11}{12}\selectfont,
    keywordstyle=\sourcecodepro\EmptyMapping\fontsize{11}{12}\selectfont\keywordstyle,
}

\lstdefinestyle{size105}{
    basicstyle=\sourcecodepro\EmptyMapping\fontsize{10.5}{11.5}\selectfont,
    keywordstyle=\sourcecodepro\EmptyMapping\fontsize{10.5}{11.5}\selectfont\keywordstyle,
}

\lstdefinestyle{size10}{
    basicstyle=\sourcecodepro\EmptyMapping\fontsize{10}{11}\selectfont,
    keywordstyle=\sourcecodepro\EmptyMapping\fontsize{10}{11}\selectfont\keywordstyle,
}


\lstdefinestyle{script}{
    basicstyle=\sourcecodepro\EmptyMapping\scriptsize,
    keywordstyle=\sourcecodepro\EmptyMapping\scriptsize\bfseries,
}




\begin{frame}{last time}
    \begin{itemize}
    \item some strategies for static analysis
        \begin{itemize}
        \item not running code like symbolic exec/fuzzing/etc.
        \item can use approximations of possible code paths
        \item track abstraction of value\ldots
            \begin{itemize}
            \item things pointer could point to
            \item is pointed-to-value free/not-free at this time
            \end{itemize}
        \end{itemize}
    \item difficulty of sandboxing
        \begin{itemize}
        \item what applications need to do
        \end{itemize}
    \item privilege separation
    \item sandboxing via system call filters
    \end{itemize}
\end{frame}

\begin{frame}{quiz}
\end{frame}

\begin{frame}[fragile]{quiz Q2}
\begin{Verbatim}[fontsize=\small]
struct CalendarEntry{ ...,
    related_document: RefCell<Document>,
    attendees: Vec<Rc<Attendee>>
}
struct Attendee { ..., calendar: RefCell<Calendar> }
\end{Verbatim}
    \begin{itemize}
    \item B: new calendar entry with different attendees + same document?
        \begin{itemize}
        \item RefCell doesn't do reference counting --- only one `master' reference
        \end{itemize}
    \item D: remove an attendee from entry's list + update attendee's calendar
        \begin{itemize}
        \item yes, provided no other writer for attendee's calendar
        \item calendar RefCell is internally mutable even though Attendee overall is not
        \end{itemize}
    \end{itemize}
\end{frame}

\begin{frame}{quiz Q3}
\begin{itemize}
    \item A: add entry to attendee with mutable reference
        \begin{itemize}
        \item RefCell --- internally mutable
        \item fine, provided not in the middle of code also trying to modify Calendar
        \end{itemize}
    \item D: multiple mutable references to different attendees, same calendar entry
        \begin{itemize}
        \item fine, provided we don't access the calendar entry from multiple references at once
        \item RefCell checks at runtime when we actually borrow reference
        \end{itemize}
\end{itemize}
\end{frame}

\begin{frame}[fragile]{quiz Q4}
\begin{Verbatim}[fontsize=\small]
struct CalendarEntry { ...,
    attendees: Vec<Rc<Attendee>>,
}
struct Calander { 
    entries: Vec<Rc<RefCell<CalendarEntry>>>,
}
struct Attendee { ...,
    calendar: RefCell<Calendar>,
}
\end{Verbatim}
    \begin{itemize}
    \item B: works, but I think less good design
    \item C: Weak ref in Calendar + Entry = everything gets freed
        \begin{itemize}
        \item only weak refs = free CalendarEntry
        \item nothing making sure CalendarEntry's are not freed, \\ unless 
        something else keeps list of all CalendarEntrys and all Attendees
        \end{itemize}
    \end{itemize}
\end{frame}

\begin{frame}[fragile]{quiz Q5}
    \begin{itemize}
    \item B: \verb|x + y == x * y|
        \begin{itemize}
        \item coverage information isn't going to help
        \item always says that all the code runs
        \end{itemize}
    \item C: \verb|x == y && x + y == 4|
        \begin{itemize}
        \item problem: if we have test case where x == y, 
            randomly changing it probably makes that false 
        \item so, challenge of finding x + y == 4 not helped by information about
            when we reach second half of \verb|&&|
        \end{itemize}
    \end{itemize}
\end{frame}

\begin{frame}{assignment Q+A}
\end{frame}

\section{Linux system call filtering}
\subsection{simple Linux system call filtering}
\newmintinline{C}{}
\begin{frame}{Linux system call filtering API}
    \begin{itemize}
    \item privilege seperation support: system call filtering
    \item simple API: \Cinline|prctl(SECCOMP_SET_MODE_STRICT, 0, 0)|
        \vspace{.5cm}
            \item ``The only system calls the calling thread is permitted to make are \texttt{read},
                \texttt{write}, \texttt{\_exit}, and \texttt{sigreturn}. Other system calls [kill
                the program].''
            \item read/write only work on \myemph{already open files}
    \vspace{.5cm}
    \item later: what if we want to be finer-grained?
    \end{itemize}
\end{frame}


\subsection{more fine-grained filtering?}
\begin{frame}{Linux system call filtering: detailed}
    \begin{itemize}
    \item Linux supports more fine-grained system call filtering
    \item using BPF (Berkeley Packet Filter) programming language
        \begin{itemize}
        \item compiled in the kernel to assembly to check system calls
        \end{itemize}
    \vspace{.5cm}
    \item can check system call argument values, but\ldots
        \begin{itemize}
        \item problems with pointer arguments
        \item too many system calls
        \end{itemize}
    \end{itemize}
\end{frame}

\begin{frame}[fragile,label=open]{Linux system call: open}
\begin{lstlisting}[language=C,style=small]
open("foo.txt", O_RDONLY);
\end{lstlisting}
\begin{itemize}
\item parameters:
    \begin{itemize}
    \item system call number: 2 (``open'')
    \item argument 1: 0x7fffe983 (address of string ``foo.txt'')
    \item argument 2: 0 (value of ``\texttt{O\_RDONLY}'')
    \end{itemize}
\item very problematic to filter using BPF interface
\vspace{.5cm}
\item can deal with using `ptrace' --- Linux debugging interface
    \begin{itemize}
    \item BPF can trigger something like a debugger breakpoint
    \item breakpoint wakes up monitor program (attached like debugger)
    \item `monitor' program can perform system call on program's behalf
    \end{itemize}
\end{itemize}
\end{frame}


\begin{frame}[fragile]{BPF filter example (1)}
\begin{itemize}
\item showing syntax for producing machine code from C macros / non-extended BPF
\end{itemize}
\begin{Verbatim}[fontsize=\fontsize{10}{11}]
// memory[offset of "nr"] --> accumulator
BPF_STMT(BPF_LD | BPF_W | BPF_ABS, (offsetof(struct seccomp_data, nr))),
// if (accumulator == SYS_write) PC += 1
BPF_STMT(BPF_JMP | BPF_JEQ, BPF_K, SYS_write, 1, 0),
// return "kill process"
BPF_STMT(BPF_RET | BPF_K, SECCOMP_RET_KILL_PROCESS),
// return "allow"
BPF_STMT(BPF_RET | BPF_K, SECCOMP_RET_ALLOW),
\end{Verbatim}
\end{frame}

\begin{frame}[fragile]{BPF filter example (2)}
\begin{Verbatim}[fontsize=\fontsize{10}{11}]
// memory[offset of "nr"] --> accumulator
BPF_STMT(BPF_LD | BPF_W | BPF_ABS, (offsetof(struct seccomp_data, nr))),
// if (accumulator == SYS_write) PC += 1 else PC += 0
BPF_STMT(BPF_JMP | BPF_JEQ, BPF_K, SYS_write, 1, 0),
// return "kill process"
BPF_STMT(BPF_RET | BPF_K, SECCOMP_RET_KILL_PROCESS),
// memory[offset of args[0]] --> accumulator
BPF_STMT(BPF_LD | BPF_W | BPF_ABS, (offsetof(struct seccomp_data, args[0]))),
// if (accumulator == 2) PC += 1 else PC += 0
BPF_STMT(BPF_JMP | BPF_JEQ, BPF_K, 2, 1, 0),
BPF_STMT(BPF_RET | BPF_K, SECCOMP_RET_KILL_PROCESS),
BPF_STMT(BPF_RET | BPF_K, SECCOMP_RET_ALLOW),
\end{Verbatim}
\end{frame}

\begin{frame}{other BPF operations}
    \begin{itemize}
    \item arithmetic (add, or, xor, \ldots)
    \item in eBPF (extended BPF): 10 additional registers
        \begin{itemize}
        \item not just accumulator
        \end{itemize}
    \end{itemize}
\end{frame}

\begin{frame}{running BPF fast/safely}
    \begin{itemize}
    \item idea: can verify in advance that\ldots
    \item there are no loops
    \item there are no out-of-bounds accesses
    \vspace{.5cm}
    \item convert to assembly function to run very fast
    \end{itemize}
\end{frame}


\subsection{aside: strace}
\begin{frame}[fragile]{strace hello\_world (1)}
\begin{lstlisting}[language=C,style=small]
#include <stdio.h>
int main() { puts("Hello, World!"); }
\end{lstlisting}
\hrule
when statically linked:
\begin{Verbatim}[fontsize=\fontsize{10}{11}\selectfont]
execve("./hello_world", ["./hello_world"], 0x7ffeb4127f70 /* 28 vars */)
                                        = 0
brk(NULL)                               = 0x22f8000
brk(0x22f91c0)                          = 0x22f91c0
arch_prctl(ARCH_SET_FS, 0x22f8880)      = 0
uname({sysname="Linux", nodename="reiss-t3620", ...}) = 0
readlink("/proc/self/exe", "/u/cr4bd/spring2023/cs3130/slide"..., 4096)
                                        = 57
brk(0x231a1c0)                          = 0x231a1c0
brk(0x231b000)                          = 0x231b000
access("/etc/ld.so.nohwcap", F_OK)      = -1 ENOENT (No such file or
                                                     directory)
fstat(1, {st_mode=S_IFCHR|0620, st_rdev=makedev(136, 4), ...}) = 0
write(1, "Hello, World!\n", 14)         = 14
exit_group(0)                           = ?
+++ exited with 0 +++
\end{Verbatim}
\end{frame}

\begin{frame}{aside: what are those syscalls?}
\begin{itemize}
\item execve: run program
\item brk: allocate heap space
\item arch\_prctl(ARCH\_SET\_FS, ...): thread local storage pointer
    \begin{itemize}
    \item may make more sense when we cover concurrency/parallelism later
    \end{itemize}
\item uname: get system information
\item readlink of /proc/self/exe: get name of this program
\item access: can we access this file [in this case, a config file]?
\item fstat: get information about open file
\item exit\_group: variant of exit
\end{itemize}
\end{frame}

\begin{frame}[fragile]{only after starting? (1)}
\begin{itemize}
\item okay, but that's only after starting up, right\ldots?
\item surely simpler if we limit system calls after startup
\item yes, but\ldots
\end{itemize}
\end{frame}

\begin{frame}[fragile]{only after starting? (2)}
\begin{lstlisting}[language=C,style=smaller]
#include <stdio.h>
int main() {
    FILE *fh = fopen("output.txt", "w");
    fprintf(fh, "example");
    fclose(fh);
}
\end{lstlisting}
\hrulefill
\begin{Verbatim}[fontsize=\fontsize{9}{10}]
$ strace ...
... [startup stuff, not shown] ...
openat(AT_FDCWD, "output.txt", O_WRONLY|O_CREAT|O_TRUNC, 0666) = 3
newfstatat(3, "", {st_mode=S_IFREG|0664, st_size=0, ...}, AT_EMPTY_PATH) = 0
write(3, "example", 7)                  = 7
close(3)                                = 0
\end{Verbatim}
\end{frame}

\begin{frame}[fragile]{only after starting? (2)}
\begin{lstlisting}[language=C,style=script]
#include <curl/curl.h>
int main() {
    CURL *handle = curl_easy_init();
    curl_easy_setopt(handle, CURLOPT_URL, "https://www.cs.virginia.edu/~cr4bd/test.txt");
    curl_easy_perform(handle);
    ...
}
\end{lstlisting}
\vspace{-.5cm}
\hrulefill
\begin{Verbatim}[fontsize=\fontsize{8}{9}]
$ strace ...
... [startup stuff, not shown] ...
futex(0x73f0bd640ba4, FUTEX_WAKE_PRIVATE, 2147483647) = 0                                               
...
openat(AT_FDCWD, "/usr/lib/ssl/openssl.cnf", O_RDONLY) = 3
...
sysinfo({...}) = 0
...
socket(AF_INET6, SOCK_DGRAM, IPPROTO_IP) = 3
close(3)                                = 0                                                             
socketpair(AF_UNIX, SOCK_STREAM, 0, [3, 4]) = 0                                                         
fcntl(3, F_GETFL)                       = 0x2 (flags O_RDWR)                                            
fcntl(3, F_SETFL, O_RDWR|O_NONBLOCK)    = 0                                                             
...
rt_sigaction(SIGPIPE, NULL, {sa_handler=SIG_DFL, sa_mask=[], sa_flags=0}, 8) = 0                        
...
socket(AF_INET, SOCK_STREAM, IPPROTO_TCP) = 5
setsockopt(5, SOL_TCP, TCP_NODELAY, [1], 4) = 0
...
getrandom("\xd6\x8c\xc3\x42\x07\x92"..., 48, 0) = 48 
...
\end{Verbatim}
\end{frame}



\subsection{filtering is hard}
\input{../sandbox/linux-syscall-open-filter-hard}

\section{libseccomp}
\begin{frame}[fragile]{libseccomp}
    \begin{itemize}
    \item wrapper for writing BPF programs
    \item specify list of rules re: system call identifiers/arguments
    \item it generates BPF program with LDs, JMPs, etc.
    \end{itemize}
\begin{Verbatim}[fontsize=\small]
#define CHECK(x) if (!(x)) handle_error();
...
scmp_filter_ctx filter = seccomp_init(SCMP_ACT_KILL_PROCESS);
CHECK(seccomp_rule_add(filter, SCMP_ACT_ALLOW, SCMP_SYS(read), 0) == 0);
CHECK(seccomp_rule_add(filter, SCMP_ACT_ALLOW, SCMP_SYS(write), 0) == 0);
CHECK(seccomp_load(filter) == 0);
\end{Verbatim}
\end{frame}



\section{OpenSSH architecture}
\begin{frame}{OpenSSH privilege seperation}
    \begin{itemize}
    \item OpenSSH uses privilege seperation for its SSH server
    \item what runs on the lab machines when you log into them
    \vspace{.5cm}
    \item separate network processing code from authentication code
    \item seperate process per connection --- users don't share
    \vspace{.5cm}
    \item developed before system call filtering was widely available
        \begin{itemize}
        \item uses separate user + chroot (we'll talk later) to isolate
        \end{itemize}
    \end{itemize}
\end{frame}

\begin{frame}{OpenSSH privsep protocol}
    \begin{itemize}
    \item sandboxed process tells ``monitor'' to:
        \vspace{.25cm}
    \item perform \myemph{cryptographic operations}
        \begin{itemize}
        \item long-term keys never in sandboxed process
        \item commands to ask for cryptographic messages they need
        \end{itemize}
    \item ask to switch to user --- if given user password, etc.
        \begin{itemize}
            \item \myemph{monitor process verifies} login information
        \end{itemize}
    \item after authentication: new process running as logged-in user 
        \begin{itemize}
            \item (normally) no issues with special privileges
        \end{itemize}
    \end{itemize}
\end{frame}


\begin{frame}{privilege seperation overall}
    \begin{itemize}
    \item large application changes
        \begin{itemize}
        \item OpenSSH: 3k lines of code for communication/etc. added
        \item OpenSSH: 2\% of existing code (950 of 44k lines) changed
        \item (but most changes simple)
        \end{itemize}
    \item lots of application knowledge
        \begin{itemize}
        \item what is a meaningful separation of `privileged' and `unprivileged'?
        \end{itemize}
    \item better application design anyways?
    \end{itemize}
\end{frame}



\section{Chrome architecture}
\usetikzlibrary{positioning,shapes.callouts}

\begin{frame}{Chrome architecture}
    \includegraphics[height=0.8\textheight]{../sandbox/chrome-arch}
\end{frame}

\begin{frame}{talking to the sandbox}
    \begin{itemize}
    \item browser kernel sends commands to sandbox
    \item sandbox sends commands to browser kernel
    \item idea: commands only allow necessary things
    \end{itemize}
\end{frame}

\begin{frame}{original Chrome sandbox interface}
    \begin{itemize}
        \item sandbox to browser ``kernel''
            \begin{itemize}
                \item show this image on screen
                \begin{itemize}
                    \item (using shared memory for speed)
                \end{itemize}
            \item \myemph<2-3>{make request\tikzmark{request} for this URL}
            \item \myemph<4>{download\tikzmark{download} files to local FS}
            \item \myemph<5>{upload\tikzmark{upload} user requested files}
            \end{itemize}
        \item browser ``kernel'' to sandbox
            \begin{itemize}
                \item send user input
            \end{itemize}
    \end{itemize}
    \begin{tikzpicture}[overlay,remember picture]
        \coordinate (middle) at ([yshift=-1cm]current page.center);
        \begin{visibleenv}<2>
            \node[my callout=request,anchor=center,align=center]  at (middle) {
                needs filtering --- at least no \texttt{file:} (local file) URLs
            };
        \end{visibleenv}
        \begin{visibleenv}<3>
            \node[my callout=request,anchor=center,align=center] at (middle) {
                can still read any website! \\
                still sends normal cookies!
            };
        \end{visibleenv}
        \begin{visibleenv}<4>
            \node[my callout=download,anchor=center,align=center] at (middle) {
                files go to download directory only \\
                can't choose arbitrary filenames
            };
        \end{visibleenv}
        \begin{visibleenv}<5>
            \node[my callout=upload,anchor=center,align=center] at ([yshift=-1cm]middle) {
                browser kernel displays file choser \\
                only permits files selected by user
            };
        \end{visibleenv}
    \end{tikzpicture}
\end{frame}



% FIXME: Chrome Site Isolation
\subsection{Site Isolation}
% https://www.usenix.org/conference/usenixsecurity19/presentation/reis

\begin{frame}{Site Isolation}
\begin{itemize}
\item Chrome since version 67 (desktop)/77 (Mobile) has process per site
\item site $\approx$ registered domain name (example.com, example.co.uk, etc.)
    \begin{itemize}
    \item slightly different than same origin policy
    \end{itemize}
\vspace{.5cm}
\item complicated to implement:
    \begin{itemize}
    \item single web page can embed content from multiple other sites
        \begin{itemize}
        \item and those other sites can embed content from yet more sites
        \end{itemize}
    \item web page can call services on other websites with ``permission'' of other website
    \item clicking link may or may not requiring switching to new process
    \end{itemize}
\vspace{.5cm}
\item same separation being prototyped in recent Firefox builds
\end{itemize}
\end{frame}


\subsection{exercise: priv sep for}
\begin{frame}{privilege separation for}
    \begin{itemize}
    \item let's say we wanted to add sandboxing/privilege separation to an (standalone) mail program
    \vspace{.5cm}
    \item exercise 1: where would be concerned about security problems?
    \item exercise 2: propose a way of dividing up the program
    \end{itemize}
\end{frame}


\section{normal application confinement?}
\begin{frame}{application confinement}
    \begin{itemize}
    \item confining whole browsers was hard
        \begin{itemize}
            \item we trust them to do a lot of things --- e.g. write arbitrary files
        \end{itemize}
    \item but maybe we can do this for simpler applications?
    \item idea 1: applications send system calls to OS 
        \begin{itemize}
        \item \myemph{limit syscalls} like we limited browser kernel commands
        \item constructing command language ``in reverse''
        \end{itemize}
    \end{itemize}
\end{frame}




\subsection{applied to VLC?}
\begin{frame}{filtering system calls?}
    \begin{itemize}
    \item example: video player VLC playing a local file on my laptop
    \item uses \myemph{73 unique kinds of system calls}
    \item opens many files that \myemph{are not the video file}
        \begin{itemize}
        \item libraries
        \item fonts
        \item configuration files
        \item translations of messages
        \end{itemize}
    \vspace{.5cm}
    \item can I limit the files my video player can read?
    \item how do I come up with a useful filter?
    \end{itemize}
\end{frame}


\subsection{shared services?}
\begin{frame}{shared services?}
    \begin{itemize}
    \item often programs do operations by talking to ``server'' program
        \begin{itemize}
        \item example: GUI management on Linux (X11 or Wayland), OS X (WindowServer)
        \item example: mixing sound from multiple applications
        \item \ldots
        \end{itemize}
    \item whole extra set of calls to sanitize
        \begin{itemize}
        \item when to allow ``get keyboard input'' for GUI
        \item when to allow ``get microphone input'' for sound manager
        \item making sure one isn't manipulating wrong program's windows?
        \end{itemize}
    \item also, server programs might have security problems
        \begin{itemize}
        \item common ``sandbox escape''
        \end{itemize}
    \end{itemize}
\end{frame}



\subsection{pro/con shared services}
\begin{frame}{exercise: app confinement options}
    \begin{itemize}
    \item sandboxed applications want to access display server
    \item which option seems best for security/performance?
    \begin{itemize}
    \item A. proxy for protocol display server supports natively that filters display calls
    \item B. custom protocol that sends bitmaps + receives inputs, plus copy of display server runs with application
    \item C. divide application into UI and non-UI part, sandbox just the non-UI part
    \item D. have application take over screen when running, give its own display server
    \end{itemize}
    \end{itemize}
\end{frame}



\subsection{versus capability-type approach}
\begin{frame}{changing what programs can name}
    \begin{itemize}
    \item seccomp, SELinux: program tries to access X, checks if allowed
    \vspace{.5cm}
    \item alternate idea: changing what Xs program can name
    \end{itemize}
\end{frame}


\subsection{chroot}
\begin{frame}{Unix filesystems and mounting}
    \begin{itemize}
    \item my Linux desktop has two disks:
        \begin{itemize}
        \item \texttt{/} --- an SSD
        \item \texttt{/mnt/extradisk} --- a hard drive
        \end{itemize}
    \item hard drive appears as \textit{subdirectory} of SSD
    \item subdirectory called a \textit{mount point}
    \end{itemize}
\end{frame}

\begin{frame}[fragile,label=perProcessRoot]{per-process root}
    \begin{itemize}
    \item on Unix: each process tracks its own root directory (/)
    \item can be changed with chroot() system call
        \begin{itemize}
        \item command-line tool to access: \texttt{chroot}
        \end{itemize}
    \vspace{.5cm}
    \item usage: can isolate program from other files on system
        \begin{itemize}
        \item example: limit what public file server can access?
        \end{itemize}
    \end{itemize}
\end{frame}

\begin{frame}[fragile,label=lsChrootExample]{chroot ls}
\begin{lstlisting}[language={},style=smaller]
# mkdir /tmp/example
# cp /bin/ls /tmp/example/ls
# chroot /tmp/example /ls
chroot: failed to run command ‘/ls’: No such file or directory
# cp -r /lib64 /tmp/example/lib64
# mkdir -p /tmp/example/lib
# cp -r /lib/x86_64-linux-gnu /tmp/example/lib/x86_64-linux-gnu
# chroot /tmp/example /ls
/ls: error while loading shared libraries: libpcre2-8.so.0: cannot open shared object file: No such file or directory
# cp /usr/lib/x86_64-linux-gnu/libpcre2-8* /tmp/example/lib/x86_64-linux-gnu
# chroot /tmp/example /ls /
lib  lib64  ls
# chroot /tmp/example /ls /..
lib  lib64  ls
# 
\end{lstlisting}
\end{frame}

\begin{frame}{chroot escapes}
    \begin{itemize}
    \item chroot prevents accessing files outside the new \texttt{/}
    \item but root (system adminstrator) user in chroot can access disks, etc.
    \vspace{.5cm}
    \item typical usage: combine chroot with extra user
    \end{itemize}
\end{frame}

\begin{frame}{chroot impracticality}
    \begin{itemize}
    \item some things make chroot impractical in general:
    \vspace{.5cm}
    \item seems like one needs extra copies of most of the system
    \item hard to communicate between separate roots
    \item requires administrator permissions to configure
        \begin{itemize}
        \item dangerous to let normal users configure b/c they could confuse priviliged (set-user-ID) programs like \texttt{sudo}
        \end{itemize}
    \end{itemize}
\end{frame}


\subsubsection{exercise}
\begin{frame}{exercise}
    \begin{itemize}
    \item what scenarios does chroot make most/least sense for?
    \begin{itemize}
        \item A. the rendering part of web browser
        \item B. a web server
        \item C. a media player
        \item D. a network time server (for other machines to set their clocks)
    \end{itemize}
    \end{itemize}
\end{frame}


\subsection{Linux namespaces}
\begin{frame}{Linux namespaces (1)}
    \begin{itemize}
    \item Linux: alternate sandboxing features
    \item ``namespaces'' for other resources
    \item chroot: each process has own idea of root directory
        \begin{itemize}
        \item change to OS: look up root directory in process, not global variable
        \end{itemize}
    \item can apply this to other resources:
        \begin{itemize}
        \item what filesystems (disks) are available
        \item what network devices are available
        \item what user identifier numbers are
        \item \ldots
        \end{itemize}
    \end{itemize}
\end{frame}

\begin{frame}{Linux namespaces (2)}
    \begin{itemize}
    \item user namespace:
    \vspace{.5cm}
    \item can run programs with new view of users:
    \vspace{.5cm}
    \item inside namespace: running as root
    \item outside namespace: root translated to innocent user ID
    \item allows running programs that expect different users
        \begin{itemize}
        \item \ldots without changes, but without giving special permissions
        \end{itemize}
    \vspace{.5cm}
    \item mechanism: reassign user ID numbers in kernel
    \end{itemize}
\end{frame}

\begin{frame}[fragile,label=LinuxCloneUnshare]{aside: Linux clone(), unshare() syscalls}
\begin{itemize}
\item Linux clone system call: start new process (or thread)
\item flags to specify environment of new process
\item these flags can include ``make a new namespace of a type''
\end{itemize}
\begin{lstlisting}[style=smaller,language=C++]
int id = clone(start_function, ..., CLONE_NEWUSER | other-flags);
\end{lstlisting}
\begin{itemize}
\item above option: new user namespace for new process
\vspace{.5cm}
\item alternative: for changing current process's namespace:
\end{itemize}
\begin{lstlisting}[style=smaller,language=C++]
unshare(CLONE_NEWUSER);
\end{lstlisting}
\end{frame}

\begin{frame}{user namespaces API}
\begin{itemize}
\item Linux: users identified by numerical \textit{user IDs} (UIDs)
\vspace{.5cm}
\item with user namespaces:
\item control file \texttt{/proc/PROCESS-ID/UID\_MAP} contains lines like:
    \begin{itemize}
    \item \texttt{0 1000 2} --- UID 0--1 maps to UID 1000--1001
    \item \texttt{1000 2000 100} --- UID 1000-1100 maps to UID 2000--2100
    \end{itemize}
\item can write to that file to reconfigure (if enough permissions)
\end{itemize}
\end{frame}

\begin{frame}{Linux namesapces (3)}
    \begin{itemize}
    \item mount namespaces:
        \begin{itemize}
        \item Unix: mounting disk = making the contents of the disk available as directories+files
        \end{itemize}
    \vspace{.5cm}
    \item different idea of what filesystems are available
    \item can be setup with \textit{bind mounts} to ``real FS''
        \begin{itemize}
        \item but otherwise: no access to directories outside mount namespace
        \item normally requires root --- but special case with user namespaces
        \end{itemize}
    \end{itemize}
\end{frame}

\begin{frame}[fragile,label=mountNSCmdLine]{mount namespaces API}
from command line:
\begin{lstlisting}[language={},style=smaller]
    # runs shell (/bin/sh) in new mount namesapce
shell1$ unshare --mount /bin/sh

    # setup directories in /tmp/workdir and make them aliases of things on normal FS 
    # these aliases will only exist for processes in mount namespace
shell2$ mkdir -p /tmp/workdir/bin
shell2$ mkdir -p /tmp/workdir/lib
shell2$ mkdir -p /tmp/workdir/usr
shell2$ mkdir -p /tmp/workdir/current
shell2$ mount -o bind,ro /bin /tmp/workdir/bin
shell2$ mount -o bind,ro /lib /tmp/workdir/lib
shell2$ mount -o bind,ro /usr /tmp/workdir/usr
shell2$ mount -o bind /home/someuser /tmp/workdir/current

    # start new shell with the root directory being /tmp/workdir
shell2$ chroot /tmp/workdir /bin/sh
shell3$ cd /
shell3$ /bin/ls
bin     current     lib     usr
\end{lstlisting}
\end{frame}

\begin{frame}{Linux namespaces (3)}
    \begin{itemize}
    \item user namespace and mount namespace together:
    \vspace{.5cm}
    \item run program in new user namespace
    \item map regular root (in namespace) to regular user
        \begin{itemize}
        \item ``opts out'' of programs like sudo
        \end{itemize}
    \item move to new mount namespace
    \item setup bind mounts + chroot
        \begin{itemize}
        \item special case: allowed because root in user namespce
        \item can't get ``real'' root (administrator) privileges ever
        \end{itemize}
    \item run program with subset of available files
    \end{itemize}
\end{frame}

\begin{frame}{Linux namespaces (4)}
    \begin{itemize}
    \item other resources with namespaces
        \begin{itemize}
        \item network --- common usage: virtual network device for set processes
        \item hostname (``UTS'')
        \item process identifiers
        \item control groups (resource limits for memory, CPU usage, disk I/O, etc.)
        \end{itemize}
    \end{itemize}
\end{frame}

\begin{frame}{Linux control groups}
    \begin{itemize}
    \item control groups --- tied to namespaces
    \item primarily: CPU/memory/IO performance restrictions
        \begin{itemize}
        \item primarily intended for `friendly sharing' (containers, etc.)
        \item important for preventing denial-of-service/etc.
        \item not as big a security conern as file/user/etc. access
        \end{itemize}
    \item also mechanism for adding IO device restrictions
    \item also mechanism to start/stop a bunch of processes together
    \end{itemize}
\end{frame}


% FIXME: Chrome sandboxing failing:
    % https://theori.io/research/escaping-chrome-sandbox/
        % Binder (Chrome internal IPC library)
    % % https://googleprojectzero.blogspot.com/2020/04/you-wont-believe-what-this-one-line.html
    % https://bugs.chromium.org/p/project-zero/issues/detail?id=1991
    % https://bugs.chromium.org/p/project-zero/issues/detail?id=1985

% FIXME: SELinux sandbox escape:
    % https://www.openwall.com/lists/oss-security/2016/09/25/1

\subsection{Linux programs that attempt confinement}
\begin{frame}{Linux sandboxing programs, generally}
    \begin{itemize}
    \item docker, lxc, lxd, containerd
        \begin{itemize}
        \item use namespaces to create ``container'' with own copy of OS libraries, services
        \item but containers share OS `kernel' and potentially files with host unlike VM
        \item (might also have options to use other ways of getting this functionality --- VM's, etc.)
        \end{itemize}
    \item bubblewrap, firejail
        \begin{itemize}
        \item use Linux namespace tools + ``bind mounts'' to give programs only subset of files, etc.
        \item firejail has option of running a ``proxy'' windowing system server
        \end{itemize}
    \item SELinux's sandbox
        \begin{itemize}
        \item uses Security Enhanced Linux's mandatory access controls instead of Linux namespaces
        \item includes option for ``proxy'' for windoing system server
        \end{itemize}
    \end{itemize}
\end{frame}


\subsection{containers}
\begin{frame}{containers}
    \begin{itemize}
    \item Linux's seccomp + namespaces + SELinux commonly used to implement containers
        \begin{itemize}
        \item (plus cgroups (control groups) for performance isolation)
        \end{itemize}
    \vspace{.5cm}
    \item usual goal: looks like virtual machine, but much lower overhead
    \item examples: Docker, Kubernetes
        \begin{itemize}
        \item (note: these may also support other ways of creating `lightweight VMs')
        \end{itemize}
    \end{itemize}
\end{frame}





\subsubsection{runC bug}
\begin{frame}{runc bug}
    \begin{itemize}
    \item 2019 bug in Docker, other container implementations (CVE-2019-5736)
        \begin{itemize}
        \item blog post for vulnerability finders: \\\scriptsize \url{https://blog.dragonsector.pl/2019/02/cve-2019-5736-escape-from-docker-and.html}
        \end{itemize}
    \vspace{.5cm}
    \item bug setup:
        \begin{itemize}
        \item user starts malicious container X
        \item user tells docker to start a new command in malicious container X
        \item \myemph<2>{malicious container X hijacks the ``new command'' starting program}
        \item hijacked program used to access stuff outside container
        \end{itemize}
    \item part of problem: Docker and others weren't using user namespaces at the time
        \begin{itemize}
        \item compatability problems
        \end{itemize}
    \end{itemize}
\end{frame}

\begin{frame}{setup: /proc/PID}
    \begin{itemize}
    \item Linux provides /proc directory to access info about programs
    \item used for implementing process list utils, debugging
        \begin{itemize}
        \item needed to make a functional container
        \end{itemize}
    \item subdirectory for each process in current container
        \begin{itemize}
        \item process ID PID has /proc/PID subdirectory
        \item /proc/self is alias for current process's subdirectory
        \end{itemize}
    \vspace{.5cm}
    \item included is /proc/PID/exe file --- alias for executable file
    \end{itemize}
\end{frame}

\begin{frame}{running a command in existing container}
    \begin{itemize}
    \item to run command X in existing container:
    \vspace{.5cm}
    \item step 1: switch current process to that container
    \item<2-> \myemph{code in container can access /proc here?}
    \item<2-> \myemph{including overwriting /proc/self/exe!}
        \begin{itemize}
        \item which is a program run as root!
        \end{itemize}
    \vspace{.25cm}
    \item step 2: execute command X
    \end{itemize}
\end{frame}


\begin{frame}{partial fix}
    \begin{itemize}
    \item can disable access to /proc/PID/exe (and related things)
    \item system call: \texttt{prctl(PR\_SET\_DUMPABLE, 0)}
    \item but\ldots the run-in-container tool did this for a while
    \vspace{.5cm}
    \item<2-> problem: this gets reset on executing a new program
    \item<2-> and attacker could make the new program be /proc/PID/exe
        \begin{itemize}
        \item one mechanism: symbolic links (file aliases)
        \end{itemize}
    \item<2-> but change dynamic linking setup to run attacker code
    \item<2-> \ldots which accesses /proc/self/exe
    \end{itemize}
\end{frame}

\begin{frame}{full fix}
    \begin{itemize}
    \item make single-use copy of start-in-container tool each time command run
        \begin{itemize}
        \item in-memory file
        \end{itemize}
    \item \ldots so modifying it doesn't change anything
        \begin{itemize}
        \item (but it's also protected from modification)
        \end{itemize}
    \vspace{.5cm}
    \item other solutions:
        \begin{itemize}
        \item make executable non-writable (e.g. SELinux, don't run container as root)
        \end{itemize}
    \end{itemize}
\end{frame}


\subsection{SELinux sandbox escape}
\begin{frame}{SELinux escape}
\includegraphics[height=0.9\textheight]{../sandbox/selinux-escape-email}
\end{frame}

\section{the Android sandbox}
\begin{frame}{Android sandbox}
    \begin{itemize}
    \item Android --- Linux based OS for phones/tablets
    \vspace{.5cm}
    \item \url{https://source.android.com/security/app-sandbox}
    \item current version: SELinux + seccomp (system call filter)
    \end{itemize}
\end{frame}


% FIXME: Android sandbox

\subsection{OSX sandboxing}

\begin{frame}{OS X sandboxing}
    \begin{itemize}
    \item OS X (tries to) implement system call filtering
    \item main challenge: what about files?
        \begin{itemize}
        \item user can open a file anywhere --- we expect that to work
        \end{itemize}
    \item<2> OS X solution: OS service displays file-open dialog
        \begin{itemize}
        \item OS knows user really choose a file
        \end{itemize}
    \item<2> application can ask to remember file was chosen previously
    \item<2> not chosen/remembered --- can't access
        \begin{itemize}
        \item requires changes to how applications open files
        \end{itemize}
    \end{itemize}
\end{frame}


\subsection{Qubes}

\begin{frame}{another sandboxing OS: Qubes}
    \begin{itemize}
    \item Qubes: heavily sandboxed OS
    \item runs \myemph{seperate VMs} instead of filtering syscalls
    \item UI that clearly shows what VM each window is from
    \vspace{.5cm}
    \item advantage: easier to gaurentee isolation
        \begin{itemize}
        \item many, many more bugs in system call filtering than VMs
        \end{itemize}
    \item disadvantage: harder to share between VMs
    \item disadvantage: much more runtime overhead
    \end{itemize}
\end{frame}

\begin{frame}{Qubes screenshot}
    \includegraphics[width=\textwidth]{../sandbox/qubes-desktop1}
\end{frame}




\section{Which sandboxing?}
\begin{frame}{which sandboxing?}
    \begin{itemize}
    \item which whole-application sandboxing technique seems better for 
        \begin{itemize}
        \item security, performance, usability, handling unchanged applications
        \end{itemize}
    \item (full answer: could mix techniques + probably depends on details of app)
    \vspace{.5cm}
    \item A. chroot + system call filtering
    \item B. chroot + mount and user namespaces
    \item C. virtual machine dedicated to application
    \item D. SELinux-like mandatory access control
    \end{itemize}
\end{frame}


\section{sandboxing without OS support}
\begin{frame}{sandboxing without OS support}
    \begin{itemize}
    \item so far: relying on OS features for sandboxing
    \item good reasons:
        \begin{itemize}
        \item primarily want to filter system calls
        \item hardware-assisted, strong protection
        \end{itemize}
    \vspace{.5cm}
    \item but problems with relying on OS:
        \begin{itemize}
        \item sending information in/out of sandbox relatively slow
        \item requires heavily OS-specific code
        \end{itemize}
    \end{itemize}
\end{frame}

\begin{frame}{sandboxing without OS ideas}
    \begin{itemize}
    \item `dynamic' language virtual machine, like Java VM, .Net CLR
        \begin{itemize}
        \item hard to use with code intended to compile to native machine code
        \end{itemize}
    \vspace{.5cm}
    \item virtual machine targetted for C/C++-like code, like WebAssembly
    \vspace{.5cm}
    \item assembly-to-assembly conversion
        \begin{itemize}
        \item example: Wahbe, Lucco, Anderson, and Graham, ``Efficient Software-Based Fault Isolation'' (1993)
        \item example: Ford and Cox, ``Vx32: Lightweight User-level Sandboxing on the x86'' (2008)
        \end{itemize}
    \end{itemize}
\end{frame}


\subsection{Wasm}
\begin{frame}{WebAssembly}
    \begin{itemize}
    \item WebAssembly: language virtual machine specification intended\ldots
        \begin{itemize}
        \item similar idea to Java VM
        \end{itemize}
    \vspace{.5cm}
    \item to be compiled to from C/C++
        \begin{itemize}
        \item support by Clang/LLVM
        \end{itemize}
    \item to be easy to just-in-time compile to native machine code
    \item to be run in web browsers (fast web apps)
    \end{itemize}
\end{frame}

\begin{frame}{WebAssembly memory management}
    \begin{itemize}
    \item WebAssembly `modules' have a single ``linear memory''
    \item starts at index 0, goes to some maximum
    \item load/store instructions take index into current memory
    \vspace{.5cm}
    \item observation 1: close to memory model ``normal'' C/C++ code expects
    \vspace{.5cm}
    \item observation 2: only goal is to prevent sandbox (WebAssembly) code from interfering with outside code
    \item \ldots so no need to check array bound or similar
    \vspace{.5cm}
    \item observation 3: no need to worry about garbage collection
    \end{itemize}
\end{frame}

\begin{frame}{WebAssembly validation}
    \begin{itemize}
    \item WebAssembly virtual machine code designed to be \textit{validated} before running
    \vspace{.5cm}
    \item allows for efficient interpreters or conversion to assembly
        \begin{itemize}
        \item validation ensures that you can safely skip certain type checks, etc.
        \end{itemize}
    \item language specification very explicit about what needs to be checked at runtime
    \end{itemize}
\end{frame}

\begin{frame}{example WebAssembly validation}
    \begin{itemize}
    \item check that instructions have right number of operands available
        \begin{itemize}
        \item WebAssembly instructions use stack (compile \texttt{2 + 2} into \texttt{2 2 +})
        \end{itemize}
    \item check operands that can be checked (constants)
    \item check the calls go to only functions listed in table
        \begin{itemize}
        \item should make it easier to do just-in-time compilation to machine code?
        \end{itemize}
    \item check the branches go to only locations listed in table, and only within one function
        \begin{itemize}
        \item should make it easier to do just-in-time compilation to machine code?
        \end{itemize}
    \end{itemize}
\end{frame}

\begin{frame}{example WebAssembly instruction specification}
\includegraphics[height=0.8\textheight]{../sandbox/wasm-return-spec}
\end{frame}

\begin{frame}{WebAssembly as sandboxing}
    \begin{itemize}
    \item can compile existing C/C++ library using WebAssembly\ldots
    \item then call using language virtual machine
    \end{itemize}
\end{frame}



\section{sandboxing API: RLBox}
\begin{frame}{RLBox}
    \begin{itemize}
    \item saw interfaces for using sandboxes from user perspective?
    \item what about for privilege separation?
        \begin{itemize}
        \item recall: like Chrome separate renderer process idea
        \item need to navigate OS sandboxing API + create interface for sandboxed part?
        \end{itemize}
    \vspace{.5cm}
    \item some reusable tools have appeared for this (but no clear winner)
    \item one example: RLBox (published in Usenix Security 2020)
        \begin{itemize}
        \item Shravan Narayan and Craig Disselkoen, UC San Diego; Tal Garfinkel, Stanford University; Nathan Froyd and Eric Rahm, Mozilla; Sorin Lerner, UC San Diego; Hovav Shacham, UT Austin; Deian Stefan, UC San Diego
        \end{itemize}
    \end{itemize}
\end{frame}

\begin{frame}[fragile,label=rlboxusage]{RLBox usage}
\begin{itemize}
\item part of example from author's presentation:
    \begin{itemize}
    \item goal: invoke JPEG parser in sandbox
    \end{itemize}
\end{itemize}
\begin{lstlisting}[language=C++,style=script]
autosandbox = rlbox::create_sandbox<wasm>();
tainted<jpeg_decompress_struct*> p_jpeg_img = sandbox.malloc_in_sandbox<jpeg_decompress_struct>();
tainted<jpeg_source_mgr*> p_jpeg_input_source_mgr = sandbox.malloc_in_sandbox<jpeg_source_mgr>();
sandbox.invoke(jpeg_create_decompress, p_jpeg_img);
p_jpeg_img->src = p_jpeg_input_source_mgr;
p_jpeg_img->src->fill_input_buffer = ...;
sandbox.invoke(jpeg_read_header,p_jpeg_img/*...*/);
\end{lstlisting}
\begin{itemize}
\item tool handles running `jpeg\_create\_decompress', `jpeg\_read\_header' in sandbox
\item values shared with sandbox marked as ``tainted''
    \begin{itemize}
    \item C++ (template) class
    \end{itemize}
\item this example: using WebAssembly-based sandbox
\item used in firefox
\end{itemize}
\end{frame}


\section{application permissions}
\usetikzlibrary{calc}
\begin{frame}{some Android prompts}
\includegraphics[width=\textwidth]{../sandbox/android-perm-screens}
\imagecredit{from Clark et al, ``No Time At All: Opportunity Cost of Android Permissions'' (HotWireless'16)}
\end{frame}


\subsection{UI problems}

\begin{frame}<1>[label=appPermUI]{UI problems with application permissions}
    \begin{itemize}
    \item \myemph<2>{do applications request sensible permissions?}
    \item \myemph<3>{do users pay attention to permission requests?}
    \item \myemph<3>{do users understand what permissions mean?}
    \item are permissions fine-grained enough?
    \item are permissions coarse-grained enough?
    \end{itemize}
\end{frame}


\subsection{do request right permissions?}
\againframe<2>{appPermUI}
\begin{frame}{right permissions?}
    \begin{itemize}
    \item Felt, Chin, Hanna, Song and Wagner, ``Android Permissions Demystified'' (CCS 2011)
    \item used static analysis to compare requested permissions to what applications did
        \begin{itemize}
        \item at the time: permissions requested at installation
        \end{itemize}
    \item sample of 900 applications
    \item estimate approx 200 over-privileged
        \begin{itemize}
        \item (estimate because using false positive rate from manual checking)
        \end{itemize}
    \end{itemize}
\end{frame}

\begin{frame}{why extra permissions?}
    \begin{itemize}
    \item selected from Felt et al's analysis:
    \item developers confused similar permissions
        \begin{itemize}
        \item \texttt{ACCESS\_NETWORK\_STATE} versus \texttt{ACCESS\_WIFI\_STATE}
        \end{itemize}
    \item developers thought permissions were needed for delegated tasks
        \begin{itemize}
        \item \texttt{CALL\_PHONE} not needed to invoke phone app
        \item \texttt{INSTALL\_APPLICATION} not needed to open app store install dialog
        \end{itemize}
    \item developers thought permissions needed for all methods of class
        \begin{itemize}
        \item \texttt{WRITE\_SETTINGS} when using (no-permission) read-settings operations
        \end{itemize}
    \item copy-and-paste
    \end{itemize}
\end{frame}


\subsection{do users understand permissions?}
\againframe<3>{appPermUI}
\begin{frame}{a user study (2012)}
    \begin{itemize}
    \item Felt, Ha, Egelman, Haney, Chin, Wagner, ``Android Permissions: User Attention, Comprehension, and Behavior''
    \item performed lab study; task: find + install coupon app
    \item at the time: Android prompted for permissions on installation
    \vspace{.5cm}
    \item<2-> 17\% looked at app permissions detail
    \item<2-> 42\% aware of permissions
    \item<2-> 42\% unaware of permissions
    \vspace{.5cm}
    \item<2-> versus: 88\% read reviews 
    \end{itemize}
\end{frame}

\begin{frame}{a user survey (2012)}
    \begin{itemize}
    \item same paper did survey about what permissions meant
    \item three multiple choice questions 
        \begin{itemize}
        \item selected from bank of 11
        \end{itemize}
    \item 302 respondents; 3 fully correct
    \item average 21\%
    \end{itemize}
\end{frame}

\begin{frame}{example survey question}
    \begin{itemize}
    \item `Read phone state and identity' allows which of these?
    \vspace{.5cm}
    \item Read your phone number
    \item See who you have called
    \item Track you across applications
    \item Load adverisements
    \end{itemize}
\end{frame}

\begin{frame}{survey questions (1)}
\includegraphics[width=\textwidth]{../sandbox/android-perm-survey1}
\end{frame}

\begin{frame}{survey questions (2)}
\includegraphics[width=\textwidth]{../sandbox/android-perm-survey2}
\end{frame}

\begin{frame}{survey questions (3)}
\includegraphics[width=\textwidth]{../sandbox/android-perm-survey3}
\end{frame}

\begin{frame}{survey questions (4)}
\includegraphics[width=\textwidth]{../sandbox/android-perm-survey4}
\end{frame}


\subsection{how to ask for permission?}
\begin{frame}
\includegraphics[width=\textwidth]{../sandbox/app-perm-how}
\scriptsize from Felt et al, ``How To Ask For Permission'' (HotSec'12)
\end{frame}

\begin{frame}{principles}
    \begin{itemize}
    \item Felt et al list ``principles'':
    \vspace{.5cm}
    \item ``Conserve user attention, utilizaing it for only permissions that have severe consquences''
        \begin{itemize}
        \item too many security warnings means users won't pay attention
        \end{itemize}
    \item ``When possible, avoid interrupting the user's primary task with explicit security decisions''
        \begin{itemize}
        \item users will dismiss warnings because they get in the way of work
        \end{itemize}
    \end{itemize}
\end{frame}


\subsection{permissions abuse: Cloak and Dagger}
\begin{frame}{Cloak and Dagger}
\includegraphics[width=\textwidth]{../sandbox/cloak-and-dagger}
\end{frame}

\begin{frame}{cloak and dagger permissions}
    \begin{itemize}
    \item the two permissions:
        \begin{itemize}
        \item SYSTEM\_ALERT\_WINDOW: \\
            draw windows on top of screen \\
            (at time: enabled by default)
        \item BIND\_ACCESSIBILITY\_SERVICE: \\
            ``Observe your actions'' \\
            ``Retrieve window content''
        \end{itemize}
    \item can hide window content while user interacts with it
    \item \ldots and stealthy get user to do more things
    \end{itemize}
\end{frame}

\begin{frame}{also, a clickjacking attack}
    \begin{itemize}
    \item at the time, could draw overlay window over permissions dialog
    \item \ldots convince user to press where ``OK'' button is
    \item countermeasure: permissions dialog would detect this, ignore clicks
    \vspace{.5cm}
    \item problem: wouldn't detect if overlay didn't cover enough of button
    \end{itemize}
\end{frame}


\subsection{permissions abuse: information leak}
\begin{frame}{privacy and permissions}
\includegraphics[width=0.7\textwidth]{../sandbox/priv-and-perm}
    \begin{itemize}
    \item 2019 paper
    \item many mobile application permissions related to privacy
    \item getting phone ID, email address, location, \ldots
    \item but applications (especially ad libraries) find workarounds
    \end{itemize}
\end{frame}

\begin{frame}{permissions being insufficient}
    \begin{itemize}
    \item permissions check limited API calls for getting private info,\ldots
    \item \ldots but there were alternative, unfiltered system calls for
    \vspace{.5cm}
    \item getting MAC address (effectively phone ID)
        \begin{itemize}
        \item Linux \texttt{ioctl} system call on socket
        \end{itemize}
    \item WiFi base station address
        \begin{itemize}
        \item ARP cache (recently seen machines on network, to know where to send packets)
        \end{itemize}
    \item location
        \begin{itemize}
        \item geolocation tag on recent photos
        \end{itemize}
    \end{itemize}
\end{frame}

\begin{frame}{covert channels}
    \begin{itemize}
    \item advertising libraries would store phone ID/account info in a file
        \begin{itemize}
        \item \ldots when they had permissions to retrieve it
        \end{itemize}
    \item and would read phone ID/account info from a file
        \begin{itemize}
        \item \ldots when they did not
        \end{itemize}
    \end{itemize}
\end{frame}



\section{backup slides}
\begin{frame}{backup slides}
\end{frame}

\end{document}
