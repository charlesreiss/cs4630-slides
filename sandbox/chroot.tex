\begin{frame}{Unix filesystems and mounting}
    \begin{itemize}
    \item my Linux desktop has two disks:
        \begin{itemize}
        \item \texttt{/} --- an SSD
        \item \texttt{/mnt/extradisk} --- a hard drive
        \end{itemize}
    \item hard drive appears as \textit{subdirectory} of SSD
    \item subdirectory called a \textit{mount point}
    \end{itemize}
\end{frame}

\begin{frame}[fragile,label=perProcessRoot]{per-process root}
    \begin{itemize}
    \item on Unix: each process tracks its own root directory (/)
    \item can be changed with chroot() system call
        \begin{itemize}
        \item command-line tool to access: \texttt{chroot}
        \end{itemize}
    \vspace{.5cm}
    \item usage: can isolate program from other files on system
        \begin{itemize}
        \item example: limit what public file server can access?
        \end{itemize}
    \end{itemize}
\end{frame}

\begin{frame}[fragile,label=lsChrootExample]{chroot ls}
\begin{lstlisting}[language={},style=smaller]
# mkdir /tmp/example
# cp /bin/ls /tmp/example/ls
# chroot /tmp/example /ls
chroot: failed to run command ‘/ls’: No such file or directory
# cp -r /lib64 /tmp/example/lib64
# mkdir -p /tmp/example/lib
# cp -r /lib/x86_64-linux-gnu /tmp/example/lib/x86_64-linux-gnu
# chroot /tmp/example /ls
/ls: error while loading shared libraries: libpcre2-8.so.0: cannot open shared object file: No such file or directory
# cp /usr/lib/x86_64-linux-gnu/libpcre2-8* /tmp/example/lib/x86_64-linux-gnu
# chroot /tmp/example /ls /
lib  lib64  ls
# chroot /tmp/example /ls /..
lib  lib64  ls
# 
\end{lstlisting}
\end{frame}

\begin{frame}{chroot escapes}
    \begin{itemize}
    \item chroot prevents accessing files outside the new \texttt{/}
    \item but root (system adminstrator) user in chroot can access disks, etc.
    \vspace{.5cm}
    \item typical usage: combine chroot with extra user
    \end{itemize}
\end{frame}

\begin{frame}{chroot impracticality}
    \begin{itemize}
    \item some things make chroot impractical in general:
    \vspace{.5cm}
    \item seems like one needs extra copies of most of the system
    \item hard to communicate between separate roots
    \item requires administrator permissions to configure
        \begin{itemize}
        \item dangerous to let normal users configure b/c they could confuse priviliged (set-user-ID) programs like \texttt{sudo}
        \end{itemize}
    \end{itemize}
\end{frame}
