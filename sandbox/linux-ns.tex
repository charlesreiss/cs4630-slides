\begin{frame}{Linux namespaces (1)}
    \begin{itemize}
    \item Linux: alternate sandboxing features
    \item ``namespaces'' for other resources
    \item chroot: each process has own idea of root directory
        \begin{itemize}
        \item change to OS: look up root directory in process, not global variable
        \end{itemize}
    \item can apply this to other resources:
        \begin{itemize}
        \item what filesystems (disks) are available
        \item what network devices are available
        \item what user identifier numbers are
        \item \ldots
        \end{itemize}
    \end{itemize}
\end{frame}

\begin{frame}{Linux namespaces (2)}
    \begin{itemize}
    \item user namespace:
    \vspace{.5cm}
    \item can run programs with new view of users:
    \vspace{.5cm}
    \item inside namespace: running as root
    \item outside namespace: root translated to innocent user ID
    \item allows running programs that expect different users
        \begin{itemize}
        \item \ldots without changes, but without giving special permissions
        \end{itemize}
    \vspace{.5cm}
    \item mechanism: reassign user ID numbers in kernel
    \end{itemize}
\end{frame}

\begin{frame}[fragile,label=LinuxCloneUnshare]{aside: Linux clone(), unshare() syscalls}
\begin{itemize}
\item Linux clone system call: start new process (or thread)
\item flags to specify environment of new process
\item these flags can include ``make a new namespace of a type''
\end{itemize}
\begin{lstlisting}[style=smaller,language=C++]
int id = clone(start_function, ..., CLONE_NEWUSER | other-flags);
\end{lstlisting}
\begin{itemize}
\item above option: new user namespace for new process
\vspace{.5cm}
\item alternative: for changing current process's namespace:
\end{itemize}
\begin{lstlisting}[style=smaller,language=C++]
unshare(CLONE_NEWUSER);
\end{lstlisting}
\end{frame}

\begin{frame}{user namespaces API}
\begin{itemize}
\item Linux: users identified by numerical \textit{user IDs} (UIDs)
\vspace{.5cm}
\item with user namespaces:
\item control file \texttt{/proc/PROCESS-ID/UID\_MAP} contains lines like:
    \begin{itemize}
    \item \texttt{0 1000 2} --- UID 0--1 maps to UID 1000--1001
    \item \texttt{1000 2000 100} --- UID 1000-1100 maps to UID 2000--2100
    \end{itemize}
\item can write to that file to reconfigure (if enough permissions)
\end{itemize}
\end{frame}

\begin{frame}{Linux namesapces (3)}
    \begin{itemize}
    \item mount namespaces:
        \begin{itemize}
        \item Unix: mounting disk = making the contents of the disk available as directories+files
        \end{itemize}
    \vspace{.5cm}
    \item different idea of what filesystems are available
    \item can be setup with \textit{bind mounts} to ``real FS''
        \begin{itemize}
        \item but otherwise: no access to directories outside mount namespace
        \item normally requires root --- but special case with user namespaces
        \end{itemize}
    \end{itemize}
\end{frame}

\begin{frame}[fragile,label=mountNSCmdLine]{mount namespaces API}
from command line:
\begin{lstlisting}[language={},style=smaller]
    # runs shell (/bin/sh) in new mount namesapce
shell1$ unshare --mount /bin/sh

    # setup directories in /tmp/workdir and make them aliases of things on normal FS 
    # these aliases will only exist for processes in mount namespace
shell2$ mkdir -p /tmp/workdir/bin
shell2$ mkdir -p /tmp/workdir/lib
shell2$ mkdir -p /tmp/workdir/usr
shell2$ mkdir -p /tmp/workdir/current
shell2$ mount -o bind,ro /bin /tmp/workdir/bin
shell2$ mount -o bind,ro /lib /tmp/workdir/lib
shell2$ mount -o bind,ro /usr /tmp/workdir/usr
shell2$ mount -o bind /home/someuser /tmp/workdir/current

    # start new shell with the root directory being /tmp/workdir
shell2$ chroot /tmp/workdir /bin/sh
shell3$ cd /
shell3$ /bin/ls
bin     current     lib     usr
\end{lstlisting}
\end{frame}

\begin{frame}{Linux namespaces (3)}
    \begin{itemize}
    \item user namespace and mount namespace together:
    \vspace{.5cm}
    \item run program in new user namespace
    \item map regular root (in namespace) to regular user
        \begin{itemize}
        \item ``opts out'' of programs like sudo
        \end{itemize}
    \item move to new mount namespace
    \item setup bind mounts + chroot
        \begin{itemize}
        \item special case: allowed because root in user namespce
        \item can't get ``real'' root (administrator) privileges ever
        \end{itemize}
    \item run program with subset of available files
    \end{itemize}
\end{frame}

\begin{frame}{Linux namespaces (4)}
    \begin{itemize}
    \item other resources with namespaces
        \begin{itemize}
        \item network --- common usage: virtual network device for set processes
        \item hostname (``UTS'')
        \item process identifiers
        \item control groups (resource limits for memory, CPU usage, disk I/O, etc.)
        \end{itemize}
    \end{itemize}
\end{frame}

\begin{frame}{Linux control groups}
    \begin{itemize}
    \item control groups --- tied to namespaces
    \item primarily: CPU/memory/IO performance restrictions
        \begin{itemize}
        \item primarily intended for `friendly sharing' (containers, etc.)
        \item important for preventing denial-of-service/etc.
        \item not as big a security conern as file/user/etc. access
        \end{itemize}
    \item also mechanism for adding IO device restrictions
    \item also mechanism to start/stop a bunch of processes together
    \end{itemize}
\end{frame}
