\begin{frame}{Linux system call filtering: detailed}
    \begin{itemize}
    \item Linux supports more fine-grained system call filtering
    \item using BPF (Berkeley Packet Filter) programming language
        \begin{itemize}
        \item compiled in the kernel to assembly to check system calls
        \end{itemize}
    \vspace{.5cm}
    \item can check system call argument values, but\ldots
        \begin{itemize}
        \item problems with pointer arguments
        \item too many system calls
        \end{itemize}
    \end{itemize}
\end{frame}

\begin{frame}[fragile,label=open]{Linux system call: open}
\begin{lstlisting}[language=C,style=small]
open("foo.txt", O_RDONLY);
\end{lstlisting}
\begin{itemize}
\item parameters:
    \begin{itemize}
    \item system call number: 2 (``open'')
    \item argument 1: 0x7fffe983 (address of string ``foo.txt'')
    \item argument 2: 0 (value of ``\texttt{O\_RDONLY}'')
    \end{itemize}
\item very problematic to filter using BPF interface
\vspace{.5cm}
\item can deal with using `ptrace' --- Linux debugging interface
    \begin{itemize}
    \item BPF can trigger something like a debugger breakpoint
    \item breakpoint wakes up monitor program (attached like debugger)
    \item `monitor' program can perform system call on program's behalf
    \end{itemize}
\end{itemize}
\end{frame}
