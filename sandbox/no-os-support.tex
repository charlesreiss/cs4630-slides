\begin{frame}{sandboxing without OS support}
    \begin{itemize}
    \item so far: relying on OS features for sandboxing
    \item good reasons:
        \begin{itemize}
        \item primarily want to filter system calls
        \item hardware-assisted, strong protection
        \end{itemize}
    \vspace{.5cm}
    \item but problems with relying on OS:
        \begin{itemize}
        \item sending information in/out of sandbox relatively slow
        \item requires heavily OS-specific code
        \end{itemize}
    \end{itemize}
\end{frame}

\begin{frame}{sandboxing without OS ideas}
    \begin{itemize}
    \item `dynamic' language virtual machine, like Java VM, .Net CLR
        \begin{itemize}
        \item hard to use with code intended to compile to native machine code
        \end{itemize}
    \vspace{.5cm}
    \item virtual machine targetted for C/C++-like code, like WebAssembly
    \vspace{.5cm}
    \item assembly-to-assembly conversion
        \begin{itemize}
        \item example: Wahbe, Lucco, Anderson, and Graham, ``Efficient Software-Based Fault Isolation'' (1993)
        \item example: Ford and Cox, ``Vx32: Lightweight User-level Sandboxing on the x86'' (2008)
        \end{itemize}
    \end{itemize}
\end{frame}
