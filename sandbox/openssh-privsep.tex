\begin{frame}{OpenSSH privilege seperation}
    \begin{itemize}
    \item OpenSSH uses privilege seperation for its SSH server
    \item what runs on the lab machines when you log into them
    \vspace{.5cm}
    \item separate network processing code from authentication code
    \item seperate process per connection --- users don't share
    \end{itemize}
\end{frame}

\begin{frame}{OpenSSH privsep protocol}
    \begin{itemize}
    \item sandboxed process tells ``monitor'' to:
        \vspace{.25cm}
    \item perform \myemph{cryptographic operations}
        \begin{itemize}
        \item long-term keys never in sandboxed process
        \item commands to ask for cryptographic messages they need
        \end{itemize}
    \item ask to switch to user --- if given user password, etc.
        \begin{itemize}
            \item \myemph{monitor process verifies} login information
        \end{itemize}
    \item after authentication: new process running as logged-in user 
        \begin{itemize}
            \item (normally) no issues with special privileges
        \end{itemize}
    \end{itemize}
\end{frame}


\begin{frame}{privilege seperation overall}
    \begin{itemize}
    \item large application changes
        \begin{itemize}
        \item OpenSSH: 3k lines of code for communication/etc. added
        \item OpenSSH: 2\% of existing code (950 of 44k lines) changed
        \item (but most changes simple)
        \end{itemize}
    \item lots of application knowledge
        \begin{itemize}
        \item what is a meaningful separation of `privileged' and `unprivileged'?
        \end{itemize}
    \item better application design anyways?
    \end{itemize}
\end{frame}

