\begin{frame}[fragile]{strace hello\_world (1)}
\begin{lstlisting}[language=C,style=small]
#include <stdio.h>
int main() { puts("Hello, World!"); }
\end{lstlisting}
\hrule
when statically linked:
\begin{Verbatim}[fontsize=\fontsize{10}{11}\selectfont]
execve("./hello_world", ["./hello_world"], 0x7ffeb4127f70 /* 28 vars */)
                                        = 0
brk(NULL)                               = 0x22f8000
brk(0x22f91c0)                          = 0x22f91c0
arch_prctl(ARCH_SET_FS, 0x22f8880)      = 0
uname({sysname="Linux", nodename="reiss-t3620", ...}) = 0
readlink("/proc/self/exe", "/u/cr4bd/spring2023/cs3130/slide"..., 4096)
                                        = 57
brk(0x231a1c0)                          = 0x231a1c0
brk(0x231b000)                          = 0x231b000
access("/etc/ld.so.nohwcap", F_OK)      = -1 ENOENT (No such file or
                                                     directory)
fstat(1, {st_mode=S_IFCHR|0620, st_rdev=makedev(136, 4), ...}) = 0
write(1, "Hello, World!\n", 14)         = 14
exit_group(0)                           = ?
+++ exited with 0 +++
\end{Verbatim}
\end{frame}

\begin{frame}{aside: what are those syscalls?}
\begin{itemize}
\item execve: run program
\item brk: allocate heap space
\item arch\_prctl(ARCH\_SET\_FS, ...): thread local storage pointer
    \begin{itemize}
    \item may make more sense when we cover concurrency/parallelism later
    \end{itemize}
\item uname: get system information
\item readlink of /proc/self/exe: get name of this program
\item access: can we access this file [in this case, a config file]?
\item fstat: get information about open file
\item exit\_group: variant of exit
\end{itemize}
\end{frame}

\begin{frame}[fragile]{only after starting? (1)}
\begin{itemize}
\item okay, but that's only after starting up, right\ldots?
\item surely simpler if we limit system calls after startup
\item yes, but\ldots
\end{itemize}
\end{frame}

\begin{frame}[fragile]{only after starting? (2)}
\begin{lstlisting}[language=C,style=smaller]
#include <stdio.h>
int main() {
    FILE *fh = fopen("output.txt", "w");
    fprintf(fh, "example");
    fclose(fh);
}
\end{lstlisting}
\hrulefill
\begin{Verbatim}[fontsize=\fontsize{9}{10}]
$ strace ...
... [startup stuff, not shown] ...
openat(AT_FDCWD, "output.txt", O_WRONLY|O_CREAT|O_TRUNC, 0666) = 3
newfstatat(3, "", {st_mode=S_IFREG|0664, st_size=0, ...}, AT_EMPTY_PATH) = 0
write(3, "example", 7)                  = 7
close(3)                                = 0
\end{Verbatim}
\end{frame}

\begin{frame}[fragile]{only after starting? (2)}
\begin{lstlisting}[language=C,style=script]
#include <curl/curl.h>
int main() {
    CURL *handle = curl_easy_init();
    curl_easy_setopt(handle, CURLOPT_URL, "https://www.cs.virginia.edu/~cr4bd/test.txt");
    curl_easy_perform(handle);
    ...
}
\end{lstlisting}
\vspace{-.5cm}
\hrulefill
\begin{Verbatim}[fontsize=\fontsize{8}{9}]
$ strace ...
... [startup stuff, not shown] ...
futex(0x73f0bd640ba4, FUTEX_WAKE_PRIVATE, 2147483647) = 0                                               
...
openat(AT_FDCWD, "/usr/lib/ssl/openssl.cnf", O_RDONLY) = 3
...
sysinfo({...}) = 0
...
socket(AF_INET6, SOCK_DGRAM, IPPROTO_IP) = 3
close(3)                                = 0                                                             
socketpair(AF_UNIX, SOCK_STREAM, 0, [3, 4]) = 0                                                         
fcntl(3, F_GETFL)                       = 0x2 (flags O_RDWR)                                            
fcntl(3, F_SETFL, O_RDWR|O_NONBLOCK)    = 0                                                             
...
rt_sigaction(SIGPIPE, NULL, {sa_handler=SIG_DFL, sa_mask=[], sa_flags=0}, 8) = 0                        
...
socket(AF_INET, SOCK_STREAM, IPPROTO_TCP) = 5
setsockopt(5, SOL_TCP, TCP_NODELAY, [1], 4) = 0
...
getrandom("\xd6\x8c\xc3\x42\x07\x92"..., 48, 0) = 48 
...
\end{Verbatim}
\end{frame}

