\begin{frame}[fragile]{symbolic memory?}
\begin{lstlisting}[style=smaller]
int array[10];
void example(int a, int b) {
    assert(a >= 0 && b >= 0 && a < 10 && b < 10);
    int *p = &array[a]; int *q = &array[b];
    *p += 1;
    *q -= 1;
    if (*p == *q) {
        ...
    }
}
\end{lstlisting}
\begin{itemize}
\item question: how to deal with \texttt{*p}, \texttt{*q} accesses?
\end{itemize}
\end{frame}

\begin{frame}[fragile]{lots of branches?}
\begin{lstlisting}[style=smaller]
int array[10];
void example(int a, int b) {
    assert(a >= 0 && b >= 0 && a < 10 && b < 10);
    int *p = &array[a]; int *q = &array[b];
    *p += 1;
    *q -= 1;
    if (*p == *q) {
        ...
    }
}
\end{lstlisting}
\begin{itemize}
\item separate branch for all 10 possible values of x, y
\item gets correct answer --- but 100x extra branches to explore?
\end{itemize}
\end{frame}

\begin{frame}{angr `concretization' strategies}
    \begin{itemize}
    \item angr's design: choose specific set of real addresses for reads and writes
    \item problem: really common to have thousands/millions of possibilities
    \vspace{.5cm}
    \item default policy:
        \begin{itemize}
        \item choose any of range of 1024 possibilities for read
        \item only choose maximum address for write, except for annotated values
        \end{itemize}
    \end{itemize}
\end{frame}

\begin{frame}{higher-level strategies}
    \begin{itemize}
    \item KLEE (2008; OSDI): symbolic execution for bugfinding
    \item works from LLVM bytecode, not assembly
    \vspace{.5cm}
    \item track \textit{memory objects} that pointers point to
        \begin{itemize}
        \item angr can't really do this --- info not available in assembly
        \item really want to have `pointers' that aren't integers
        \end{itemize}
    \item if pointer goes outside intended memory object: found a bug!
        \begin{itemize}
        \item deliberate caveat: find overflow bugs, not \textit{exploits}
        \end{itemize}
    \item if pointer can have multiple objects: fork execution for each one
        \begin{itemize}
        \item like it was an if statement
        \end{itemize}
    \item solve equation for all possible indices in memory object
    \end{itemize}
\end{frame}
