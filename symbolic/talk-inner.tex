\section{symbolic execution}

\begin{frame}{symbolic execution}
    \begin{itemize}
    \item have an emulator/virtual machine
    \item but represent input values as \myemph{symbolic variables}
        \begin{itemize}
            \item like in algebra
        \end{itemize}
    \item choose a path through the program, track \myemph{constraints}
        \begin{itemize}
        \item what values did input need to have to get here?
        \end{itemize}
    \item then solve constraints based on variables to create real test case
        \begin{itemize}
        \item no solution? impossible path
        \item find solution? test case
        \end{itemize}
    \end{itemize}
\end{frame}



% FIXME:
    % really trivial example with single instruction
    % trivial example with angr, one path, symbolic input number
        % diversion into Z3, etc.
    % example with angr, two paths, symbolic input number
    % example with angr, symbolic input number, array lookup
        % diversion into strategies for writing down array lookup
    % example with angr

\subsection{examples}
% FIXME: expand example more
\usetikzlibrary{calc,shapes}
\usetikzlibrary{graphdrawing}
\usetikzlibrary{graphs}
\usegdlibrary{trees}

\begin{frame}[fragile,label=choosePath0]{example 0}
\lstset{language=C,style=smaller}
\begin{lstlisting}
int foo(int a, int b) {
    // (0)
    a += b * 2;
    // (1)
    b *= 4;
    // (2)
    return a + b;
}
\end{lstlisting}
    \begin{tikzpicture}[overlay, remember picture]
        \tikzset{
            every node/.style={font=\scriptsize},
            condition/.style={draw=red,thick,ellipse},
            question/.style={fill=red!20,draw=black,thick,align=left},
            dashedQuestion/.style={fill=red!20,draw=black,dashed,thick,align=left},
            state/.style={draw=blue,thick,rectangle,align=center},
            invisible/.style={opacity=0,text opacity=0},
        }
        \begin{scope}[tree layout,grow=down]
            \node[state,desired at={([xshift=-5cm,yshift=-1.5cm]current page.north east)}] (top) {at (0): \\
                a: $\alpha$, b: $\beta$}
                child {
                    node[state,visible on=<2->] {at (1): \\
                        a: $\alpha + 2\beta$, b: $\beta$
                    } edge from parent[visible on=<2->]
                    child {
                        node[state,visible on=<3->] {at (2): \\
                            a: $\alpha + 2\beta$, b: $4\beta$
                        } edge from parent[visible on=<3->]
                        child {
                            node[state,visible on=<4->] {after func: \\
                            return: $\alpha + 2\beta + 4\beta = \alpha+6\beta$
                            }edge from parent[visible on=<4->]
                        } 
                    } 
                } ;
        \end{scope}
    \end{tikzpicture}
\begin{itemize}
\item<5-> can express return value of function in terms of arguments
\item<5-> then can solve for possible value of arguments
\item<5-> example: if return == 10, then can enumerate:
    \begin{itemize}
    \item (\alpha,\beta) = (10,0)
    \item (\alpha,\beta) = (4,1)
    \item (\alpha,\beta) = (-2,2)
    \item \ldots
    \end{itemize}
\end{itemize}
\end{frame}

\begin{frame}{actually doing this}
\begin{itemize}
\item angr is a binary analysis toolkit written in Python
    \begin{itemize}
    \item has Ghidra-like GUI, but not very stable/maintained as far as I can tell
    \end{itemize}
\item among other things, converts assembly into intermediate form
\item supports symbolic execution
\end{itemize}
\end{frame}

\begin{frame}[fragile]{angr setup}
\begin{Verbatim}[fontsize=\fontsize{10}{11},commandchars=\\\{\}]
import angr
import claripy

p = angr.Project("./example0",
                 load_options={'auto_load_libs': False})

foo_addr = p.loader.main_object.get_symbol('foo').rebased_addr
input_a = claripy.BVS('initial_a', 32) # \textit{32-bit bit vector}
input_b = claripy.BVS('initial_b', 32) # \textit{32-bit bit vector}
init_state = p.factory.call_state(foo_addr, input_a, input_b)
simgr = p.factory.simulation_manager(init_state)
# \textit{<SimulationManager with 1 active>}
\end{Verbatim}
\end{frame}

\begin{frame}[fragile]{angr running}
\begin{Verbatim}[fontsize=\fontsize{10}{11},commandchars=\\\{\}]
print(f"RIP={simgr.active[0].regs.rip} versus {foo_addr:#x}")
    # \textit{RIP=<BV64 0x4011f9> versus 0x4011f9}
print(f"EAX={simgr.active[0].regs.eax}")
    # \textit{RAX=<BV reg_eax_3_32>} (unknown value)
simgr.step()
    # simgr = \textit{<SimulationManager with 1 active>}
simgr.step()
    # simgr = \textit{<SimulationManager with 1 deadended>}
state = simgr.deadended[0]
print(f"EAX={state.regs.eax}")
    # \textit{EAX=initial_a_0_32 + }
    # \textit{     (initial_b_1_32[30:0] .. 0) +}
    # \textit{     (initial_b_1_32[29:0] .. 0)}
state.solver.add(state.regs.eax == 10)
print(state.solver.eval(input_a), state.solver.eval(input_b))
    # \textit{10 0}
state.solver.add(input_b != 0)
print(state.solver.eval(input_a), state.solver.eval(input_b))
    # \textit{4294901754 715838808}
\end{Verbatim}
\end{frame}

\usetikzlibrary{calc,shapes}
\usetikzlibrary{graphdrawing}
\usetikzlibrary{graphs}
\usegdlibrary{trees}

\begin{frame}[fragile,label=choosePath0]{example 1}
\lstset{language=C,style=smaller}
\begin{lstlisting}
void foo(int a, int b) {
  /* (0) */
  if (a != 0) {
    b -= 2;
    a += b;
  }
  /* (1) */
  if (b < 5) {
    b += 4;
  }
  /* (2) */
  if (a + b == 5)
    INTERESTING();
}
\end{lstlisting}
    \begin{tikzpicture}[overlay, remember picture]
        \tikzset{
            every node/.style={font=\scriptsize},
            condition/.style={draw=red,thick,ellipse},
            question/.style={fill=red!20,draw=black,thick,align=left},
            dashedQuestion/.style={fill=red!20,draw=black,dashed,thick,align=left},
            state/.style={draw=blue,thick,rectangle,align=center},
            invisible/.style={opacity=0,text opacity=0},
        }
        \begin{scope}[tree layout,grow=down]
            \node[state,desired at={([xshift=-5cm,yshift=-2cm]current page.north east)}] (top) {at (0): \\ a: $\alpha$, b: $\beta$}
                child { node[condition,visible on=<2->] { a != 0 }
                    child { node[state,visible on=<3->,alt=<4>{very thick}{thick}] {at (1): \\ $\alpha\not=0$\\a: $\alpha+\beta - 2$, b: $\beta - 2$} edge from parent[visible on=<3->] node {true}
                        child { node[condition,visible on=<4->] { b < 5 } edge from parent[visible on=<4->]
                            child {
                                node[state,visible on=<5->] {
                                    at (2): \\ $\alpha\not=0\text{; }\beta-2<5$ \\ a: $\alpha+\beta-2$, b: $\beta + 2$
                                } edge from parent[visible on=<5->] node {true}
                                child {
                                    node[question,visible on=<6->] {
                                        $\alpha \not= 0$; $\beta - 2 < 5$; \\
                                        $\alpha + 2\beta = 5$? \\
                                        \hrulefill
                                        \text{can} happen: $(\alpha,\beta)=(5,0)$
                                    } edge from parent[visible on=<6->]
                                }
                            }
                            child { node[state,visible on=<7->] {at (2): \\ $\alpha\not=0\text{; }\beta-2\ge5$ \\ a: $\alpha+\beta-2$, b: $\beta - 2$}
                                edge from parent[visible on=<7->] node {false}
                                child { node[visible on=<7->,dashedQuestion] {} edge from parent[visible on=<7->] }
                            }
                        }
                    }
                    child { node[state,visible on=<3->] {at (1): \\ $\alpha=0$\\a: $\alpha$, b: $\beta$} edge from parent[visible on=<3->] node {false}
                        child { node[condition,visible on=<8->] { b < 5 } 
                            edge from parent[visible on=<8->]
                            child { node[state,visible on=<8->] {at (2): \\ $a=0\text{; }\beta<5$ \\ a: $\alpha$, b: $\beta+4$} edge from parent[visible on=<8->] node {true}
                                child { node[visible on=<8->,dashedQuestion] {} edge from parent[visible on=<8->] }
                            }
                            child { node[state,visible on=<8->] {at (2): \\ $a=0\text{; }\beta\ge5$ \\ a: $\alpha$, b: $\beta$} edge from parent[visible on=<8->] node {false}
                                child { node[visible on=<8->,dashedQuestion] {} edge from parent[visible on=<8->] }
                            }
                        }
                    }
                }
                ;
        \end{scope}
    \end{tikzpicture}
    \begin{itemize}
        \item<2> every variable represented as an \myemph{equation}
        \item<2> final step: generate solution for each path
            \begin{itemize}
                \item 100\% test coverage
            \end{itemize}
    \end{itemize}
\imagecredit{Adapted from Hicks, ``Symbolic Execution for Finding Bugs''}
\end{frame}

\begin{frame}[fragile]{example 1 in angr}
\begin{Verbatim}[fontsize=\fontsize{10}{11},commandchars=\\\{\}]
p = angr.Project("./example1", load_options={'auto_load_libs': False})

foo_addr = p.loader.main_object.get_symbol('foo').rebased_addr
INTERESTING_addr = p.loader.main_object.get_symbol('INTERESTING').rebased_addr
input_a = claripy.BVS('initial_a', 32)
input_b = claripy.BVS('initial_b', 32)
init_state = p.factory.call_state(foo_addr, input_a, input_b)

simgr = p.factory.simulation_manager(init_state)
print("at beginning:", simgr)
simgr.explore(find=INTERESTING_addr)
print("after explore:", simgr)
for state in simgr.found:
    found_a = state.solver.eval(input_a)
    found_b = state.solver.eval(input_b)
    print(f'(a, b) = ({found_a}, {found_b})')
\end{Verbatim}
\hrule
\begin{Verbatim}[fontsize=\fontsize{10}{11}]
after explore: <SimulationManager with 4 deadended, 4 found>                                             
(a, b) = (0, 1)                                                                                          
(a, b) = (0, 5)                                                                                          
(a, b) = (1, 2)                                                                                          
(a, b) = (9, 2147483648)   
\end{Verbatim}
\end{frame}


\usetikzlibrary{calc,shapes}
\usetikzlibrary{graphdrawing}
\usetikzlibrary{graphs}
\usegdlibrary{trees}


\begin{frame}[fragile,label=choosePath1]{example 2}
\lstset{language=C,style=smaller}
\begin{lstlisting}
void foo(unsigned a,
         unsigned b,
         unsigned c) {
  if (a != 0) {
    b -= c; // W
  }
  if (b < 5) {
    if (b > c) {
      a += b; // X
    }
    b += 4; // Y
  } else {
    a += 1; // Z
  }
  if (a + b != 7)
    INTERESTING();
}
\end{lstlisting}
    \begin{tikzpicture}[overlay, remember picture]
        \tikzset{
            every node/.style={font=\scriptsize},
            condition/.style={draw=red,thick,ellipse},
            question/.style={fill=red!20,draw=black,thick,align=left},
            dashedQuestion/.style={fill=red!20,draw=black,dashed,thick,align=left},
            state/.style={draw=blue,thick,rectangle,align=center},
            dashedState/.style={draw=blue,dashed,thick,rectangle,align=center},
            invisible/.style={opacity=0,text opacity=0},
        }
        \begin{scope}[tree layout,grow=down]
            \node[state,desired at={([xshift=-3cm,yshift=-.5cm]current page.north east)}] (top) {a: $\alpha$, b: $\beta$, c: $\delta$}
                child { node[condition,visible on=<2->] { a != 0 }
                    child { node[state,visible on=<3->,alt=<4>{very thick}{thick}] {$\alpha\not=0$\\a: $\alpha$, b: $\beta - \delta$} edge from parent[visible on=<3->] node {true}
                        child { node[condition,visible on=<4->] { b < 5 } edge from parent[visible on=<4->]
                            child {
                                node[state,visible on=<5->] {
                                    $\alpha\not=0\text{; }\beta-\delta<5$ \\ a: $\alpha$, \\ b: $\beta -\delta$
                                } edge from parent[visible on=<5->] node {true}
                                child {
                                    node[condition,visible on=<6->] {
                                        a > c
                                    }
                                    child {
                                        node[state,visible on=<6->] {
                                            $\alpha\not=0\text{; }\beta-\delta<5\text{; }\beta>\delta$ \\
                                            a: $\alpha+\beta-\delta$, \\ b: $\beta -\delta + 4$
                                        }
                                        child {
                                            node[question,visible on=<7->] {
                                                $\alpha \not= 0$; $\beta- \delta < 5$; $\beta>\delta$ \\
                                                $\alpha + 2\beta- 2\delta + 4= 7$? \\
                                                \hrulefill
                                                \text{can} happen: $(\alpha,\beta,\delta)=(5, 0, 1)$
                                            } edge from parent[visible on=<7->]
                                        }
                                    }
                                    child {
                                        node[state,visible on=<8->] {
                                            $\alpha \not= 0$; $\beta-\delta < 5$; $\beta\le\delta$ \\
                                            a: $\alpha$, \\ b: $\beta - \delta + 4$
                                        } edge from parent[visible on=<8->]
                                        child {
                                            node[dashedQuestion] {}
                                        }
                                    }
                                }
                            }
                            child { node[state,visible on=<9->] {$\alpha\not=0\text{; }\beta-\delta\ge5$ \\ a: $\alpha+\beta-2$, \\ b: $\beta - 2$}
                                edge from parent[visible on=<9->] node {false}
                                child { node[visible on=<9->,dashedQuestion] {} edge from parent[visible on=<7->] }
                            }
                        }
                    }
                    child { node[state,visible on=<3->] {$\alpha=0$\\a: $\alpha$, \\ b: $\beta$} edge from parent[visible on=<3->] node {false}
                        child { node[dashedQuestion] {} }
                        }
                    }
                ;
        \end{scope}
    \end{tikzpicture}
\imagecredit{Adapted from Hicks, ``Symbolic Execution for Finding Bugs''}
\end{frame}




% FIXME:
    % applying to reverse engineering:
        % symbolic execution on assembly
        % example with call to suspicious function

    % challenge we'll talk about later
        % how to model pointers?
        

\subsection{for bounds checking}
\begin{frame}[fragile,label=symToBounds]{using for bounds checking}
\lstset{language=C,style=smaller}
\begin{lstlisting}
void foo() {
    char array[100];
    ...
    /* check inserted automatically: */
        assert(i >= 0 && i < 100);
    array[i] = ...;
    ...
}
\end{lstlisting}
\begin{itemize}
\item using symbolic execution to find memory bugs?
\item add assertions for bounds checks
\item need to track array sizes to do symbolic execution anyways
\end{itemize}
\end{frame}


\subsection{with pointers}
\usetikzlibrary{calc,shapes}
\usetikzlibrary{graphdrawing}
\usetikzlibrary{graphs}
\usegdlibrary{trees}
\begin{frame}[fragile,label=choosePath2]{example 3}
\lstset{language=C,style=smaller}
\begin{lstlisting}
unsigned a, b;
void foo(unsigned c) {
    int *p;
    if (a > 100) {
        p = &a;
    } else {
        p = &b;
    }
    *p += c;
    assert(a + b == c);
}
\end{lstlisting}
\begin{tikzpicture}[overlay,remember picture]
    \tikzset{
            every node/.style={font=\scriptsize},
            condition/.style={draw=red,thick,ellipse},
            question/.style={fill=red!20,draw=black,thick,align=left},
            dashedQuestion/.style={fill=red!20,draw=black,dashed,thick,align=left},
            state/.style={draw=blue,thick,rectangle,align=center},
            invisible/.style={opacity=0,text opacity=0},
        }
        \begin{scope}[tree layout,grow=down]
            \node[state,desired at={([xshift=-5cm,yshift=-1cm]current page.north east)}] (top) {a: $\alpha$, b: $\beta$, c: $\delta$, *p: (error); \\ $\alpha\ge 0$, $\beta \ge 0$\, $\delta \ge 0$ }
                child { node[condition,] { a > 100 }
                    child { node[state] {
                            $\alpha > 100$; $\alpha\ge 0$, $\beta \ge 0$\, $\delta \ge 0$
                            \\ *p: same as a, a: $\alpha$, b: $\beta$} edge from parent[] node {true}
                        child { node[condition,] { *p += c } edge from parent[]
                            child {
                                node[state,] {
                                    $\delta > 100$; $\alpha\ge 0$, $\beta \ge 0$\, $\delta \ge 0$
                                    \\ *p: same as a, a: $\alpha+\delta$, b: $\beta$
                                } edge from parent[] node {}
                                child {
                                    node[question,] {
                                        $\alpha > 100$;
                                        $\alpha+\delta+\beta=\delta$ \\
                                        \hrulefill \\
                                        \text{cannot} happen
                                    } edge from parent[]
                                }
                            }
                        }
                    }
                    child { node[state,] {
                            $\alpha \le 100$; $\alpha\ge 0$, $\beta \ge 0$\, $\delta \ge 0$ 
                            \\ *p: same as b, a: $\alpha+\delta$, b: $\beta$
                        } edge from parent[] node {false}
                        child { node[condition,] { *p += c } 
                            edge from parent[]
                            child { node[state,] {
                                    $\alpha \le 100$; $\alpha\ge 0$, $\beta \ge 0$\, $\delta \ge 0$
                                    \\ *p: same as a, a: $\alpha$, b: $\beta+\delta$
                                } edge from parent[] node {}
                                child { node[,question] {
                                    $\alpha \le 100$; $\alpha\ge 0$, $\beta \ge 0$\, $\delta \ge 0$; \\
                                    $\alpha+\delta+\beta=\delta$ \\
                                    \hrulefill \\
                                    (\alpha,\beta)=0 \\
                                    \delta{} is anything less than 100
                                    } edge from parent[] }
                            }
                        }
                    }
                }
                ;
        \end{scope}
\end{tikzpicture}
\end{frame}



\subsection{exercise}
\begin{frame}[fragile,label=symExecExer]{exercise}
\begin{lstlisting}
void example(unsigned x, unsigned y) {
    if (x > y) return;
    x = x + y;
    assert(x + y + 1 > y);
}
\end{lstlisting}
\begin{itemize}
\item 1: to see if the assertion is meant, the equation we should solve (if initial values of x, y, are X, Y)?
\item 2: what is an input that fails the assertion? (hint: integer overflow)
\end{itemize}
\end{frame}


\subsection{solving equations??}

\begin{frame}{equation solving}
    \begin{itemize}
        \item can generate formula with bounded inputs
        \item can always be solved by trying all possibilities
            \vspace{.5cm}
        \item but actually solving is \myemph{NP-hard (i.e. not generally possible)}
        \item luck: there exists solvers that are \textit{often} good enough
        \item \ldots for small programs
        \item \ldots with lots of additional heuristics to make it work
    \end{itemize}
\end{frame}
 % FIXME: screenshot of Z3
    % FIXME: explanation of SMT solver performance

\subsection{summary: tricky problems with symbolic execution}

\begin{frame}{tricky parts in symbolic execution}
    \begin{itemize}
    \item dealing with pointers?
        \begin{itemize}
        \item one method: one path for each valid value of pointer
        \end{itemize}
    \item solving equations?
        \begin{itemize}
        \item NP-hard (boolean satisfiablity) --- not practical in general
        \item ``good enough'' for small enough programs/inputs
        \item \ldots after lots of tricks
        \end{itemize}
    \item how many paths?
        \begin{itemize}
        \item $<100\%$ coverage in practice
        \item small input sizes (limited number of variables)
        \end{itemize}
    \end{itemize}
\end{frame}




\section{dealing with pointers}
% FIXME: write
\begin{frame}{angr's pointer strategies}
\end{frame}

\section{avoiding state explosion?}
% FIXME: write

\section{symbolic execution for ROP}
% FIXME: write
\begin{frame}{angrop}
\end{frame}

