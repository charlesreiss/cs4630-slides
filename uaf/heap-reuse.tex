\begin{frame}{easy heap reuse}
    \begin{itemize}
    \item strategy of keeping linked list of free items?
    \item simplest way to write code:
        \begin{itemize}
        \item free() = add to head of list
        \item malloc() = scan from head of list
        \end{itemize}
    \item if done, makes it easy to predict what will reuse allocation
    \end{itemize}
\end{frame}

\begin{frame}{complicating easy reuse}
    \begin{itemize}
    \item usually can't precisely control what is allocated/free'd
    \item some allocators mostly use different ordering than last in, first-out
        \begin{itemize}
        \item example: lowest to highest address
        \end{itemize}
    \item often different lists for different size ranges/threads
    \item freeing big object may make space for multiple future allocations
    \end{itemize}
\end{frame}

\begin{frame}{aside: heap feng shui/grooming}
    \begin{itemize}
    \item \url{http://www.phreedom.org/research/heap-feng-shui/heap-feng-shui.html}
    \item one idea:
    \item allocate lots of objects to fill up likely holes
        \begin{itemize}
        \item choose sizes/etc. based on allocator
        \item allocators usually have separate `regions' for different sizes
        \end{itemize}
    \item allocate three objects of appropriate size
        \begin{itemize}
        \item probably three consecutive allocations
        \end{itemize}
    \item free `middle' object + expect it to be reused
    \end{itemize}
\end{frame}
