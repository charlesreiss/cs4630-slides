\begin{frame}{exercise: cavity finding difficulty?}
    \begin{itemize}
    \item exercise: which is least, most difficult kind of cavity
           to \textit{write code} to find
    \vspace{.5cm}
    \item A. gaps between segments
    \item B. space added for alignment between instructions
    \item C. unused space in headers
    \end{itemize}
\end{frame}

\begin{frame}[fragile,label=alignSpace]{exercise: scheme for finding alignment space}
example unused space:
\begin{Verbatim}[fontsize=\fontsize{9}{10}\selectfont,commandchars=Q\{\}]
  404a01:	c3                   	retq   
  Qtextbf{404a02:	0f 1f 40 00          	nopl   0x0(%rax)}
  Qtextbf{404a06:	66 2e 0f 1f 84 00 00 	nopw   %cs:0x0(%rax,%rax,1)}
  Qtextbf{404a0d:	00 00 00 }
\end{Verbatim}
\begin{itemize}
\item suppose a virus scanned the executable for this pattern of bytes (c3 0f 1f \ldots) \\ to find a cavity
\item for compiler generated code, which are likely causes of false positive?
    \begin{itemize}
    \item A. \texttt{c3} might be part of non-ret instruction
    \item B. \texttt{ret} might return to byte immediately after (0x404a02 in example)
    \item C. some other instruction might jump to one of the bytes after
    \item D. the entire sequence could be part of the program's data \\
             (if the virus didn't check that it was in the text section)
    \end{itemize}
\end{itemize}
\end{frame}
