\begin{frame}<1>[fragile,label=runAnyways]{run anyways?}
    \begin{itemize}
    \item add code at start of program (Vienna)
        \begin{itemize}
        \item \myemph<2>{plus restore replaced code after running malware code}
        \end{itemize}
    \item return with padding after it:
\begin{Verbatim}[fontsize=\fontsize{10}{11}\selectfont,commandchars=Q\{\}]
  404a01:       c3                      Qtextbf{retq}
  404a02:       0f 1f 40 00             nopl   0x0(%rax)
                Qtextit{replace with}
  404a01:       e9 XX XX XX XX          Qtextbf{jmpq    YYYYYYY}
\end{Verbatim}
        \begin{itemize}
        \item plus return after running malware code
        \end{itemize}
    \item any random place in program?
        \begin{itemize}
        \item just not in the \myemph{middle of instruction}
        \item and \myemph<2>{replace orignal code after running malware code}
        \end{itemize}
    \end{itemize}
\end{frame}

\begin{frame}[fragile,label=findValidChallenge]{challenge: valid locations}
    \begin{itemize}
    \item x86: probably don't want a full instruction parser
    \item x86: might be non-instruction stuff mixed in with code:
\begin{lstlisting}[language=myasm,style=smaller]
do_some_floating_point_stuff:
            movss float_one(%rip), %xmm0
            ...
            retq
float_one: .float 1
\end{lstlisting}
    \begin{itemize}
        \item floating point value one ({\tt 00 00 80 3f}) is not valid machine code
        \item disassembler might lose track of instruction boundaries
    \end{itemize}
    \end{itemize}
\end{frame}

\begin{frame}[fragile,label=findValidFindFunc]{finding function calls}
    \begin{itemize}
    \item one idea: replace calls
    \item normal x86 call FOO: {\tt E8 \textit{(32-bit value: PC - address of foo)}}
    \item could look for E8 in code --- \myemph{lots of false positives}
        \begin{itemize}
        \item probably even if one excludes out-of-range addresses
        \end{itemize}
    \end{itemize}
\end{frame}

\begin{frame}[fragile,label=findValidFindFunc2]{really finding function calls (1)}
\lstset{language=myasm,style=small}
    \begin{itemize}
    \item e.g. some popular compilers started x86-32 functions with
\begin{lstlisting}
foo:
    push %ebp       // push old frame pointer
    // 0x55
    mov %esp, %ebp  // set frame pointer to stack pointer
    // 0x89 0xec
\end{lstlisting}
    \item use to identify when {\tt e8} refers to real function
    \begin{itemize}
    \item (full version: also have some other function start patterns)
    \end{itemize}
    \end{itemize}
\end{frame}

\begin{frame}{really finding function calls (2)}
    \begin{itemize}
    \item x86-64 assembly seen a lot of ENDBR64 (hex {\tt f3 0f 1e fa})
    \item marker for valid locations to jump to
        \begin{itemize}
        \item intention: part of possible defense against return-oriented-programming-style attacks
        \item (we'll talk about what this means later)
        \end{itemize}
    \item likely only seen at beginning of functions, switch statement cases, etc.
    \end{itemize}
\end{frame}

\againframe<2>{runAnyways}

\begin{frame}{restoring replaced code?}
    \begin{itemize}
    \item Vienna: just write to memory addres
    \vspace{.5cm}
    \item modern OS: segfault/general protection fault
        \begin{itemize}
        \item code loaded read-only
        \end{itemize}
    \item easy solution: make library call to make it writable
        \begin{itemize}
        \item Linux: \texttt{mprotect}
        \item functionality exists to, e.g., allow compiling code at runtime
        \end{itemize}
    \end{itemize}
\end{frame}
