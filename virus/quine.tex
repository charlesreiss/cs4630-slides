
\begin{frame}{quines}
\begin{itemize}
    \item exercise: write a C program that outputs its source code
        \begin{itemize}
        \item (pseudo-code only okay)
        \end{itemize}
    \item possible in any {\small (Turing-complete)} programming language
    \item called a ``quine''
\end{itemize}
\end{frame}

\begin{frame}[fragile,label=quineClever]{clever quine solution}
\lstset{language=C,style=small,
    moredelim={**[is][\btHL<2|handout:0>]{@hi2@}{@endhi@}},
    moredelim={**[is][\btHL<3|handout:0>]{@hi3@}{@endhi@}},
    showstringspaces=false,
}
\begin{lstlisting}
#include <stdio.h>
char*x="int main(){
       `\btHL<2>{printf(p,10,34,x,34,10,34,p,34,10,x,10);}`
       }";
@hi3@char*p@endhi@="#include <stdio.h>%c
    char*x=%c%s%c;%cchar*p=%c%s%c;
    %c%s%c";
int main(){
    @hi2@printf(p,10,34,x,34,10,34,p,34,10,x,10);@endhi@
}
\end{lstlisting}
\begin{itemize}
\item some line wrapping for readability --- shouldn't be in actual quine
\end{itemize}
\begin{tikzpicture}[overlay,remember picture]
    \begin{visibleenv}<2|handout:0>
        \node[fill=white,draw,thick,align=left] at (current page.center) {
            {\tt printf} to fill template: \\
            {\tt 10} = newline; {\tt 34} = double-quote; \\
            {\tt x}, {\tt p} = template/constant strings
        };
    \end{visibleenv}
    \begin{visibleenv}<3|handout:0>
        \node[fill=white,draw,thick,align=left] at ([yshift=1cm]current page.center) {
            template filled by printf
        };
    \end{visibleenv}
\end{tikzpicture}
\end{frame}

\begin{frame}[fragile,label=quineDumb]{dumb quine solution}
\begin{lstlisting}[language=C,style=small]
#include <stdio.h>
int main(void) {
    char buffer[1024];
    FILE *f = fopen("quine.c", "r");
    size_t bytes = fread(buffer, 1,
                         sizeof(buffer), f);
    fwrite(buffer, 1, bytes, stdout);
    return 0;
}
\end{lstlisting}
\begin{itemize}
\item a lot more straightforward!
\item but ``cheating''
\end{itemize}
\end{frame}

