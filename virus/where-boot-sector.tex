\usetikzlibrary{arrows.meta,chains}

\begin{frame}<1-2>[fragile,label=bootProc]{boot process}
    \pdftooltip{
    \begin{tikzpicture}
        \begin{scope}[start chain=going below,every node/.style={draw,align=center,join,on chain,thick,minimum width=3.5cm},
                      every join/.style={-Latex,thick}]
        \node[draw=none] (fixedLoc) { processor reset };
        \node[alt=<3>{draw=orange,very thick}{}] (bios) {BIOS/EFI \\ \small (chip on motherboard)};
        \node[alt=<2>{draw=orange,very thick}{}] (bootloader) {bootloader};
        \node[alt=<4>{draw=orange,very thick}{}] (os) {operating system};
        \end{scope}
        \node[left=.5cm of bios,font=\small] (biosLabel) {very CPU/motherboard-specific code};
        \draw[-Latex,thick] (biosLabel) -- (bios);
        \node[left=.5cm of bootloader,font=\small,align=right] (bootloaderLabel) {fixed location on disk \\ code that understands files};
        \draw[-Latex,thick] (bootloaderLabel) -- (bootloader);
        \node[left=.5cm of os,font=\small] (osLabel) {files in a filesystem};
        \draw[-Latex,thick] (osLabel) -- (os);
    \end{tikzpicture}
    }{
        Boot process, starting from processor reset.
        First, the BIOS of EFI is loadeed from a chip on the motherboard. 
        Then, a bootloader in a fixed location on disk runs, which has the code to figure out
        where individual files are on disk. This finally loads the operating system code
        from its files.
    }
\end{frame}

\begin{frame}{bootloaders in the DOS era}
    \begin{itemize}
    \item used to be common to boot from floppies
    \item \myemph{default to booting from floppy} if present
        \begin{itemize}
        \item even if hard drive to boot from
        \end{itemize}
    \item applications distributed as bootable floppies
    \item so bootloaders on all devices were a target for viruses
    \end{itemize}
\end{frame}

\begin{frame}{historic bootloader layout}
    \begin{itemize}
    \item bootloader in \myemph{first sector} (512 bytes) of device
    \item (along with partition information)
    \item code in BIOS to copy bootloader into RAM, start running
    \item bootloader responsible for disk I/O etc.
        \begin{itemize}
        \item some library-like functionality in BIOS for I/O
        \end{itemize}
    \end{itemize}
\end{frame}

\begin{frame}[fragile]{bootloader viruses}
    \newcommand{\tearBox}{
        \draw[very thick] (0, -6) -- (0, 0) -- (4, 0) -- (4, -6);
        \draw[thick] (0, -6) -- (1.2, -5.5) -- (2., -6.6) -- (2.6, -5.8) -- (3.5, -6.6) -- (4, -6);
    }
    \begin{itemize}
    \item example: Stoned
    \end{itemize}
    \pdftooltip{
    \begin{tikzpicture}
        \draw[fill=green!20] (0, 0) rectangle (4, -.2);
        \draw[fill=yellow!20] (0, -.2) rectangle (4, -0.5) node[midway,font=\tiny] { partition table };
        \draw[fill=green!20] (0, -.5) rectangle (4, -1.0) node[midway,font=\small] { bootloader };
        \tearBox
        \begin{scope}[xshift=7cm]
        \draw[fill=violet!20] (0, 0) rectangle (4, -.2);
        \draw[fill=yellow!20] (0, -.2) rectangle (4, -0.5) node[midway,font=\tiny] { partition table };
        \draw[fill=violet!20] (0, -.5) rectangle (4, -1.0) node[midway,font=\small] { virus code };
        \begin{scope}[yshift=-4cm]
        \draw[fill=green!20] (0, 0) rectangle (4, -.2);
        \draw[fill=green!20] (0, -.5) rectangle (4, -1.0) node[midway,font=\small] { saved bootloader };
        \draw[dashed,fill=yellow!20] (0, -.2) rectangle (4, -0.5) node[midway,font=\tiny] { partition table (unused) };
        \end{scope}
        \tearBox 
        \end{scope}
        \draw[line width=1mm,black!50, -Latex] (4.2, -3) -- (6.8, -3);
        %
        \begin{pgfonlayer}{bg}
            \begin{visibleenv}<2>
                \path[fill=blue!30] (0, -4) rectangle (4, -5) node[midway] { data here??? };
            \end{visibleenv}
        \end{pgfonlayer}
    \end{tikzpicture}
    }{
        The original contents of disk shows a bootloader at the beginning with a partition table
        in the middle of it.
        This is transformed to the bootloader being replaced by the virus code (but the partition table
        is preserved). So the virus can
        run the original bootloader after it finishes, a copy of the original bootloader is saved
        elsewhere on disk
    }
\end{frame}

\begin{frame}{data here???}
    \begin{itemize}
        \item might be data there --- risk
        \item some unused space after partition table/boot loader common
            \begin{itemize}
            \item (allegedly)
            \end{itemize}
        \item also be filesystem metadata not used on smaller floppies/disks
        \vspace{.5cm}
        \item but could be wrong --- oops
    \end{itemize}
\end{frame}

\begin{frame}{modern bootloaders --- UEFI}
    \begin{itemize}
    \item BIOS-based boot is going away (slowly)
    \item new thing: UEFI (Universal Extensible Firmware Interface)
    \item like BIOS:
        \begin{itemize}
        \item library functionality for bootloaders
        \item loads initial code from disk/DVD/etc.
        \end{itemize}
    \item unlike BIOS:
        \begin{itemize}
        \item much more understanding of file systems
        \item much more modern set of library calls
        \end{itemize}
    \end{itemize}
\end{frame}

\againframe<3>{bootProc}

\begin{frame}{BIOS/UEFI implants}
    \begin{itemize}
    \item infrequent
    \item BIOS/UEFI code is \myemph{very non-portable}
    \item BIOS/UEFI update may require physical access
    \item BIOS/UEFI code may require cryptographic signatures
    \item \ldots but \myemph{very hard to remove} --- ``persist'' other malware
    \item reports of BIOS/UEFI-infecting ``implants''
        \begin{itemize}
        \item sold by Hacking Team (Milan-based malware company) 
        \item listed in leaked NSA Tailored Access Group catalog
        \end{itemize}
    \end{itemize}
\end{frame}

\againframe<4>{bootProc}

\begin{frame}{system files}
    \begin{itemize}
    \item simpliest strategy: stuff that runs when you start your computer
    \item add a new startup program, run in the background
        \begin{itemize}
        \item easy to blend in
        \end{itemize}
    \vspace{.5cm}
    \item alternatively, infect one of many system programs automatically run
    \end{itemize}

    % FIXME: example from CodeRed or similar?
\end{frame}

