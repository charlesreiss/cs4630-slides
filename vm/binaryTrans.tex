% From 20170123
\usetikzlibrary{positioning}

\begin{frame}{binary translation}
    \begin{itemize}
    \item compile assembly to new assembly
    \vspace{1cm}
    \item works without instruction set support
    \item early versions of VMWare on x86 (before x86 added virtualisation support)
    \item can be used to run one platform on another
    \end{itemize}
\end{frame}

\begin{frame}[fragile,label=binTransIdea]{binary translation idea}
\lstset{
    language=myasm,
    style=small,
    morekeywords={movq,addss,subss}
}
\begin{tikzpicture}
\tikzset{
    code/.style={inner sep=0mm,align=left},
    hiOn/.style={alt=#1{rounded corners,fill=green,fill opacity=0.3,text opacity=1.0}{}},
    markOn/.style={alt=#1{rounded corners,draw,thick}{}},
}
\node[code,markOn=<2>,hiOn=<3>] (bb1) {
\begin{lstlisting}
0x40FE00: addq %rax, %rbx
movq 14(%r14,4), %rdx
addss %xmm0, (%rdx)
...
0x40FE3A: jne 0x40F404
\end{lstlisting}
};
\node[right=.25cm of bb1,visible on=<2>,align=left] {
    divide machine code \\
    into \textit{basic blocks} \\
    (= ``straight-line'' code) \\
    (= code till \\ 
    jump/call/etc.)
};
\node[code,right=.25cm of bb1,visible on=<3>] (bb1New) {
generated code: \\
\begin{lstlisting}
// addq %rax, %rbx
movq rax_location, %rdi
movq rbx_location, %rsi
call checked_addq
movq %rax, rax_location
...
// jne 0x40F404
... // get CCs 
je do_jne
movq $0x40FE3F, %rdi
jmp translate_and_run
do_jne:
movq $0x40F404, %rdi
jmp translate_and_run
\end{lstlisting}
};
\node[code,markOn=<2>,anchor=north west] (bb2) at (bb1.south west){
\begin{lstlisting}
subss %xmm0, 4(%rdx)
...
je 0x40F543
\end{lstlisting}
};
\node[code,markOn=<2>,anchor=north west] (bb3) at (bb2.south west){
\begin{lstlisting}
ret
\end{lstlisting}
};
\end{tikzpicture}
\end{frame}

\begin{frame}[fragile,label=binTransIdea2]{a binary translation idea}
    \begin{itemize}
    \item convert whole \textit{basic blocks}
        \begin{itemize}
        \item code upto branch/jump/call
        \end{itemize}
    \item end with call to {\tt translate\_and\_run}
        \begin{itemize}
        \item compute new \myemph{simulated PC} address to pass to call
        \end{itemize}
    \end{itemize}
\end{frame}

\begin{frame}[fragile,label=binTransIdea3]{making binary translation fast}
    \lstset{style=small,language=myasm,morekeywords={movq}}
    \begin{itemize}
    \item cache converted code
        \begin{itemize}
        \item {\tt translate\_and\_run} checks cache first
        \end{itemize}
    \item patch calls to {\tt translate\_and\_run} to refer directly to cached code
    \item do something more clever than \lstinline|movq rax_location, ...|
        \begin{itemize}
        \item map (some) registers to registers, not memory
        \end{itemize}
    \item ends up being ``just-in-time'' compiler
    \end{itemize}
\end{frame}


\begin{frame}{binary translation? really?}
    \begin{itemize}
    \item early VMWare: for instructions without hardware virtualization support
        \begin{itemize}
        \item only needed for little bits of OS code
        \end{itemize}
    \item used by Apple to handle changing CPU designs
    \item Rosetta 2: run Intel on ARM (current)
    \item Rosetta: run Power PC on Intel (2005--2011)
    \item Mac 68k emulator: Run Motorola 680x0 on Power PC (1994--2005)
    \end{itemize}
\end{frame}

