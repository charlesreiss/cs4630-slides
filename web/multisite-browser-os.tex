
\begin{frame}{the web}
    \begin{tikzpicture}
        \node[draw,thick,fill=blue!30] (browser) {
            Web Browser
        };
        \node[draw,thick,fill=green!30,right=2cm of browser] (webSite1) {
            facebook.com
        };
        \node[draw,thick,fill=green!30,below=.5cm of webSite1] (webSite2) {
            foobar.com (uses facebook login)
        };
        \node[draw,thick,fill=red!30,below=.5cm of webSite2] (webSite3) {
            evil.com (run by attacker)
        };
        \draw[thick,Latex-Latex] (browser) -- (webSite1.west);
        \draw[thick,Latex-Latex] (browser) -- (webSite2.west);
        \draw[thick,Latex-Latex] (browser) -- (webSite3.west);
    \end{tikzpicture}
\begin{itemize}
    \item one web browser talks to multiple websites
    \item how does it (or does it) keep each websites seperate?
    \item even though websites can link to each other/etc.?
\end{itemize}
\end{frame}

\begin{frame}{the browser is basically an OS}
    \begin{itemize}
    \item websites are JavaScript programs
    \item websites can communicate with each other
        \begin{itemize}
        \item one website can embed another
        \item cause browser to send requests to another
        \end{itemize}
    \item websites can store data on the browser
        \begin{itemize}
        \item cookies
        \item local storage
        \end{itemize}
    \end{itemize}
\end{frame}

