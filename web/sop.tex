
\begin{frame}[fragile,label=webInWeb]{web pages in web pages (1)}
    \vspace{-.25cm}
\begin{minted}[bgcolor=bg,fontsize=\small]{HTML}
<iframe id="localFrame" src="./localsecret.html"
    onload="readLocalSecret()"></iframe>
<script>
function readLocalSecret() {
    alert(document.getElementById('localFrame').
          contentDocument.innerHTML);
}
</script>
\end{minted}
    \begin{itemize}
    \vspace{-.25cm}
    \item displays localsecret.html's \myemph{contents} in an alert box
    \item can also extract specific parts of page
    \item same idea works for sending it to remote server
    \end{itemize}
\end{frame}

\begin{frame}[fragile,label=webInWebOther]{web pages in web pages (2)}
    \vspace{-.25cm}
    \setlength{\parskip}{-0.5\baselineskip}
\begin{minted}[bgcolor=bg,fontsize=\small,highlightlines=2,highlightcolor=green!30!white]{HTML}
<iframe id="remoteFrame"
    src="https://collab.virginia.edu/..."
    onload="readRemoteSecret()></iframe>
<script>
function doIt() {
    alert(document.getElementById('remoteFrame').
          contentDocument.innerHTML);
}
</script>
\end{minted}
    \begin{itemize}
    \item will this work?
    \end{itemize}
\end{frame}

\begin{frame}{what happened?}
    \begin{itemize}
        \item ``TypeError: document.getElementById(...).contentDocument is null''
        \item web browser denied access
        \vspace{.5cm}
    \item \myemph{Same Origin Policy}
    \end{itemize}
\end{frame}


\section{same-origin policy}

\begin{frame}{browser protection}
    \begin{itemize}
        \item websites want to \myemph{load content dynamically}
    \begin{itemize}
        \item Google docs --- send what others are typing
        \item webmail clients autoloading new emails, etc.
        \item \ldots
    \end{itemize}
    \item but shouldn't be able to do so from any other website
        \begin{itemize}
        \item e.g. read grades of Canvas if I'm logged in
        \end{itemize}
    \end{itemize}
\end{frame}

\begin{frame}{same-origin policy}
    \begin{itemize}
        \item two pages from same \myemph{\textbf{origin}}: scripts can do anything
        \item two pages from different \myemph{\textbf{origins}}: almost no information
            \vspace{.5cm}
        \item idea: different websites can't interfere with each other
            \begin{itemize}
            \item facebook can't learn what you do on Google --- unless Google allows it
            \end{itemize}
        \item \myemph{enforced by browser}
    \end{itemize}
\end{frame}

\begin{frame}{origins}
    \begin{itemize}
        \item origin: part of URL up to server name:
            \begin{itemize}
                \item \texttt{\myemph{https://example.com}/foo/bar}
                \item \texttt{\myemph{http://localhost}\tikzmark{local1}/foo/bar}
                \item \texttt{\myemph{http://localhost:8000}\tikzmark{local2}/foo/bar}
                \item \texttt{\myemph{https://www.example.com}/foo/bar}
                \item \texttt{\myemph{http://example.com}/foo/bar}
                \item \texttt{\myemph{https://other.com}/foo/bar}
                \item \texttt{\myemph{file:///}home/cr4bd}
            \end{itemize}
    \end{itemize}
\end{frame}

\againframe<2>{cookieFields}

\begin{frame}{origins and shared servers}
    \begin{itemize}
        \item very hard to safely share a \myemph{domain name}
    \item can never let attacker write scripts on same domain
        \begin{itemize}
        \item even if cookies don't matter
        \end{itemize}
    \item similar issues with plugins (e.g. Flash)
    \vspace{.5cm}
\item can share server --- one server can host \myemph{multiple names}
    \end{itemize}
\end{frame}
