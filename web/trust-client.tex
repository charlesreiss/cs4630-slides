
\begin{frame}[fragile,label=trustCli1]{trusting the client (1)}
\begin{minted}[fontsize=\small]{HTML}
<form action="https://example.com/formmail.pl" method="POST">
<input type="hidden" name="recipient"
       value="webmaster@example.com">
Your email: <input name="from" value=""><br>
Your message: <textarea name="message"></textarea>
...
<input type="submit">
</form>
\end{minted}
    \begin{itemize}
        \item if this my form, can I get a recipient of \texttt{spamtarget@foo.com}?
            \begin{itemize}
                \item Am I \myemph{enabling spammers}??
            \end{itemize}
        \item<2> Yes, because attacker could \myemph{make own version of form}
    \end{itemize}
\end{frame}
\begin{frame}[fragile,label=RefererHeader]{Referer header}
Submitting form at \texttt{https://example.com/feedback.html}:
\begin{framed}
\tt\small
POST /formmail.pl HTTP/1.1 \\
Host: example.com \\
Content-Type: application/x-www-form-urlencoded \\
\myemph{Referer: https://example.com/feedback.html} \\
~ \\
recipient=webmaster@example.com\&from=\ldots \\
\end{framed}
    \begin{itemize}
        \item \textbf{sometimes} sent by web browser
        \item if browser always sends, \myemph{does this help?}
    \end{itemize}
\end{frame}

\begin{frame}[fragile,label=trustCli2]{trusting the client (2)}
\begin{minted}[fontsize=\small]{HTML}
<form action="https://example.com/formmail.pl" method="POST">
<input type="hidden" name="recipient"
       value="webmaster@example.com">
...
<input type="submit">
</form>
\end{minted}
    \begin{itemize}
    \item can I get a recipient of \texttt{spamtarget@example.com} \myemph{and the right referer header}?
        \begin{itemize}
        \item attacker can't modify the form on example.com!
        \item browser sends header with URL of form
        \end{itemize}
    \item<2> Yes, because attacker can \myemph{customize their browser}
    \end{itemize}
\end{frame}

\begin{frame}[fragile,label=trustCliNoOne]{trusting the client (3)}
\setlength{\parskip}{0em}
\fontsize{10}{11}\selectfont\tt
ISS E-Security Alert  \\
February 1, 2000 

Form Tampering Vulnerabilities in Several Web-Based Shopping Cart Applications

\ldots

Many web-based shopping cart applications \myemph{use hidden fields in HTML forms} to
hold parameters for items in an online store. These parameters can include
the item's name, weight, quantity, product ID, and \myemph{price}.\ldots

\ldots

Several of these applications use a security method based on \myemph{the HTTP header}
to verify the request is coming from an appropriate site.\ldots

The ISS X-Force has identified \myemph{eleven shopping cart applications} that are vulnerable to form tampering. \ldots

~
\end{frame}
